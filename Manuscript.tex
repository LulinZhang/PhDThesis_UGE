\documentclass[a4paper,11pt,twoside]{ThesisStyle}

\include{formatAndDefs}
\usepackage{subfigure}
\usepackage{xcolor}

\begin{document}

%!TEX root = Manuscript.tex

\pagenumbering{Alph}

\newgeometry{top=3cm,bottom=3cm,left=3cm,right=3cm}
\begin{titlepage}
\begin{center}
\begin{minipage}[t]{1\linewidth}
	\centering
	\includegraphics[width=15cm]{images/TitlePage/logo.png}
\end{minipage}%
%\begin{minipage}[t]{0.23\linewidth}
%	\centering
%	\includegraphics[width=3.5cm]{images/TitlePage/Lastig.png}
%\end{minipage}%
%\begin{minipage}[t]{0.23\linewidth}
%	\centering
%	\includegraphics[width=3.5cm]{images/TitlePage/Acte.png}
%\end{minipage}%
%\begin{minipage}[t]{0.23\linewidth}
%	\centering
%	\includegraphics[width=3.5cm]{images/TitlePage/Micmac.png}
%\end{minipage}%
%\begin{minipage}[t]{0.23\linewidth}
%	\centering
%	\includegraphics[width=3.5cm]{images/TitlePage/UPE.png}
%\end{minipage}%

	
%\noindent {\large \textbf{Université Gustave Eiffel}} \\
\vspace*{0.3cm}
\noindent {\LARGE \textbf{MSTIC Doctoral School}} \\
\noindent \textbf{Mathematics \& Sciences and Technologies\\of Information and Communication} \\
\vspace*{0.5cm}
\noindent \Huge \textbf{Ph.D Thesis} \\
\vspace*{0.3cm}
\noindent \large {to obtain the title of} \\
\vspace*{0.3cm}
%\noindent \LARGE \textbf{PhD of Science} \\
%\vspace*{0.3cm}
\noindent \Large Doctorate of the Gustave Eiffel University \\
%\noindent \Large \textbf{Specialty : \textsc{Computer Science}}\\
\vspace*{0.4cm}
\noindent \large {Defended by\\}
\noindent \huge Lulin \textsc{Zhang} \\
\vspace*{0.8cm}
\noindent {\Huge \textbf{Feature matching for multi-epoch historical aerial images}} \\
%quasi-nadir
%\noindent {\Huge \textbf{Supervised learning for interest points matching in archive/analogue and modern/digital images}} \\
\vspace*{0.8cm}
\noindent \Large Thesis Advisor: \\
Marc \textsc{Pierrot Deseilligny}, Ewelina \textsc{Rupnik}\\
\vspace*{0.2cm}
\noindent \Large prepared at Univ. Gustave Eiffel/Lastig ACTE/IGN/ENSG\\
\vspace*{0.2cm}
\noindent \large defended on May 20, 2022 \\
\vspace*{0.5cm}
\noindent \large \textbf{President : Livio De Luca} \\
\end{center}

\noindent \large \textbf{Jury :} \\
\begin{center}
%\noindent \large \textbf{Reviewer:} Marie-Odile Berger  INRIA Nancy\\
%\noindent \large \textbf{Reviewer:} El Mustapha Mouaddib\\
%\noindent \large \textbf{Examiner:} Denis Feurer\\
%\noindent \large \textbf{Examiner:} Livio De Luca\\
%\noindent \large \textbf{Invited Member:} Yann Klinger\\
%\noindent \large \textbf{Invited Member:} Michele Santangelo\\
\begin{tabular}{lr}
\textbf{Reviewer:} Marie-Odile Berger & INRIA Nancy\\
\textbf{Reviewer:} El Mustapha Mouaddib & UPJV\\
%Université de Picardie Jules Verne (UPJV)
\textbf{Examiner:} Denis Feurer & IRD\\
%Institut de recherche pour le développement (IRD)
\textbf{Examiner:} Livio De Luca & CNRS\\
\textbf{Invited Member:} Yann Klinger & IPGP\\
\textbf{Invited Member:} Michele Santangelo & CNR-IRPI\\
\end{tabular}

%\begin{tabular}{llcl}
%      \textit{Reviewers :}	& Patrick \textsc{Clarysse}		& - & CNRS (CREATIS)\\
%				& Louis \textsc{Collins}		& - & McGill University\\
%      \textit{Advisor :}	& Grégoire \textsc{Malandain}		& - & INRIA (Asclepios)\\
%      \textit{President :}	& Nicholas \textsc{Ayache}		& - & INRIA (Asclepios)\\
%      \textit{Examinators :}   & Pierre-Yves \textsc{Bondiau}          & - & Centre Antoine Lacassagne (Nice)\\
%      				& Guido \textsc{Gerig}			& - & University of North Carolina\\
%      				& Vincent \textsc{Grégoire}		& - & Université Catholique de Louvain\\
%      \textit{Invited :}		& Hanna \textsc{Kafrouni}		& - & DOSISoft S.A.
%\end{tabular}
\end{center}
\end{titlepage}

\pagenumbering{arabic}
\sloppy

\titlepage

\restoregeometry


\pagenumbering{roman}


\setcounter{page}{0}
\cleardoublepage

\section*{Acknowledgments}
%After nearly 3 years of research, I am about to finish my PhD career. Looking back on these years, tons of feelings well up in my mind, among them there are content, promising, stressful, determined, and most importantly grateful feelings. I am deeply grateful for the people and things I met.\\
%First of all, I would like to thank my supervisors Marc and Ewelina. They offered me this PhD position when I needed a new start in France, and I thank them for their recognition and trust in me. During my PhD, they gave me a lot of guidance and help in my work, I learned a lot from them. Marc is a senior specialist in photogrammetry, he possesses profound experience in this domain, from which I benefited a lot. He is like a steersman, very good at controlling the direction. Ewelina is versatile, she is there for me whenever I have a problem. She is capable of many unexpected skills that surprises me, both in algorithm and operation. More importantly, both of them are rigorous scholars, they set up role models for me in my future research.
%I'd also like to thank Yann Klinger and Institut de Physique du Globe de Paris who made this PhD position possible. 
%%sebastian, KAGAM for providing data
%\\
%Secondly, I would like to thank my husband, Teng Wu, who is also an academic in photogrammetry. He gave me a lot of useful advises in my work. He is also a very reliable partner in life, I got countless help and support from him. Although he is not very expressive, he is fully devoted to me and our family. I know that he will support me unconditionally whenever I need him, he met all the expectations I had for a life partner.\\
%Thanks to the gift of life, we have two healthy and lovely daughters who bring joy and hope into our lives. Since having them, we have a deeper sense of responsibility and a stronger motivation to become better examples for them.\\
%Besides, I am grateful to my parents for raising me and for supporting all the choices I made in my life. Thanks to my brother, who took over the responsibility of taking care of my parents for me after I left China. As an aged PhD student, it is not easy to balance work and life, I can't imagine pursuing my PhD without the support from all the family members I mentioned before.\\
%During my research at Lastig, I met many warm-hearted friends: Manchun, Yilin, Imane, Mohamed, Arthur, Christophe, Jean-Michael, Jean-Philippe, Lanfa, Nathan, Raphael, Evelyn etc., who made my work environment full of friendliness and laughter. I am grateful to all the friends in my life who enriched me and I look forward to meeting and talking with them more often after the covid is alleviated.

\dominitoc
\tableofcontents

%!TEX root = Manuscript.tex

\chapter*{List of Acronyms}
\mtcaddchapter[List of Acronyms]

% Define here acronyms used in the manuscript. Copy paster the example for each new acromnym you would like to use

%\begin{acronym}
%\renewcommand{\\}{}
%\newacronym{DSM}{DSM}{Digital Surface Model}
%%\acro{DTI}{Diffusion Tensor Imaging}
%\end{acronym}
%\acro{DSM}{Digital Surface Model}

\begin{acronym}
	\acro{DSM}[DSM]{Digital Surface Model}
	\acro{DoD}[DoD]{Difference of DSMs}
	\acro{GCP}[GCP]{Ground Control Point}
	\acro{CNN}[CNN]{Convolutional Neural Network}
	\acro{SfM}[SfM]{Structure from Motion}
	\acro{IGN}[IGN]{Institut national de l'information géographique et forestière}
	\acro{GT}[GT]{Ground Truth}
	\acro{BBA}[BBA]{Block Bundle Adjustment}
	\acro{RPC}[RPC]{Rational Polynomial Coefficient}
\end{acronym}

%Abbreviations
%
%IGN
%IPGP
%Lastig
%ENSG
%
%intra-epoch: from the same time
%inter-epoch: from different times
%DSM: Digital Surface Model
%DoD: Difference of DSMs
%GCP: Ground Control Point
%BBA: Bundle block adjustment
%GSD: Ground sampling distance




\mainmatter

Abbreviations

IGN
IPGP
Lastig
ENSG

intra-epoch: from the same time
inter-epoch: from different times
DSM: Digital Surface Model
DoD: Difference of DSMs
GCP: Ground Control Point
BBA: Bundle block adjustment

%!TEX root = Manuscript.tex

\chapter{Introduction}
\label{chap:intro}
\minitoc

\section{Motivation and objectives}
\subsection{Why are historical images interesting}
Historical (i.e., analogue or archival) aerial images play an important role in providing unique information about evolution of land-covers. 
They are valuable assets for a wide range of applications such as (1) analyzing of natural disasters (e.g., earthquake, landslide, volcano, flood, avalanche, etc), (2) eco-environmental monitoring (e.g., forest, atmosphere, glacier, water, coastline, etc), (3) urban expansion and (4) environmental pollution and protection and so on.
\par
Historical aerial images have been regularly acquired since the 1920’s by mapping, military or cadastral agencies all over the world. A mass amount of them have been digitized and made accessible through web services~\cite{sebastien2019archiving,earthexplorer,remonterletemps}. 
For example, according to a survey taken place at the beginning of 2017 in Europe~\cite{sebastien2019archiving}, there are approximately 50 million of aerial images archived in Europe, with around 37.8\% of them digitized. 
The images are of high spatial resolution, and are acquired in stereoscopic configuration, allowing for 3D restitution of territories. 
They are often accompanied by metadata, in most cases including the camera focal length, flight height, scale and the physical sensor size, which are usually saved or mentioned on the films. Other metadata such as flight plans, camera calibration certificates or orientations are not commonly available. 
\par
When the camera calibration parameters are unknown, they should be evaluated by a procedure called self-calibrating bundle adjustment. \ac{GCP}s are required, otherwise inaccurately estimated camera parameters will lead to systematic error surfaces called dome effect (i.e., bowl effect).
Generally, \ac{GCP}s originate from (1) field surveys \cite{micheletti2015application,walstra2004time,cardenal2006use}, (2) recent orthophotos and \ac{DSM} \cite{nurminen2015automation,ellis2006measuring,fox2008unlocking} and (3) recent satellite images \cite{ellis2006measuring,ford2013shoreline}. The most challenging part is to identify the \ac{GCP}s on the historical images, which is not easy due to inevitable scene changes. \ac{GCP}s are usually manually measured with the help of recent photos, however, it is still monotonous and time-consuming. 
There is an urgent need to automatically identify corresponding points (i.e., matches) on historical and recent images.\\
When users are only interested in comparing different historical epochs, the self-calibration can be accomplished without \ac{GCP}s. Matches between different epochs would serve as observations in bundle adjustment to eliminate the systematic errors in surfaces. In conclusion, the bottleneck of historical image self-calibration is recovering matches on images taken at different times (i.e., multi-epoch).

\subsection{How to match multi-epoch historical images}
However, matching multi-epoch historical images remains challenging, despite the fact that there exists a large number of image matching algorithms with their effectiveness proven on modern images. The reasons include:
\begin{enumerate}
	\item Multi-epoch images are often acquired at different times of day and in various weathers and seasons, which unavoidably leading to appearance differences.
	\item The scene changes over time due to anthropogenic phenomena (e.g., urban planning) or natural ones (e.g., earthquake), especially for large time gaps.
	\item Multi-epoch images often exhibit heterogeneous spatial resolutions, accompanied with different acquisition conditions (sensors, spectral channels, etc).
	\item Historical images are often facing low radiometric quality, including low contrast, image noise, deterioration due to the aging of films, or even scratches on the films.
\end{enumerate}
Simply applying \textit{state-of-the-art} feature matching methods (e.g., SIFT~\cite{lowe2004distinctive} or SuperGlue~\cite{sarlin2020superglue}) on multi-epoch image pair often leads to unsatisfactory results. An example is given in Figure~\ref{MultiEpochImgPair}. A pair of multi-epoch images are demonstrated with red rectangles indicating the overlapping area in Figure~\ref{MultiEpochImgPair}(a). The left and right images are taken at the same place in 1954 and 1970 individually. The scene changed significantly, a lot of new buildings arose, the color tones were very different. In Figure~\ref{MultiEpochImgPair}(b-d), the matching result of SIFT, SuperGlue and Ours are displayed for comparison. As can be seen, SIFT failed to find any matches. SuperGlue recovered 369 matches, most of which seem good, but at a closer look the details reveal poor localization precision. Our method found 1463 matches with high accuracy, thanks to the help of (1) 3D geometry and (2) the divide and conquer (i.e., rough-to-precise) strategy, which are elaborated in the following texts.
\begin{figure*}[htbp]
	\begin{center}
		\subfigure[Multi-epoch image pair]{
			\begin{minipage}[t]{0.45\linewidth}
				\centering
				\includegraphics[width=6.2cm]{images/Chapitre1/OIS-Reech_IGNF_PVA_1-0__1954-03-06__C3544-0211_1954_CDP866_0630_OIS-Reech_IGNF_PVA_1-0__1970__C3544-0221_1970_CDP6452_1409.png}
			\end{minipage}%
		}
		\subfigure[Result of SIFT (0 matches)]{
	\begin{minipage}[t]{0.45\linewidth}
		\centering
		\includegraphics[width=6.2cm]{images/Chapitre1/Homol-SIFT_OIS-Reech_IGNF_PVA_1-0__1954-03-06__C3544-0211_1954_CDP866_0630_OIS-Reech_IGNF_PVA_1-0__1970__C3544-0221_1970_CDP6452_1409.png}
	\end{minipage}%
}
		\subfigure[Result of SuperGlue]{
	\begin{minipage}[t]{0.45\linewidth}
		\centering
		\includegraphics[width=6.2cm]{images/Chapitre1/Homol-SuperGlue_OIS-Reech_IGNF_PVA_1-0__1954-03-06__C3544-0211_1954_CDP866_0630_OIS-Reech_IGNF_PVA_1-0__1970__C3544-0221_1970_CDP6452_1409.png}
	\end{minipage}%
}
		\subfigure[Result of Ours]{
	\begin{minipage}[t]{0.45\linewidth}
		\centering
		\includegraphics[width=6.2cm]{images/Chapitre1/Homol-Ours_OIS-Reech_IGNF_PVA_1-0__1954-03-06__C3544-0211_1954_CDP866_0630_OIS-Reech_IGNF_PVA_1-0__1970__C3544-0221_1970_CDP6452_1409.png}
	\end{minipage}%
}
		\caption{(a) A pair of multi-epoch images with red rectangles indicating the overlapping area. (b-d) Matching result of SIFT, SuperGlue and Ours.}
		\label{MultiEpochImgPair}
	\end{center}
\end{figure*}

%\subsubsection{Take advantage of 3D geometry}
\paragraph{Advantages of 3D geometry}
RGB images are widely used for image matching. However, they have the following shortcomings:\\
(1) Their appearances change over time (see Figure~\ref{AppearanceChange}), and over varying view angles on non-Lambertian surfaces (see Figure~\ref{PoorlyTextured}).\\
(2) Self similarities (e.g., repetitive patterns) favor false matches (see Figure~\ref{PoorlyTextured}).\\
Fortunately, 3D geometry such as \ac{DSM} makes up for these shortcomings perfectly. As can be seen in Figure~\ref{AppearanceChange}, the RGB images look very different because the scene changed a lot. However, the corresponding \ac{DSM}s look similar, which is reasonable, as the 3D landscape is more stable over time. Besides, \ac{DSM} is more distinctive than RGB image when it comes to non-Lambertian surfaces and repetitive patterns, as shown in Figure~\ref{PoorlyTextured}. 
Even though 3D geometry lacks textures and details compared to RGB image, it serves as an ideal supplement. Besides, it plays an important role in providing the 3D information for establishing 3D Helmert transformation model between epochs to (1) move different epochs into the same coordinate frame and (2) remove false matches in a RANSAC routine which is more reliable than 2D transformation models.

\begin{figure*}[htbp]
	\begin{center}
		\subfigure[RGB image 1971]{
			\begin{minipage}[t]{0.45\linewidth}
				\centering
				\includegraphics[width=6.2cm]{images/Chapitre1/AppearanceChangeRGBL.png}
			\end{minipage}%
		}
		\subfigure[RGB image 2015]{
			\begin{minipage}[t]{0.45\linewidth}
				\centering
				\includegraphics[width=6.2cm]{images/Chapitre1/AppearanceChangeRGBR.png}
			\end{minipage}%
		}
		\subfigure[\ac{DSM} 1971]{
			\begin{minipage}[t]{0.45\linewidth}
				\centering
				\includegraphics[width=6.2cm]{images/Chapitre1/AppearanceChangeDSML.png}
			\end{minipage}%
		}
		\subfigure[\ac{DSM} 2015]{
			\begin{minipage}[t]{0.45\linewidth}
				\centering
				\includegraphics[width=6.2cm]{images/Chapitre1/AppearanceChangeDSMR.png}
			\end{minipage}%
		}
		\caption{The same zone observed in different times. The RGB images changed a lot while the \ac{DSM}s stayed stable over time.}
		\label{AppearanceChange}
	\end{center}
\end{figure*} 


\begin{figure*}[htbp]
	\begin{center}
		\subfigure[RGB image 1971]{
			\begin{minipage}[t]{0.45\linewidth}
				\centering
				\includegraphics[width=6.2cm]{images/Chapitre1/PoorlyTexturedRGBL.png}
			\end{minipage}%
		}
		\subfigure[RGB image 2015]{
			\begin{minipage}[t]{0.45\linewidth}
				\centering
				\includegraphics[width=6.2cm]{images/Chapitre1/PoorlyTexturedRGBR.png}
			\end{minipage}%
		}
		\subfigure[\ac{DSM} 1971]{
			\begin{minipage}[t]{0.45\linewidth}
				\centering
				\includegraphics[width=6.2cm]{images/Chapitre1/PoorlyTexturedDSML.png}
			\end{minipage}%
		}
		\subfigure[\ac{DSM} 2015]{
			\begin{minipage}[t]{0.45\linewidth}
				\centering
				\includegraphics[width=6.2cm]{images/Chapitre1/PoorlyTexturedDSMR.png}
			\end{minipage}%
		}
		\caption{The same vegetation observed in different times. Non-Lambertian reflection and self similarities present in RGB images, while the \ac{DSM}s stay distinctive.}
		\label{PoorlyTextured}
	\end{center}
\end{figure*} 

%\subsubsection{Rough-to-precise strategy}
\paragraph{Divide and Conquer}
Since the task of recovering robust and precise matches on multi-epoch image pairs is difficult, we divide the task into two sub-tasks and conquer them individually with the rough-to-precise strategy. It is illustrated in Figure~\ref{rough-to-precise}. The two sub-tasks includes:\\
\begin{enumerate}
	\item Rough co-registration, as illustrated in Figure~\ref{rough-to-precise}(b). Its goal is to roughly align the multi-epoch image pairs by focusing on robustness and relaxing the requirement for accuracy.
	\item Precise matching, as illustrated in Figure~\ref{rough-to-precise}(c). It refines the matches predicted by the rough co-registration result by searching only the local neighborhood to reduce ambiguity.
\end{enumerate}

\begin{figure*}[htbp]
	\begin{center}
		\subfigure[Example of an inter-epoch image pair]{
			\begin{minipage}[t]{1\linewidth}
				\centering
				\includegraphics[width=1\columnwidth]{images/Chapitre1/imagepair.png}
			\end{minipage}%
		}
		\subfigure[Rough co-registration]{
			\begin{minipage}[t]{1\linewidth}
				\centering
				\includegraphics[width=0.68\columnwidth]{images/Chapitre1/CoReg.jpg}
			\end{minipage}%
		}
		\subfigure[Precise matching]{
			\begin{minipage}[t]{1\linewidth}
				\centering
				\includegraphics[width=0.55\columnwidth]{images/Chapitre1/Precise.png}
			\end{minipage}%
		}
		\caption{ Rough-to-precise strategy. (a) An example of an inter-epoch image pair to be matched. $I^{e_1}$ and $I^{e_2}$ represents images take at $epoch_1$ and $epoch_2$ individually. (b) Illustration of rough co-registration between $I^{e_1}$ and $I^{e_2}$. As a result, $I^{e_1}$ is roughly aligned with $I^{e_2}$. (c) Illustration of precise matching. For keypoints in $I^{e_1}$ (green cross), a location is predicted in $I^{e_2}$ (purple cross) based on rough co-registration, whose local neighborhood will be searched to find the precise match (yellow cross).}
		\label{rough-to-precise}
	\end{center}
\end{figure*}

\section{Contributions}
\label{sec:contributions}
In this thesis we present rough-to-precise pipelines for matching multi-epoch images. They are suitable for aerial, satellite and mixed images, which open the possibility of geo-referencing millions of historical images without requiring any \ac{GCP}s. 
Six variants are provided for the rough co-registration stage and two variants for the precise matching stage. Each variant has its own characteristic:\\
\begin{enumerate}
	\item For rough co-registration variants: (1) the ones based on the idea of matching \ac{DSM}s generally lead to the most robust results; (2) the ones that match orthophotos could serve as an alternates in rare scenarios of perfectly flat terrain where \ac{DSM}s fail to provide useful information; (3) the others that match original image pairs often lead to less satisfactory results, but they are the only options suitable for terrestrial images.
	\item For precise matching variants: (1) $Patch$ is based on learned matching methods, it generally results in more matches as it is more invariant over time. (2) $Guided$ is based on hand-crafted methods, it is more efficient in terms of the use of memory and CPU resources as it doesn't involve resampling patches, which is necessary for $Patch$. 
\end{enumerate}
\par
Our pipelines aim to unlock the potential of historical images for tracking environmental conditions. 
We are currently collaborating with several institutes to apply our pipelines in different applications, including:
\begin{enumerate}
	\item Institut de Physique du Globe de Paris (IPGP) and Korea Institute of Geoscience and Mineral Resources (KIGAM) for analyzing deformations of the earth crust to understand the seismic events.
	\item National Research Council, Research Institute for Hydrogeological Protection (CNR-IRPI) for analyzing landslide evolution in Italy.
	\item Department of Earth and Environmental Sciences in University of Pavia for analyzing badland evolution in Europe.
\end{enumerate}
\par
We also developed two thorough tutorials accompanied with test datasets to familiarize users with our pipelines implemented in MicMac\cite{HistoPcode} (more details are introduced in Appendix \ref{chap:appendixC}):
\begin{enumerate}
	\item Tutorial of matching aerial images \cite{tuto-aerial} 
	\item Tutorial of matching mixed images (i.e., aerial and satellite images) \cite{tuto-mixed} 
\end{enumerate}

\par
Publications of the author:
\begin{enumerate}
	\item Lulin Zhang, Ewelina Rupnik and Marc Pierrot-Deseilligny. Feature matching for multi-epoch historical aerial images. ISPRS Journal of Photogrammetry and Remote Sensing, 182, 176-189, 2021.
	\item Lulin Zhang, Ewelina Rupnik and Marc Pierrot-Deseilligny.	Guided feature matching for multi-epoch historical image blocks pose estimation. In ISPRS Ann. Photogramm. Remote Sens. Spatial Inf. Sci., 2020.
\end{enumerate}
We also provide video \cite{HistoPVideo}, slides \cite{HistoPSlides} and project website \cite{HistoPProj} to improve the visibility of our work.

\section{Organization of the thesis}
This thesis presents fully automatic pipelines to match multi-epoch images.
A brief presentation of the \textit{state-of-the-art} is given in \textbf{Chapter}~\ref{chap:review}. \\

In \textbf{Chapter}~\ref{chap:ApplicationsAndDatasets}, applications as well as 5 sets of representative datasets are introduced, which are latter used to test our pipelines.\\

In \textbf{Chapter}~\ref{chap:RoughCoReg}, six rough co-registration variants are elaborated to roughly align the whole block by building a globally consistent transformation model between different epochs.\\

In \textbf{Chapter}~\ref{chap:Precisematching}, two precise matching variants are introduced to get accurate matches under the guidance of roughly co-registered orientations and \ac{DSM}s.\\

Finally, in \textbf{Chapter} ~\ref{chap:conclusion} conclusion and perspective are given.\\



%!TEX root = Manuscript.tex

\chapter{Literature review}
\label{chap:review}
\minitoc

\section{Local feature matching}
Local feature refers to a discriminative structure found in an image, such as a point, corner, blob, edge or image patch. It is often accompanied with a descriptor, which is a compact vector representing the local neighborhood.
\par
According to different data storage types, descriptors can be divided into two categories: floating-point and binary descriptors. The former is recorded in floating-point format, which has the advantage of being informative. It is widely used in various matching scenarios.
The latter is stored in binary type, which guarantees faster processing while demanding less memory. It is particularly suitable for real-time and/or smartphone applications.
Since our goal is to match multi-epoch images for high accuracy cartography, we are interested in floating-point descriptors rather than binary ones.
\par
According to whether machine learning techniques are applied, local features can be categorized as hand-crafted or learned. We subsequently elaborate on the two categories of approaches.
\subsection{Hand-crafted methods}
In the early stage, Moravec detects corner feature by measuring the sum-of-squared-differences (SSD) by applying a small shift in a number of directions to the patch around a candidate feature \cite{moravec1980obstacle}. Based on this, Harris computes an approximation to the second derivative of the SSD with respect to the shift \cite{harris1988combined}. Since both Moravec and Harris are sensitive to changes in image scale, algorithms invariant to scale and affine transformations based on Harris are presented \cite{mikolajczyk2004scale}. Other than corner feature, SIFT (Scale-invariant feature transform) \cite{lowe2004distinctive} detects blob feature in scale-space, which is an entire pipeline including detection and description. It uses a difference-of-Gaussian function to identify potential feature points that are invariant to scale and orientation. SIFT is a milestone among hand-crafted features, and is able to outperform machine learning alternatives when matching conditions are favorable. RootSIFT \cite{arandjelovic2012three} uses a square root (Hellinger) kernel instead of the standard Euclidean distance to measure the similarity between SIFT descriptors, which leads to a dramatic performance boost. Similar to SIFT, SURF \cite{bay2006surf} resorts to integral images and Haar filters to extract blob feature in a computationally efficient way. DAISY \cite{tola2009daisy} is a local image descriptor, which uses convolutions of gradients in specific directions with several Gaussian filters to make it very efficient to extract dense descriptors. KAZE \cite{alcantarilla2012kaze} is an algorithm that detects and describes multi-scale 2D feature in nonlinear scale spaces. AKAZE \cite{Alcantarilla13bmvc} is an accelerated version based on KAZE.
\subsection{Learned methods}
With the rise of machine learning, learned features have shown their feasibility in the image matching problem when enough ground truth data is available. 
FAST \cite{rosten2006machine} uses decision tree to speed up the process of finding corner feature. 
LIFT (Learned Invariant Feature Transform) \cite{yi2016lift} is a deep network architecture that implements a full pipeline including detection, orientation estimation and feature description. It is based on the previous work TILDE \cite{verdie2015tilde}, the method of \cite{moo2016learning} and DeepDesc \cite{simo2015discriminative}. 
Tian et al. introduce L2-Net~\cite{tian2017l2} to learn high performance descriptor in Euclidean space via the \ac{CNN}. 
Afterwards Mishchuk et al.~\cite{mishchuk2017working} introduce a compact descriptor named HardNet, by applying a novel loss to L2Net \cite{tian2017l2}. 
DELF \cite{noh2017DELF} is an attentive local feature descriptor based on \ac{CNN}, which works particularly well for illumination changes.
SuperPoint \cite{detone2018superpoint} is a self-supervised, fully-convolutional model that operates on full-sized images and jointly computes pixel-level feature point locations and associated descriptors in one forward pass. 
LF-Net \cite{ono2018lf} is a deep architecture that embeds the entire feature extraction pipeline, and can be trained end-to-end with just a collection of images. 
D2-Net \cite{dusmanu2019d2} is a single neural network that works as a \textit{dense} feature descriptor and a feature detector simultaneously, but their keypoints are less accurate compared to classical features since they are extracted on feature maps which have a resolution of 1/4 of the input image.
ASLFeat~\cite{luo2020aslfeat} improves shape-awareness and localization accuracy by applying light-weight yet effective modifications on improved D2-Net.
R2D2 \cite{revaud2019r2d2} is a \ac{CNN} architecture that learns \textit{dense} local descriptors (one for each pixel) as well as two associated repeatability and reliability confidence maps.
Contextdesc ~\cite{luo2019contextdesc} is a unified learning framework that leverages and aggregates the cross-modality contextual information.
D2D \cite{wiles2020d2d} allows \textit{dense} features to be modified based on the differences between the images by conditioning the feature maps on both images. 
Different than the aforementioned feature extraction methods, SuperGlue \cite{sarlin2020superglue} presents a new way of thinking about the feature matching problem. 
%Instead of learning better task-agnostic local features followed by simple matching heuristics and tricks, SuperGlue
%It learns the matching process from pre-existing local features 
It matches two sets of pre-existing local features by adopting a flexible context aggregation mechanism based on attention to jointly find matches and reject non-matchable points, leading to robust matching results even in challenging situations.
\par
Early learned methods (LIFT ~\cite{yi2016lift}, L2-Net~\cite{tian2017l2}, HardNet~\cite{mishchuk2017working}, DELF ~\cite{noh2017DELF}, SuperPoint ~\cite{detone2018superpoint}, LF-Net ~\cite{ono2018lf}) use only intermediate metrics (e.g., repeatability, matching score, mean matching accuracy, etc) to evaluate the matching performance. 
Even though they demonstrate better performance when compared to hand-crafted features on certain benchmarks, it does not necessarily imply a better performance in terms of subsequent processing steps. For example, in the context of \ac{SfM}, finding additional matches for image pairs where SIFT already provides enough matches does not necessarily result in more accurate or complete reconstructions \cite{schonberger2017comparative}.
Jin et al.~\cite{jin2020image} introduce a comprehensive benchmark for local features and robust estimation algorithms, focusing on the accuracy of the reconstructed camera pose as the primary metric. Using the new metric, SIFT ~\cite{lowe2004distinctive} and SuperGlue ~\cite{sarlin2020superglue} take the lead~\cite{imagematchingchallenge2020}.

\section{Robust matching}
The goal of robust matching is to tell apart true matches (i.e., inliers) from false matches (i.e., outliers), and eliminate the latter from further processing.
\par
Typically, an iterative sampling strategy based on RANSAC (Random Sample Consensus) \cite{fischler1981random} relying on some mathematical model, such as homography \cite{sonka2014image} or essential matrix \cite{sonka2014image} is carried out to remove outliers. 
This is an important issue which was often not given sufficient attention.
LMedS (Least Median of Squares) \cite{leroy1987robust} is a meaningful groundwork before RANSAC, which is also commonly used to replace RANSAC.
MLESAC (Maximum Likelihood SAC)  \cite{torr2000mlesac} adopts the same sampling strategy as RANSAC but chooses the solution that maximizes the likelihood instead of the number of inliers. PROSAC (Progressive Sample Consensus ) \cite{chum2005matching} chooses samples from progressively larger sets of top-ranked matches, which makes it significantly faster than RANSAC. DEGENSAC \cite{chum2005two} is an algorithm for epipolar geometry estimation unaffected by planar degeneracy. It is widely used in the 2020 image matching challenge \cite{imagematchingchallenge2020}.
USAC (Universal RANSAC) \cite{raguram2012usac} framework is a synthesis of the various optimizations and improvements that have been proposed to RANSAC.
GC-RANSAC (Graph-Cut RANSAC) \cite{barath2018graph} runs graph-cut algorithm in the local optimization step.
MAGSAC \cite{barath2019magsac} eliminates the need for a user-defined inlier-outlier threshold with marginalization.
\par
Various deep learning methods have also been developed to handle the erroneous matches.
DSAC (the differentiable counterpart of RANSAC) \cite{brachmann2017dsac} replaces the deterministic hypothesis selection by a probabilistic selection.
CNe (Context Networks) \cite{moo2018learning} trains deep networks in an end-to-end fashion to label the matches as inliers or outliers, known intrinsics are required as input, and a post-processing with RANSAC is often tasked. CNe was embedded into the framework of \cite{jin2020image} to remove outliers, paired with DEGENSAC, PyRANSAC (a variant of DEGENSAC by disabling the degeneracy check, introduced in \cite{jin2020image}) and MAGSAC. The results showed that with SIFT used to train CNe, about 80\% of the outliers were filtered out. Nearly all classical methods benefited from CNe, but not the learned ones. Jin et al. \cite{jin2020image} also state that RANSAC should be tuned to particular feature detector and descriptor, and specific settings should be selected for a particular RANSAC variant.
\par
In this research, we use RANSAC to estimate the 3D Helmert transformation between surfaces (i.e., \ac{DSM}s) calculated in different epochs. Compared to the classical essential/fundamental matrix filtering, with less data we impose stricter rules on the sets of points. Lastly, we eliminate the remaining false matches by looking at their cross-correlation.

\section{Pose estimation}
Pose estimation describes the intrinsic and extrinsic parameters of an image and is classically solved with the \ac{SfM} algorithms \cite{snavely2006photo,deseilligny2011apero,schonberger2016structure} based on local feature matches. 
%If the image pose should be known in a reference coordinate system, i.e., be georeferenced, a 7-parameter transformation followed by \textit{a posteriori} bundle block adjustment must be carried out.
The accuracy of matches plays an important role throughout the \ac{SfM} process, since small inaccuracies in their locations can result in large errors in the estimated poses. 
Good matches will lead to better results on image orientation, camera calibration and 3D reconstruction \cite{lindenberger2021pixel}, \cite{sarlin2021back}, \cite{truong2018second}.
%Sarlin et al. \cite{lindenberger2021pixel}, Lindenberger et al. \cite{lindenberger2021pixel} and Truong Giang et al. \cite{truong2018second} show.

\par
Unlike in modern images where the image coordinate system overlaps with the camera coordinate system, in historical images the overlap is not maintained due to the scanning procedure. To account for this, an additional 2D transformation is estimated in the \ac{SfM} procedure \cite{article}, which puts higher demands on the matches. %Inaccurate features 
\cite{giordano2018toward} demonstrates the importance of good matches in estimating the camera calibration and its great impact on the planimetric and altimetric accuracies of the resulted \ac{DSM}. 
Systematic errors expressed as \textit{dome} effect (i.e., a vertical doming of the surface) could appear in the \ac{DSM}s when camera models are poorly estimated {(i.e., inaccurately estimated lens distortion parameters)} \cite{wackrow2008minimising}, \cite{james2014mitigating}, which restricts the wider use of \ac{DSM}s. 
%It results from inaccurately estimated lens distortion parameters \cite{wackrow2008minimising}, \cite{james2014mitigating}.

%\cite{giordano2018toward} demonstrates the importance of the self-calibration of historical images and its impact on 3D accuracy. Poorly modelled intrinsic parameters result in the known \textit{dome}-like deformations that the authors eliminate with the help of automatic GCPs from exiting orthophotos. 


%The tie-point extraction step is the opening in the photogrammetric
%processing chain and therefore plays a key role in the quality of the subsequent image orientation,
%camera calibration and 3D reconstruction. Improving its precision will have an impact on the
%obtained 3D. In this research work we describe a method which aims at enhancing the accuracy of
%image orientation by adding a second iteration photogrammetric processing.

%\par 

%The uncertain internal geometry can be partially resolved by calibration. Research recent conducted at Loughborough University indicated the potential of consumer grade digital sensors to maintain their internal geometry but also identified residual systematic error surfaces, discernible in digital elevation models (DEM), which are caused by slightly inaccurately estimated lens distortion parameters.


%\par
%Unlike in modern images where the image coordinate system overlaps with the planimetric camera coordinate system, in historical images the overlap is not maintained due to the scanning procedure. To account for this, an additional 2D transformation is estimated in the course of the processing \cite{article}. \cite{giordano2018toward} demonstrates the importance of the self-calibration of historical images and its impact on 3D accuracy. Poorly modelled intrinsic parameters result in the known \textit{dome}-like deformations that the authors eliminate with the help of automatic GCPs from exiting orthophotos. 
%\par 
%Many cases reported in the literature calculate the image poses with \ac{SfM} using features matched exclusively within the same epoch \cite{nurminen2015automation,cardenal2006use,fox2008unlocking,walstra2004time}. In multi-epoch scenarios, the individual epochs should be defined in a common frame, be it the frame of a reference epoch or an absolute reference frame (i.e., a projection coordinate frame). Control points derived from recent orthophotos and DEM \cite{nurminen2015automation,ellis2006measuring,fox2008unlocking} or GPS survey \cite{micheletti2015application,walstra2004time,cardenal2006use} serve to transform the individual epochs to the common frame. Alternatively, a coarse flight plan may provide an approximate co-registration \cite{giordano2018toward}.
%\par  
%To increase relative accuracy between several epochs,  \cite{cardenal2006use,korpela2006geometrically,micheletti2015application} manually insert inter-epoch tie-points. \cite{giordano2018toward} extracts tie-points between analogue images and recent images based on the method of \cite{aubry2014painting}. \cite{feurer2018joining} joins multi-epoch images in a single \ac{SfM} block based on SIFT-like algorithm \cite{lowe2004distinctive,semyonov2011algorithms} by making the assumption that a sufficient number of feature points remain invariant across each time period. Identifying permanent points over a long time-span has not reached a maturity and research in this domain remains insignificant. Most of the mentioned approaches rely on manual effort, auxiliary input or limiting hypothesis.

\section{Historical image processing}
%\subsection{General processing pipeline}
%Archiving and geoprocessing of historical aerial images: current status in europe,official publication no 70. European Spatial Data Research~\cite{sebastien2019archiving}
%\subsection{Inter-epoch historical images alignment}
%When it comes to inter-epoch historical images, however, directly applying SIFT or SuperGlue often results in inferior results due to large radiometric differences.
\zll{Compared to modern digital images, historical images are accompanied with particular characteristics such as poor radiometric quality and deformation during scanning. Therefore, aligning multi-epoch historical images by directly applying \textit{state-of-the-art} feature matching methods often leads to unsatisfactory results. }
In Figure~\ref{MultiEpochImgPair} we showed an example where SIFT and SuperGlue failed to recover good matches on an inter-epoch image pair with drastic scene changes. It is understandable as (1) SIFT is not sufficiently invariant over time, while (2) SuperGlue is not invariant to rotations larger than 45$^\circ$ and it underperforms on larger images because it was presumably trained on small images.\\
Therefore, many previous researches bypassed the task of extracting inter-epoch matches by processing different epochs separately followed by an inter-epoch co-registration relying on \ac{GCP}s.
Between 10 and 169 \ac{GCP}s are required in \cite{pinto2019archived}, ~\cite{bozek2019analysis}, \cite{persia2020archival}, ~\cite{micheletti2015application} and \cite{molg2017structure}.
Manually measuring \ac{GCP}s are laboursome and tedious. Furthermore, it is difficult to find salient points that are stable over time.\\
Certain attempts were made to extract inter-epoch matches. Giordano et al.~\cite{giordano2018toward} extract feature matches between historical and recent images relying on HoG descriptors~\cite{dalal2005histograms}. The authors require flight plans as input, which are not commonly available as mentioned in Section \ref{chap:intro}. 
Feurer et al.~\cite{feurer2018joining} joins multi-epoch images in a single \ac{SfM} block based on SIFT-like algorithm by making the assumption that a sufficient number of feature points remain invariant across each time period. Their methods are widely used in the subsequent works~\cite{filhol2019time}, ~\cite{cook2019simple},~\cite{parente2021automated} and~\cite{blanch2021multi}. It remains questionable whether the method is capable of handling drastic scene changes.
%Zhang et al.~\cite{zhang2020guided} extract inter-epoch matches from SIFT-detected keypoints based on the hypothesis that points follow 2D and 3D spatial similarity model. This method works in simple cases with few scene changes. 
{Additionally, a stream of previous works focus on historical terrestrial images (~\cite{maiwald2021automatic}, ~\cite{beltrami20193d}, ~\cite{bevilacqua2019reconstruction}, ~\cite{maiwald2019generation}) and historical video recordings (~\cite{maiwald2019generation}). However, their algorithms are not suitable to the aerial case.}\\
%This work is an extention of~\cite{zhang2020guided}. Unlike in~\cite{zhang2020guided}, we introduce a rough co-registration between different epochs based on matching \ac{DSM}s with SuperGlue, and use it to guide a precise matching. Our rough co-registration is robust under extreme scene changes because (1) SuperGlue utilizes context to enhance feature descriptors and (2) \ac{DSM}s are generally stable over time. With the guidance of roughly co-registered orientations and \ac{DSM}s, both SIFT and SuperGlue achieved good performance, as shown in our experiments.
In this work, we propose a rough-to-precise strategy to recover inter-epoch matches, without requiring any \ac{GCP}s or auxiliary data.


%!TEX root = Manuscript.tex

\chapter{Rough co-registration}
\label{chap:intro}
\minitoc

\section{Introduction}
\subsection{Motivation}
\textit{state-of-the-art} feature matching methods (SIFT and SuperGlue) fail on inter-epoch image pairs.
\subsection{Contribution}

\section{Methodology}
Our goal is to improve robustness by building globally consistent transformation model over the whole block.
(1) matching each potential image pair followed with global filtering based on 3D RANSAC
(2) get a global image for each epoch first, apply matching and 2D RANSAC
%零碎匹配,整体inlier
%整体匹配

\subsection{Strategy 1: Match image pairs}
%\subsection{Strategy 1: global filtering}
CoReg-R3D
\subsubsection{SIFT}
\subsubsection{SuperGlue}
\subsection{Strategy 2: Match DSM/Orthophoto}
%\subsection{trategy 2: global matching}
CoReg-DSM
CoReg-Ortho
\subsubsection{SIFT}
\subsubsection{SuperGlue}

\section{Experiments}
\subsection{Datasets}
Frejus, Pezenas, Kobe\\
Show piled image of each epoch?
\subsection{Evaluation}
\subsubsection{Quantitative evaluation}
(1)inlier ratio (RANSAC and GT)
(2)Check point diff
(3)DoD (picture and statistics)
\subsubsection{Qualitative evaluation}
inlier tie points with orthophotos as background
\subsection{Comparison}
%分三部分分别展示3组数据的结果
3*2 methods:
%1张表表示6种方法的(1)和(2),1张大图放6种方法的DoD, inlier tie pt图(R3D, DSM, ortho各3张(无点图,SIFT点图,SpG点图))

\section{Conclusion}

\section{Discussion}


%!TEX root = Manuscript.tex

\chapter{Precise matching}
\label{chap:intro}
\minitoc

\section{Introduction}
\subsection{Motivation}
\subsection{Contribution}
check SIFT scale and rotation\\
\textit{one-to-one tiling scheme}\\
3D ransac?

\section{Methodology}

\begin{figure*}[htbp]
	\begin{center}
		\subfigure[Workflow]{
			\begin{minipage}[t]{1\linewidth}
				\centering
				\includegraphics[width=1\columnwidth]{images/Chapitre4/precisematching.png}
			\end{minipage}%
		}
		\subfigure[Patch matching]{
			\begin{minipage}[t]{0.35\linewidth}
				\centering
				\includegraphics[width=4.5cm]{images/Chapitre4/patchmatching.png}
			\end{minipage}%
		}
		\subfigure[Buffer zone of tiles]{
	\begin{minipage}[t]{0.25\linewidth}
		\centering
		\includegraphics[width=3cm]{images/Chapitre4/tilingScheme.png}
	\end{minipage}%
}
		\subfigure[Guided matching]{
	\begin{minipage}[t]{0.35\linewidth}
		\centering
		\includegraphics[width=4.5cm]{images/Chapitre4/guidedmatching.png}
	\end{minipage}%
}
		\caption{(a) Workflow of precise matching. (b) and (d)   illustrate toy-examples of the patch matching and guided matching, respectively, (c) displays the feature correspondences where $\mathbf{K}^{e_1}$ exceeds the original tile size (dark green area) and therefore will be abandoned.}
		\label{WorkflowPatch}
	\end{center}
\end{figure*}

\begin{figure*}[htbp]
	\begin{center}
			\subfigure[Example of an image pair]{
		\begin{minipage}[t]{1\linewidth}
			\centering
			\includegraphics[width=1\columnwidth]{images/Chapitre4/example.png}
		\end{minipage}%
	}
		\subfigure[Example of patch pairs]{
			\begin{minipage}[t]{1\linewidth}
				\centering
				\includegraphics[width=0.5\columnwidth]{images/Chapitre4/patchexample.png}
			\end{minipage}%
		}
		\caption{(a) Example demonstration of an image pair, the master image ($I^{e_1}$) and secondary image ($I^{e_2}$) are taken at Fr{\'e}jus in 1954 and 2014 individually. (b) Patch pairs resulted from (a).}
		\label{WorkflowPatch}
	\end{center}
\end{figure*}

\begin{figure*}[htbp]
	\begin{center}
		\subfigure[Example of keypoint prediction]{
			\begin{minipage}[t]{1\linewidth}
				\centering
				\includegraphics[width=1\columnwidth]{images/Chapitre4/guidedexample.png}
			\end{minipage}%
		}
		\caption{Example demonstration of keypoint prediction (cross symbols) accompanied with search space (circles) on an image pair, the master image ($I^{e_1}$) and secondary image ($I^{e_2}$) are taken at Fr{\'e}jus in 1954 and 2014 individually.}
		\label{WorkflowPatch}
	\end{center}
\end{figure*}



\section{Experiments}

\section{Conclusion}

\section{Discussion}


%!TEX root = Manuscript.tex

\chapter{Conclusion and Perspective}
\label{chap:conclusion}
\minitoc



\appendix

\cleardoublepage
\mtcaddpart[Appendices]
\part*{Appendices}

%!TEX root = Manuscript.tex

\chapter{Comparison between $SIFT_{ours}$ and $SIFT_{orig}$}
\label{chap:appendix1}

\section{Comparison on method $SIFT_{ImgPairs}$}

\begin{figure*}[htbp]
	\scriptsize 
	\begin{center}
		\subfigure[Image pair]{
			\begin{minipage}[t]{0.48\linewidth}
				\centering
				\includegraphics[width=7.5cm]{images/appendix/OIS-Reech_IGNF_PVA_1-0__1971-06-21__C2844-0141_1971_FR2117_1124_15FD3425x00034_02911.png}
			\end{minipage}%
		}
		\subfigure[Match number (\textit{ImgPairs})]{
			\begin{minipage}[t]{0.48\linewidth}
				\centering
				\includegraphics[width=4.8cm]{images/appendix/PlotCurves-SIFTComp_OIS-Reech_IGNF_PVA_1-0__1971-06-21__C2844-0141_1971_FR2117_1124_15FD3425x00034_02911.png}
			\end{minipage}%
		}
		\subfigure[$SIFT_{ours}^{RANSAC Inliers}$]{
			\begin{minipage}[t]{0.48\linewidth}
				\centering
				\includegraphics[width=6cm]{images/appendix/Homol-SIFT2Step_Test-Rough-2DRANSAC_OIS-Reech_IGNF_PVA_1-0__1971-06-21__C2844-0141_1971_FR2117_1124_15FD3425x00034_02911.png}
			\end{minipage}%
		}
		\subfigure[$SIFT_{orig}^{RANSAC Inliers}$]{
			\begin{minipage}[t]{0.48\linewidth}
				\centering
				\includegraphics[width=6cm]{images/appendix/Homol-txt-2DRANSAC_OIS-Reech_IGNF_PVA_1-0__1971-06-21__C2844-0141_1971_FR2117_1124_15FD3425x00034_02911.png}
			\end{minipage}%
		}
		\caption{{\scriptsize Comparison between $SIFT_{ours}$ and $SIFT_{orig}$ on a pair of images from Pezenas 1971 and Pezenas 2015 individually. (a) Image pair to be matched, with red rectangles (\textcolor{red}{red rectangled to be drawn}) indicating the common zone. (b) Numbers of total matches, GT inliers and RANSAC inliers of $SIFT_{ours}$ and $SIFT_{origin}$. (c) Visualization of RANSAC inliers based on $SIFT_{ours}$. (d)Visualization of RANSAC inliers based on $SIFT_{orig}$.}}
		\label{Match result}
	\end{center}
\end{figure*} 

\section{Comparison on method $SIFT_{Ortho}$}
\begin{figure*}[htbp]
	\begin{center}
		\subfigure[Orthophotos]{
			\begin{minipage}[t]{0.48\linewidth}
				\centering
				\includegraphics[width=7.5cm]{images/appendix/Ortho-MEC-Malt_Tapas_1981_Ortho-MEC-Malt_2015.png}
			\end{minipage}%
		}
		\subfigure[Match number (\textit{Ortho})]{
			\begin{minipage}[t]{0.48\linewidth}
				\centering
				\includegraphics[width=4.8cm]{images/appendix/PlotCurves-SIFTComp_Ortho-MEC-Malt_Tapas_1981_Ortho-MEC-Malt_2015.png}
			\end{minipage}%
		}
		\subfigure[$SIFT_{ours}^{RANSAC Inliers}$]{
			\begin{minipage}[t]{0.48\linewidth}
				\centering
				\includegraphics[width=6cm]{images/appendix/Homol-SIFT2Step-Rough-2DRANSAC_Ortho-MEC-Malt_Tapas_1981_Ortho-MEC-Malt_2015.png}
			\end{minipage}%
		}
		\subfigure[$SIFT_{orig}^{RANSAC Inliers}$]{
			\begin{minipage}[t]{0.48\linewidth}
				\centering
				\includegraphics[width=6cm]{images/appendix/Homol-SIFT-2DRANSAC_Ortho-MEC-Malt_Tapas_1981_Ortho-MEC-Malt_2015.png}
			\end{minipage}%
		}
		\caption{Comparison between $SIFT_{ours}$ and $SIFT_{orig}$ on a pair of orthophotos from Pezenas 1981 and Pezenas 2015 individually. (a) Image pair to be matched, with red rectangles indicating the common zone. (b) Numbers of total matches, GT inliers and RANSAC inliers of $SIFT_{ours}$ and $SIFT_{origin}$. (c) Visualization of RANSAC inliers based on $SIFT_{ours}$. (d)Visualization of RANSAC inliers based on $SIFT_{orig}$.}
		\label{Match result}
	\end{center}
\end{figure*} 

\section{Comparison on method $SIFT_{DSM}$}
\begin{figure*}[htbp]
	\begin{center}
		\subfigure[DSMs]{
			\begin{minipage}[t]{0.65\linewidth}
				\centering
				\includegraphics[width=8.8cm]{images/appendix/MEC-Malt_Tapas_1954_MEC-Malt_2014.png}
			\end{minipage}%
		}
		\subfigure[Match number (\textit{DSM})]{
			\begin{minipage}[t]{0.3\linewidth}
				\centering
				\includegraphics[width=4.8cm]{images/appendix/PlotCurves-SIFTComp_MEC-Malt_Tapas_1954_MEC-Malt_2014.png}
			\end{minipage}%
		}
		\subfigure[$SIFT_{ours}^{RANSAC Inliers}$]{
			\begin{minipage}[t]{0.48\linewidth}
				\centering
				\includegraphics[width=6.8cm]{images/appendix/Homol-SIFT2Step-Rough-2DRANSAC_MEC-Malt_Tapas_1954_MEC-Malt_2014.png}
			\end{minipage}%
		}
		\subfigure[$SIFT_{orig}^{RANSAC Inliers}$]{
			\begin{minipage}[t]{0.48\linewidth}
				\centering
				\includegraphics[width=6.8cm]{images/appendix/Homol-SIFT-2DRANSAC_MEC-Malt_Tapas_1954_MEC-Malt_2014.png}
			\end{minipage}%
		}
		\caption{Comparison between $SIFT_{ours}$ and $SIFT_{orig}$ on a pair of DSMs from Fr{\'e}jus 1954 and Fr{\'e}jus 2014 individually. (a) Image pair to be matched, with red rectangles indicating the common zone. (b) Numbers of total matches, GT inliers and RANSAC inliers of $SIFT_{ours}$ and $SIFT_{origin}$. (c) Visualization of RANSAC inliers based on $SIFT_{ours}$. (d)Visualization of RANSAC inliers based on $SIFT_{orig}$.}
		\label{Match result}
	\end{center}
\end{figure*} 


%And I cite myself to show by bibtex style file (two authors)~\cite{Commowick_MICCAI_2007}.
%This for other bibtex stye file : only one author~\cite{Oakes_RStat_1999} and many authors~\cite{Guimond_CVIU_2000}.

\bibliographystyle{ThesisStyleWithEtAl}
\bibliography{Thesis}

% And here comes an example abstract (often required for PhDs, example extracted from my PhD just to show formatting).

\cleardoublepage
\begin{vcenterpage}
%\noindent\rule[2pt]{\textwidth}{0.5pt}
%\begin{center}
%{\large\textbf{Design and Use of Numerical Anatomical Atlases for Radiotherapy\\}}
%\end{center}
%{\large\textbf{Abstract:}}
%The main objective of this thesis is to provide radio-oncology specialists with automatic tools for delineating organs at risk of a patient undergoing a radiotherapy treatment of cerebral or head and neck tumors.
%\\
%To achieve this goal, we use an anatomical atlas, i.e. a representative anatomy associated to a clinical image representing it. The registration of this atlas allows to segment automatically the patient structures and to accelerate this process. Contributions in this method are presented on three axes.
%\\
%First, we want to obtain a registration method which is as independent as possible w.r.t. the setting of its parameters. This setting, done by the clinician, indeed needs to be minimal while guaranteeing a robust result. We therefore propose registration methods allowing to better control the obtained transformation, using outlier rejection techniques or locally affine transformations.
%\\
%The second axis is dedicated to the consideration of structures associated with the presence of the tumor. These structures, not present in the atlas, indeed lead to local errors in the atlas-based segmentation. We therefore propose methods to delineate these structures and take them into account in the registration.
%\\
%Finally, we present the construction of an anatomical atlas of the head and neck region and its evaluation on a database of patients. We show in this part the feasibility of the use of an atlas for this region, as well as a simple method to evaluate the registration methods used to build an atlas.
%\\
%All this research work has been implemented in a commercial software (Imago from DOSIsoft), allowing us to validate our results in clinical conditions.
%\\
%{\large\textbf{Keywords:}}
%Atlas-based Segmentation, non rigid registration, radiotherapy, atlas creation
%\\
%\noindent\rule[2pt]{\textwidth}{0.5pt}
\end{vcenterpage}

\end{document}
