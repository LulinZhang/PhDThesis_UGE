\documentclass[a4paper,11pt,twoside]{ThesisStyle}

\include{formatAndDefs}
\usepackage{geometry}
\usepackage{subfigure}
\usepackage{xcolor}
\usepackage{pifont}
\usepackage{url}
\usepackage{hyperref}
\usepackage[acronym]{glossaries}
\makeglossaries

\usepackage{afterpage}
\newcommand\myemptypage{
	\null
	\thispagestyle{empty}
	\addtocounter{page}{-1}
	\newpage
}
\newcommand{\zll}[1]{\textcolor{black}{#1}}  

\begin{document}

%!TEX root = Manuscript.tex

\pagenumbering{Alph}

\newgeometry{top=3cm,bottom=3cm,left=3cm,right=3cm}
\begin{titlepage}
\begin{center}
\begin{minipage}[t]{1\linewidth}
	\centering
	\includegraphics[width=15cm]{images/TitlePage/logo.png}
\end{minipage}%
%\begin{minipage}[t]{0.23\linewidth}
%	\centering
%	\includegraphics[width=3.5cm]{images/TitlePage/Lastig.png}
%\end{minipage}%
%\begin{minipage}[t]{0.23\linewidth}
%	\centering
%	\includegraphics[width=3.5cm]{images/TitlePage/Acte.png}
%\end{minipage}%
%\begin{minipage}[t]{0.23\linewidth}
%	\centering
%	\includegraphics[width=3.5cm]{images/TitlePage/Micmac.png}
%\end{minipage}%
%\begin{minipage}[t]{0.23\linewidth}
%	\centering
%	\includegraphics[width=3.5cm]{images/TitlePage/UPE.png}
%\end{minipage}%

	
%\noindent {\large \textbf{Université Gustave Eiffel}} \\
\vspace*{0.3cm}
\noindent {\LARGE \textbf{École Doctorale MSTIC}} \\
\noindent \textbf{Math{\'e}matiques \& Sciences et Technologies\\de l’Information et de la Communication} \\
\vspace*{0.5cm}
\noindent \Huge \textbf{THÈSE DE DOCTORAT} \\
\vspace*{0.3cm}
\noindent \large {En vue de l’obtention du grade de :} \\
\vspace*{0.3cm}
%\noindent \LARGE \textbf{PhD of Science} \\
%\vspace*{0.3cm}
\noindent \Large Docteur de l'Université Gustave-Eiffel\\
%\noindent \Large \textbf{Specialty : \textsc{Computer Science}}\\
\vspace*{0.4cm}
\noindent \large {Défendue par\\}
\noindent \huge Lulin \textsc{Zhang} \\
\vspace*{0.8cm}
\noindent {\Huge \textbf{Appariement des caractéristiques pour les images aériennes historiques multi-époques}} \\
%quasi-nadir
%\noindent {\Huge \textbf{Supervised learning for interest points matching in archive/analogue and modern/digital images}} \\
\vspace*{0.8cm}
\noindent \Large Superviseur de thèse: \\
Marc \textsc{Pierrot Deseilligny}, Ewelina \textsc{Rupnik}\\
\vspace*{0.2cm}
\noindent \Large préparé à Univ. Gustave Eiffel/Lastig ACTE/IGN/ENSG\\
\vspace*{0.2cm}
\noindent \large défendue le 20 mai 2022 \\
\vspace*{0.5cm}
\noindent \large \textbf{Président : Livio De Luca}
\end{center}
\noindent \large \textbf{Jury :} \\
\begin{center}
%\noindent \large \textbf{Rapporteur:} Marie-Odile Berger\\
%\noindent \large \textbf{Rapporteur:} El Mustapha Mouaddib\\
%\noindent \large \textbf{Examinateur:} Denis Feurer\\
%\noindent \large \textbf{Examinateur:} Livio De Luca\\
%\noindent \large \textbf{Membre invité:} Yann Klinger\\
%\noindent \large \textbf{Membre invité:} Michele Santangelo\\
\begin{tabular}{lr}
	\textbf{Rapporteur:} Marie-Odile Berger & INRIA Nancy\\
	\textbf{Rapporteur:} El Mustapha Mouaddib & UPJV\\
	%Université de Picardie Jules Verne (UPJV)
	\textbf{Examinateur:} Denis Feurer & IRD\\
	%Institut de recherche pour le développement (IRD)
	\textbf{Examinateur:} Livio De Luca & CNRS\\
	\textbf{Membre invité:} Yann Klinger & IPGP\\
	\textbf{Membre invité:} Michele Santangelo & CNR-IRPI\\
\end{tabular}
%\begin{tabular}{llcl}
%      \textit{Reviewers :}	& Patrick \textsc{Clarysse}		& - & CNRS (CREATIS)\\
%				& Louis \textsc{Collins}		& - & McGill University\\
%      \textit{Advisor :}	& Grégoire \textsc{Malandain}		& - & INRIA (Asclepios)\\
%      \textit{President :}	& Nicholas \textsc{Ayache}		& - & INRIA (Asclepios)\\
%      \textit{Examinators :}   & Pierre-Yves \textsc{Bondiau}          & - & Centre Antoine Lacassagne (Nice)\\
%      				& Guido \textsc{Gerig}			& - & University of North Carolina\\
%      				& Vincent \textsc{Grégoire}		& - & Université Catholique de Louvain\\
%      \textit{Invited :}		& Hanna \textsc{Kafrouni}		& - & DOSISoft S.A.
%\end{tabular}
\end{center}
\end{titlepage}

\pagenumbering{arabic}
\sloppy

\titlepage

\restoregeometry

\myemptypage
%\cleardoublepage
%\newpage
%\clearpage
%!TEX root = Manuscript.tex

\pagenumbering{Alph}

\newgeometry{top=3cm,bottom=3cm,left=3cm,right=3cm}
\begin{titlepage}
\begin{center}
\begin{minipage}[t]{1\linewidth}
	\centering
	\includegraphics[width=15cm]{images/TitlePage/logo.png}
\end{minipage}%
%\begin{minipage}[t]{0.23\linewidth}
%	\centering
%	\includegraphics[width=3.5cm]{images/TitlePage/Lastig.png}
%\end{minipage}%
%\begin{minipage}[t]{0.23\linewidth}
%	\centering
%	\includegraphics[width=3.5cm]{images/TitlePage/Acte.png}
%\end{minipage}%
%\begin{minipage}[t]{0.23\linewidth}
%	\centering
%	\includegraphics[width=3.5cm]{images/TitlePage/Micmac.png}
%\end{minipage}%
%\begin{minipage}[t]{0.23\linewidth}
%	\centering
%	\includegraphics[width=3.5cm]{images/TitlePage/UPE.png}
%\end{minipage}%

	
%\noindent {\large \textbf{Université Gustave Eiffel}} \\
\vspace*{0.3cm}
\noindent {\LARGE \textbf{MSTIC Doctoral School}} \\
\noindent \textbf{Mathematics \& Sciences and Technologies\\of Information and Communication} \\
\vspace*{0.5cm}
\noindent \Huge \textbf{Ph.D Thesis} \\
\vspace*{0.3cm}
\noindent \large {to obtain the title of} \\
\vspace*{0.3cm}
%\noindent \LARGE \textbf{PhD of Science} \\
%\vspace*{0.3cm}
\noindent \Large Doctorate of the Gustave Eiffel University \\
%\noindent \Large \textbf{Specialty : \textsc{Computer Science}}\\
\vspace*{0.4cm}
\noindent \large {Defended by\\}
\noindent \huge Lulin \textsc{Zhang} \\
\vspace*{0.8cm}
\noindent {\Huge \textbf{Feature matching for multi-epoch historical aerial images}} \\
%quasi-nadir
%\noindent {\Huge \textbf{Supervised learning for interest points matching in archive/analogue and modern/digital images}} \\
\vspace*{0.8cm}
\noindent \Large Thesis Advisor: \\
Marc \textsc{Pierrot Deseilligny}, Ewelina \textsc{Rupnik}\\
\vspace*{0.2cm}
\noindent \Large prepared at Univ. Gustave Eiffel/Lastig ACTE/IGN/ENSG\\
\vspace*{0.2cm}
\noindent \large defended on May 20, 2022 \\
\vspace*{0.5cm}
\noindent \large \textbf{President : Livio De Luca} \\
\end{center}

\noindent \large \textbf{Jury :} \\
\begin{center}
%\noindent \large \textbf{Reviewer:} Marie-Odile Berger  INRIA Nancy\\
%\noindent \large \textbf{Reviewer:} El Mustapha Mouaddib\\
%\noindent \large \textbf{Examiner:} Denis Feurer\\
%\noindent \large \textbf{Examiner:} Livio De Luca\\
%\noindent \large \textbf{Invited Member:} Yann Klinger\\
%\noindent \large \textbf{Invited Member:} Michele Santangelo\\
\begin{tabular}{lr}
\textbf{Reviewer:} Marie-Odile Berger & INRIA Nancy\\
\textbf{Reviewer:} El Mustapha Mouaddib & UPJV\\
%Université de Picardie Jules Verne (UPJV)
\textbf{Examiner:} Denis Feurer & IRD\\
%Institut de recherche pour le développement (IRD)
\textbf{Examiner:} Livio De Luca & CNRS\\
\textbf{Invited Member:} Yann Klinger & IPGP\\
\textbf{Invited Member:} Michele Santangelo & CNR-IRPI\\
\end{tabular}

%\begin{tabular}{llcl}
%      \textit{Reviewers :}	& Patrick \textsc{Clarysse}		& - & CNRS (CREATIS)\\
%				& Louis \textsc{Collins}		& - & McGill University\\
%      \textit{Advisor :}	& Grégoire \textsc{Malandain}		& - & INRIA (Asclepios)\\
%      \textit{President :}	& Nicholas \textsc{Ayache}		& - & INRIA (Asclepios)\\
%      \textit{Examinators :}   & Pierre-Yves \textsc{Bondiau}          & - & Centre Antoine Lacassagne (Nice)\\
%      				& Guido \textsc{Gerig}			& - & University of North Carolina\\
%      				& Vincent \textsc{Grégoire}		& - & Université Catholique de Louvain\\
%      \textit{Invited :}		& Hanna \textsc{Kafrouni}		& - & DOSISoft S.A.
%\end{tabular}
\end{center}
\end{titlepage}

\pagenumbering{arabic}
\sloppy

\titlepage

\restoregeometry


\pagenumbering{roman}


\setcounter{page}{0}
\cleardoublepage
\section*{Résumé}
L'imagerie historique se caractérise par une haute résolution spatiale et des acquisitions stéréoscopiques. Elle constitue une ressource précieuse pour la détection des changements et la surveillance environnementale à long terme. Des millions d'images historiques ont été numérisées. Elles sont des témoins objectifs du temps et parfois la seule source visuelle restante de la forme historique du territoire. Cependant, l'énorme potentiel des images historiques diachroniques est supprimé en raison du goulot d'étranglement que constitue leur géoréférencement précis. 
Il s'agit d'un processus appelé ajustement de faisceau auto-calibré pour estimer les paramètres de calibrage de la caméra. Il faut un nombre suffisant de correspondances dans des paysages évolutifs, qui sont difficiles à obtenir automatiquement, en raison des changements de scène et des conditions hétérogènes d'acquisition des images.\\

Dans cette recherche, nous présentons des pipelines entièrement automatiques pour trouver des correspondances entre des images historiques prises à différents temps (c'est-à-dire, inter-époques), sans données auxiliaires nécessaires. 
En profitant de la géométrie 3D et de la stratégie grossier-à-précis, nous (1) enregistrons grossièrement les différentes époques en établissant un modèle de transformation globalement cohérent sur l'ensemble du bloc, et (2) nous apparions précisément les images inter-époques sous la direction du co-enregistrement grossier pour réduire l'ambiguïté. Six variantes de deux stratégies sont explorées pour l'étape de co-enregistrement grossier, et deux variantes pour l'étape d'appariement précis. 
Nos pipelines sont adaptés à diverses applications de surveillance environnementale. Cinq données représentatifs sont choisis pour les expériences, chacun représentant une application caractéristique. 
Avec les correspondances inter-époques récupérées, nous améliorons les orientations de l'image puis calculons les Digital Surface Models (DSMs) à chaque époque, et évaluons quantitativement les résultats avec les Difference of DSMs (DoDs) et le déplacement du sol dû à un séisme. Nous démontrons que notre méthode (1) peut géoréférencer automatiquement des images historiques diachroniques ; (2) peut atténuer efficacement les erreurs systématiques induites par des paramètres de caméra mal estimés ; et (3) est robuste contre les changements drastiques de la scène. Les pipelines proposés sont mis en œuvre dans MicMac, un logiciel de photogrammétrie libre et gratuit.\\

\paragraph{Mots clefs:} Appariement des caractéristiques, Images historiques, Multi-époques, Estimation de la pose, Auto-étalonnage

\cleardoublepage

\section*{Abstract}
Historical imagery is characterized by high spatial resolution and stereoscopic acquisitions, providing a valuable resource for change detection and long-term environmental monitoring. Millions of historical images have been digitized. They are objective witness over time and sometimes the only remaining visual source of historical land-form. However, the huge potential of diachronic historical images is suppressed due to the bottleneck of their accurate geo-referencing.
%tracking environmental conditions such as the impact of natural disasters and urban expansion, finding changes in vegetation, evaluating desertification and detecting specific urban or natural variations in the environment. 
%\par
%Accurate geo-referencing of historical images is crucial in unlocking their huge potential for comparing 3D land-cover information over time. 
It involves a process called self-calibrating bundle adjustment to estimate the camera calibration parameters. Sufficient amount of matches under evolving landscapes are required, which are difficult to be obtained automatically, due to scene changes and heterogeneous image acquisition conditions.\\
%{Accurate geo-referencing of diachronic historical images by means of self-calibration} remains a bottleneck because of the difficulty to find sufficient amount of matches under evolving landscapes. 

In this research, we present fully automatic pipelines to finding matches between historical images taken at different times (i.e., inter-epoch), without auxiliary data required. 
By taking advantage of 3D geometry and rough-to-precise strategy, we (1) roughly co-register different epochs by establishing a globally consistent transformation model over the whole block, and (2) precisely match inter-epoch images under the guidance of rough co-registration to reduce ambiguity. Six variants out of 2 strategies are explored for rough co-registration stage, and two variants for precise matching stage. 
Our pipelines are suitable for diverse applications of environmental monitoring. Five representative sets of datasets are chosen for experiments, each one represents a characteristic application. 
With the recovered inter-epoch matches, we refine the image orientations followed by calculating \ac{DSM}s in each epoch, and quantitatively evaluate the results with \ac{DoD}s and ground displacement due to an earthquake. We demonstrate that our method (1) can automatically geo-reference diachronic historical images; (2) can effectively mitigate systematic errors induced by poorly estimated camera parameters; and (3) is robust against drastic scene changes. 
%Compared to the \textit{state-of-the-art}, our method improves the image georeferencing accuracy by a factor of 2. 
The proposed pipelines are implemented in MicMac, a free, open-source photogrammetric software.\\

%Recovered inter-epoch matches are visualized as qualitative and Quantitative results are demonstrated 

%Based on relative orientations computed within the same epoch (i.e., intra-epoch), we obtain \ac{DSM}s and incorporate them in a rough-to-precise matching.
%The {method consists} of: (1) an inter-epoch \ac{DSM}s matching {to roughly co-register the orientations and \ac{DSM}s} (i.e, the {3D} Helmert transformation), followed by (2) a precise inter-epoch feature matching using the original {RGB} images. The innate ambiguity of the latter is largely alleviated by narrowing down the search space using the co-registered {data}.
%With the inter-epoch features, we refine the image orientations and quantitatively evaluate the results (1) with DoD ({\textit{Difference of \ac{DSM}s}}), (2) with ground check points, and (3) by quantifying ground displacement due to an earthquake. We demonstrate that our method: (1) can automatically georeference diachronic historical images; (2) can effectively mitigate systematic errors induced by poorly estimated camera parameters; (3) is robust to drastic scene changes. Compared to the \textit{state-of-the-art}, our method improves the image georeferencing accuracy by a factor of 2. {The proposed methods are implemented in MicMac, a free, open-source photogrammetric software.}\\
\paragraph{Keywords:} Feature matching, Historical images, Multi-epoch, Pose estimation, Self-calibration

\cleardoublepage

\section*{Acknowledgments}
I recollect the beginning of my PhD career as if it was yesterday. 
I resigned from a job that lasted 6 years in China and came to Paris in search of a new start, hoping to find a more research-oriented position in photogrammetry and computer vision. 
I still remember how excited I was when I received the offer of this PhD position. 
Now that I'm approaching the end of this meaningful journey, I can't help but feel how quickly time flies. 
Looking back on the past three years, mixed feelings well up in my mind, among them there are pleasure and delight, as well as regret, and most importantly gratitude. I am deeply grateful for the people I met.\\

First of all, sincere thanks go to my supervisors Marc Pierrot-Deseilligny and Ewelina Rupnik. They offered me this precious opportunity which met all my needs, and I thank them from the bottom of my heart for their recognition and trust in me. During my PhD research, they have given me a lot of guidance and help in my work, I learned a lot from them. Marc is a senior specialist in photogrammetry. He possesses profound experience in this domain, from which I benefited a lot. He is a first-rate helmsman who is good at controlling direction with ease. Ewelina is versatile, she is always there for me whenever I have a problem. She is capable of many unexpected skills that surprises me, both in theory and application. 
More importantly, both of them are rigorous scholars, they set up role models for me in my future research.
Besides, Ewelina is an exemplary feminism to me and sets an instance for women pursuing their careers.\\

I would also like to express my special gratitude to Yann Klinger and Arthur Delorme in Institut de Physique du Globe de Paris (IPGP), Michele Santangelo in National Research Council, Research Institute for Hydrogeological Protection (CNR-IRPI), Han-Kyung Bae and Dong-Eun Kim in Korea Institute of Geoscience and Mineral Resources (KIGAM), Joaquín Muñoz-Cobo Belart in National Land Survey of Iceland. Thanks to all of them, I learned a lot about the possible applications for my research. \\

Heartfelt acknowledgment is made to Denis Feurer and Fabrice Vinatier for their contribution to the interior calibration of the Pezenas dataset, Benjamin Ferrand for the 2015 acquisition of Pezenas, Sebastien Giordano for the Fr{\'e}jus dataset, Han-Kyung Bae and Dong-Eun Kim for the Kobe dataset, Michele Santangelo for the Alberona dataset, and Joaquín Muñoz-Cobo Belart for the Hofsjökull dataset. We are thankful to ANR project DISRUPT (ANR-18-CE31-0012–0) for supporting this work.\\

In particular, I deeply appreciate my husband, Teng Wu, who is also an academic in photogrammetry. He gave me a lot of useful advises in my work and life. He is also a very reliable partner, I got countless help and support from him. Although he is not very expressive, he is fully devoted to me and our family. I know that he will support me unconditionally whenever I need him, he met all the expectations I had for a life partner. Thanks to the gift of life, we have two healthy and lovely children who brought hope and joy into our lives. Since having them, we've got a deeper sense of responsibility and a stronger motivation to become better examples for them. With a great sense of debt, I am grateful to my parents for raising me and supporting all the choices I made in my life. Thanks to my brother, who took over the responsibility of taking care of my parents for me after I left China. As an aged PhD student, it is not easy to balance work and life, I can't imagine pursuing my PhD without the support from all the family members I mentioned before.\\

Last but not least, during my research at Lastig, I met many warm-hearted friends: Manchun Lei, Yilin Zhou, Imane Fikri, Mohamed Ali Chebbi, Christophe Meynard, Jean-Michael Muller, Jean-Philippe Souchon, Olivier Martin, Lanfa Liu, Nathan Piasco, Emile Blettery, Raphael Sulzer, Stephane Guinard, Oussama Ennafii, Evelyn Paiz etc., who made my work environment full of friendliness and laughter. I am grateful to all the friends in my life who enriched me and I look forward to meeting them more often after the COVID subsides.\\



\dominitoc
\tableofcontents
\cleardoublepage
%!TEX root = Manuscript.tex

\chapter*{List of Acronyms}
\mtcaddchapter[List of Acronyms]

% Define here acronyms used in the manuscript. Copy paster the example for each new acromnym you would like to use

%\begin{acronym}
%\renewcommand{\\}{}
%\newacronym{DSM}{DSM}{Digital Surface Model}
%%\acro{DTI}{Diffusion Tensor Imaging}
%\end{acronym}
%\acro{DSM}{Digital Surface Model}

\begin{acronym}
	\acro{DSM}[DSM]{Digital Surface Model}
	\acro{DoD}[DoD]{Difference of DSMs}
	\acro{GCP}[GCP]{Ground Control Point}
	\acro{CNN}[CNN]{Convolutional Neural Network}
	\acro{SfM}[SfM]{Structure from Motion}
	\acro{IGN}[IGN]{Institut national de l'information géographique et forestière}
	\acro{GT}[GT]{Ground Truth}
	\acro{BBA}[BBA]{Block Bundle Adjustment}
	\acro{RPC}[RPC]{Rational Polynomial Coefficient}
\end{acronym}

%Abbreviations
%
%IGN
%IPGP
%Lastig
%ENSG
%
%intra-epoch: from the same time
%inter-epoch: from different times
%DSM: Digital Surface Model
%DoD: Difference of DSMs
%GCP: Ground Control Point
%BBA: Bundle block adjustment
%GSD: Ground sampling distance




\mainmatter
\cleardoublepage
%!TEX root = Manuscript.tex

%解释DSM,DoD,GCP,multi-epoch, intra and inter epoch, GT
\chapter{Introduction en français}
\label{chap:introFrench}
\minitoc

\section{Motivation et objectifs}
\subsection{Pourquoi les images historiques sont-elles intéressantes}
Les images aériennes historiques (c'est-à-dire analogiques ou d'archives) jouent un rôle important en fournissant des informations uniques sur l'évolution de la couverture terrestre. 
Ce sont des atouts précieux pour un grand nombre d'applications telles que (1) l'analyse des catastrophes naturelles (par exemple, tremblement de terre, glissement de terrain, volcan, inondation, avalanche, etc.), (2) la surveillance éco-environnementale (par exemple, forêt, atmosphère, glacier, eau, littoral, etc.), (3) l'expansion urbaine et (4) la pollution et la protection de l'environnement, etc.
\par
Les images aériennes historiques ont été régulièrement acquises depuis les années 1920 par des agences cartographiques, militaires ou cadastrales du monde entier. Une quantité massive d'entre elles ont été numérisées et rendues accessibles par des services web~\cite{sebastien2019archiving,earthexplorer,remonterletemps}. 
Par exemple, selon une enquête réalisée au début de 2017 en Europe~\cite{sebastien2019archiving}, il y a environ 50 millions d'images aériennes archivées en Europe, dont environ 37,8\% sont numérisées. 
Les images sont de haute résolution spatiale, et sont acquises en configuration stéréoscopique, permettant la restitution 3D des territoires. 
Elles sont souvent accompagnées de métadonnées, comprenant dans la plupart des cas la focale de la caméra, la hauteur de vol, l'échelle et la taille physique du capteur, qui sont généralement enregistrées ou mentionnées sur les films. D'autres métadonnées telles que les plans de vol, les certificats d'étalonnage de la caméra ou les orientations ne sont pas couramment disponibles. 
\par
Lorsque les paramètres d'étalonnage de la caméra sont inconnus, ils doivent être évalués au moyen d'une procédure appelée ajustement du faisceau d'auto-étalonnage. \ac{GCP}s sont nécessaires, sinon des paramètres de caméra estimés de manière inexacte entraîneront des surfaces d'erreur systématiques appelées effet de dôme (c'est-à-dire effet de bol).
En général, les \ac{GCP}s proviennent (1) de mesures sur le terrain \cite{micheletti2015application,walstra2004time,cardenal2006use}, (2) d'orthophotos et de \ac{DSM} récents \cite{nurminen2015automation,ellis2006measuring,fox2008unlocking} et (3) d'images satellites récentes \cite{ellis2006measuring,ford2013shoreline}. Le plus difficile est d'identifier les \ac{GCP}s sur les images historiques, ce qui n'est pas facile en raison des inévitables changements de scène. Les \ac{GCP} sont généralement mesurés manuellement à l'aide de photos récentes, mais cela reste monotone et laborieux. 
Il est urgent d'identifier automatiquement les points correspondants (c'est-à-dire les correspondances) sur des images historiques et récentes.\\
Lorsque les utilisateurs sont uniquement intéressés par la comparaison de différentes époques historiques, l'auto-calibrage peut être réalisé sans \ac{GCP}s. Les correspondances entre différentes époques serviraient d'observations dans l'ajustement du faisceau pour éliminer les erreurs systématiques des surfaces. En conclusion, le goulot d'étranglement de l'auto-calibration des images historiques est la récupération des correspondances sur des images prises à des époques différentes (c'est-à-dire multi-époques).



\subsection{Comment faire correspondre des images historiques multi-époques}
Cependant, la comparaison d'images historiques multi-époques reste difficile, malgré le fait qu'il existe un grand nombre d'algorithmes de comparaison d'images dont l'efficacité a été prouvée sur des images modernes. Les raisons en sont les suivantes:\\
\begin{enumerate}
	\item Les images multi-époques sont souvent acquises à différents moments de la journée et par différents temps et saisons, ce qui entraîne inévitablement des différences d'apparence.
	\item La scène change au fil du temps en raison de phénomènes anthropiques (par exemple, l'urbanisme) ou naturels (par exemple, un tremblement de terre), en particulier pour les grands écarts temporels.
	\item Les images multi-époques présentent souvent des résolutions spatiales hétérogènes, accompagnées de conditions d'acquisition différentes (capteurs, canaux spectraux, etc).
	\item Les images historiques sont souvent confrontées à une faible qualité radiométrique, notamment un faible contraste, du bruit d'image, une détérioration causée par le vieillissement des films, ou même des rayures sur les films.
\end{enumerate}
La simple application de méthodes d'appariement des caractéristiques (par exemple, SIFT~\cite{lowe2004distinctive} ou SuperGlue~\cite{sarlin2020superglue}) sur des paires d'images multi-époques donne souvent des résultats insatisfaisants. Un exemple est donné dans la Figure~\ref{MultiEpoqueImgPaire}. Une paire d'images multi-époques est représentée avec des rectangles rouges indiquant la zone de chevauchement sur la Figure~\ref{MultiEpoqueImgPaire}(a). Les images de gauche et de droite ont été prises au même endroit en 1954 et 1970 respectivement. La scène a changé de manière significative, beaucoup de nouveaux bâtiments sont apparus, les tons de couleur étaient très différents. Dans la Figure~\ref{MultiEpoqueImgPaire}(b-d), les résultats de correspondance de SIFT, SuperGlue et le nôtre sont affichés pour comparaison. Comme on peut le voir, SIFT n'a trouvé aucune correspondance. SuperGlue a trouvé 369 correspondances, dont la plupart semblent bonnes, mais en regardant plus attentivement, les détails révèlent une faible précision de localisation. Notre méthode a trouvé 1463 correspondances avec une grande précision, grâce à l'aide (1) de la géométrie 3D et (2) de la stratégie diviser et conquérir (c'est-à-dire grossier-à-précis), qui sont détaillées dans les textes suivants.
\begin{figure*}[htbp]
	\begin{center}
		\subfigure[Paire d'images multi-époques]{
			\begin{minipage}[t]{0.45\linewidth}
				\centering
				\includegraphics[width=6.2cm]{images/Chapitre1/OIS-Reech_IGNF_PVA_1-0__1954-03-06__C3544-0211_1954_CDP866_0630_OIS-Reech_IGNF_PVA_1-0__1970__C3544-0221_1970_CDP6452_1409.png}
			\end{minipage}%
		}
		\subfigure[Résultat de SIFT (0 correspondances)]{
			\begin{minipage}[t]{0.45\linewidth}
				\centering
				\includegraphics[width=6.2cm]{images/Chapitre1/Homol-SIFT_OIS-Reech_IGNF_PVA_1-0__1954-03-06__C3544-0211_1954_CDP866_0630_OIS-Reech_IGNF_PVA_1-0__1970__C3544-0221_1970_CDP6452_1409.png}
			\end{minipage}%
		}
		\subfigure[Résultat de SuperGlue]{
			\begin{minipage}[t]{0.45\linewidth}
				\centering
				\includegraphics[width=6.2cm]{images/Chapitre1/Homol-SuperGlue_OIS-Reech_IGNF_PVA_1-0__1954-03-06__C3544-0211_1954_CDP866_0630_OIS-Reech_IGNF_PVA_1-0__1970__C3544-0221_1970_CDP6452_1409.png}
			\end{minipage}%
		}
		\subfigure[Résultat du nôtre]{
			\begin{minipage}[t]{0.45\linewidth}
				\centering
				\includegraphics[width=6.2cm]{images/Chapitre1/Homol-Ours_OIS-Reech_IGNF_PVA_1-0__1954-03-06__C3544-0211_1954_CDP866_0630_OIS-Reech_IGNF_PVA_1-0__1970__C3544-0221_1970_CDP6452_1409.png}
			\end{minipage}%
		}
		\caption{(a) Une paire d'images multi-époques avec des rectangles rouges indiquant la zone de chevauchement. (b-d) Résultat de la correspondance de SIFT, SuperGlue et le nôtre.}
		\label{MultiEpoqueImgPaire}
	\end{center}
\end{figure*}

\paragraph{Avantages de la géométrie 3D}
Les images RGB sont largement utilisées pour pour l'appariement des images. Cependant, elles présentent les inconvénients suivants:\\
(1) Leur apparence change avec le temps (voir la Figure~\ref{Changementdapparence}), et avec des angles de vue variables sur des surfaces non-Lambertiennes (voir la Figure~\ref{Pauvrementtexture}).
(2) Les autosimilitudes (par exemple, les modèles répétitifs) favorisent les fausses correspondances (voir la Figure~\ref{Pauvrementtexture}).\\
Heureusement, la géométrie 3D telle que \ac{DSM} compense parfaitement ces défauts. Comme on peut le voir sur la Figure~\ref{Changementdapparence}, les images RGB sont très différentes car la scène a beaucoup changé. Cependant, les \ac{DSM} correspondants sont similaires, ce qui est raisonnable, car le paysage 3D est plus stable dans le temps. De plus, les \ac{DSM} sont plus distinctifs que les images RGB lorsqu'il s'agit de surfaces non-Lambertiennes et de motifs répétitifs, comme indiqué dans la Figure~\ref{Pauvrementtexture}. 
Même si la géométrie 3D manque de textures et de détails par rapport à l'image RGB, elle sert de complément idéal. En outre, elle joue un rôle important en fournissant des informations 3D pour établir un modèle de transformation de Helmert 3D entre les époques afin (1) de déplacer différentes époques dans le même cadre de coordonnées et (2) de supprimer les fausses correspondances dans une routine RANSAC qui est plus fiable que les modèles de transformation 2D.

\begin{figure*}[htbp]
	\begin{center}
		\subfigure[Image RGB 1971]{
			\begin{minipage}[t]{0.45\linewidth}
				\centering
				\includegraphics[width=6.2cm]{images/Chapitre1/AppearanceChangeRGBL.png}
			\end{minipage}%
		}
		\subfigure[Image RGB 2015]{
			\begin{minipage}[t]{0.45\linewidth}
				\centering
				\includegraphics[width=6.2cm]{images/Chapitre1/AppearanceChangeRGBR.png}
			\end{minipage}%
		}
		\subfigure[\ac{DSM} 1971]{
			\begin{minipage}[t]{0.45\linewidth}
				\centering
				\includegraphics[width=6.2cm]{images/Chapitre1/AppearanceChangeDSML.png}
			\end{minipage}%
		}
		\subfigure[\ac{DSM} 2015]{
			\begin{minipage}[t]{0.45\linewidth}
				\centering
				\includegraphics[width=6.2cm]{images/Chapitre1/AppearanceChangeDSMR.png}
			\end{minipage}%
		}
		\caption{La même zone observée à différents moments. Les images RGB ont beaucoup changé alors que les \ac{DSM}s sont restés stables au fil du temps.}
		\label{Changementdapparence}
	\end{center}
\end{figure*} 


\begin{figure*}[htbp]
	\begin{center}
		\subfigure[Image RGB 1971]{
			\begin{minipage}[t]{0.45\linewidth}
				\centering
				\includegraphics[width=6.2cm]{images/Chapitre1/PoorlyTexturedRGBL.png}
			\end{minipage}%
		}
		\subfigure[Image RGB 2015]{
			\begin{minipage}[t]{0.45\linewidth}
				\centering
				\includegraphics[width=6.2cm]{images/Chapitre1/PoorlyTexturedRGBR.png}
			\end{minipage}%
		}
		\subfigure[\ac{DSM} 1971]{
			\begin{minipage}[t]{0.45\linewidth}
				\centering
				\includegraphics[width=6.2cm]{images/Chapitre1/PoorlyTexturedDSML.png}
			\end{minipage}%
		}
		\subfigure[\ac{DSM} 2015]{
			\begin{minipage}[t]{0.45\linewidth}
				\centering
				\includegraphics[width=6.2cm]{images/Chapitre1/PoorlyTexturedDSMR.png}
			\end{minipage}%
		}
		\caption{La même végétation observée à des moments différents. Réflexion non-lambertienne et autosimilitude présentes dans les images RGB, tandis que les \ac{DSM}s restent distinctifs.}
		\label{Pauvrementtexture}
	\end{center}
\end{figure*} 

\paragraph{Diviser et conquérir}
Puisque la récupération de correspondances robustes et précises sur des paires d'images multi-époques est une tâche difficile, nous divisons la tâche en deux sous-tâches et les conquérons individuellement avec la stratégie grossier-à-précis. Cette stratégie est illustrée dans la Figure~\ref{grossier-à-précis}. Les deux sous-tâches sont les suivantes:\\
\begin{enumerate}
	\item Co-enregistrement grossier, comme illustré sur la Figure~\ref{grossier-à-précis}(b). Son objectif est d'aligner grossièrement les paires d'images multi-époques en se concentrant sur la robustesse et en relâchant l'exigence de précision.
	\item Appariement précis, comme illustré sur la Figure~\ref{grossier-à-précis}(c). Elle améliore les correspondances prédites par le résultat grossier du co-enregistrement en recherchant uniquement le voisinage local pour réduire l'ambiguïté.
\end{enumerate}
\begin{figure*}[htbp]
	\begin{center}
		\subfigure[Exemple d'une paire d'images inter-époques]{
			\begin{minipage}[t]{1\linewidth}
				\centering
				\includegraphics[width=1\columnwidth]{images/Chapitre1/imagepair.png}
			\end{minipage}%
		}
		\subfigure[Co-enregistrement grossier]{
			\begin{minipage}[t]{1\linewidth}
				\centering
				\includegraphics[width=0.68\columnwidth]{images/Chapitre1/CoReg.jpg}
			\end{minipage}%
		}
		\subfigure[Appariement précis]{
			\begin{minipage}[t]{1\linewidth}
				\centering
				\includegraphics[width=0.55\columnwidth]{images/Chapitre1/Precise.png}
			\end{minipage}%
		}
		\caption{Stratégie grossier-à-précis. (a) Un exemple de paire d'images inter-époques à apparier. $I^{e_1}$ et $I^{e_2}$ représentent les images prises à $epoch_1$ et $epoch_2$ individuellement. (b) Illustration du co-enregistrement grossier entre $I^{e_1}$ et $I^{e_2}$. En conséquence, $I^{e_1}$ est grossièrement aligné avec $I^{e_2}$. (c) Illustration de l'appariement précis. Pour les points clés de $I^{e_1}$ (croix verte), un emplacement est prédit dans $I^{e_2}$ (croix violette) sur la base d'un co-enregistrement grossier, dont le voisinage local sera recherché pour trouver l'appariement précis (croix jaune).}
		\label{grossier-à-précis}
	\end{center}
\end{figure*}

\section{Contributions}
Dans cette thèse, nous présentons des pipelines grossiers-à-précis pour l'appariement d'images multi-époques. Ils sont adaptés aux images aériennes, satellitaires et mixtes, ce qui ouvre la possibilité de géoréférencer des millions d'images historiques sans nécessiter de \ac{GCP}s. 
Six variantes sont proposées pour l'étape de co-enregistrement grossier et deux variantes pour l'étape d'appariement précis. Chaque variante a sa propre caractéristique:\\
\begin{enumerate}
	\item Pour les variantes de co-enregistrement grossier: (1) celles basées sur l'idée d'appariement des \ac{DSM}s conduisent généralement aux résultats les plus robustes ; (2) celles qui apparient les orthophotos pourraient servir d'alternatives dans les rares scénarios de terrain parfaitement plat où les \ac{DSM}s ne fournissent pas d'informations utiles ; (3) les autres qui apparient les paires d'images originales conduisent souvent à des résultats moins satisfaisants, mais ce sont les seules options adaptées aux images terrestres.
	\item Pour les variantes d'appariement précis: (1) $Patch$ est basé sur des méthodes d'appariement par apprentissage, il donne généralement plus de correspondances car il est plus invariant dans le temps. (2) $Guided$ est basé sur des méthodes artisanales, il est plus efficace en termes d'utilisation de la mémoire et des ressources CPU car il n'implique pas de rééchantillonnage des patchs, ce qui est nécessaire pour $Patch$. 
\end{enumerate}
\par
Nos pipelines visent à libérer le potentiel des images historiques pour le suivi des conditions environnementales. 
Nous collaborons actuellement avec plusieurs instituts pour appliquer nos pipelines dans différentes applications, notamment:\\
\begin{enumerate}
	\item Institut de Physique du Globe de Paris (IPGP) et Korea Institute of Geoscience and Mineral Resources (KIGAM) pour analyser les déformations de la croûte terrestre afin de comprendre les événements sismiques.
	\item Conseil national de la recherche, Institut de recherche pour la protection hydrogéologique (CNR-IRPI) pour l'analyse de l'évolution des glissements de terrain en Italie.
	\item Département des sciences de la terre et de l'environnement de l'université de Pavie pour l'analyse de l'évolution des badlands en Europe.
\end{enumerate}
\par
Nous avons également développé deux tutoriels complets accompagnés d'ensembles de données de test pour familiariser les utilisateurs avec nos pipelines implémentés dans MicMac\cite{HistoPcode} (plus de détails sont présentés dans l'annexe \ref{chap:appendixC}):\\
\begin{enumerate}
	\item Tutoriel d'appariement des images aériennes \cite{tuto-aerial} 
	\item Tutoriel d'appariement d'images mixtes (c'est-à-dire d'images aériennes et satellitaires) \cite{tuto-mixed} 
\end{enumerate}
\par
Publications de l'auteur:\\
\begin{enumerate}
	\item Lulin Zhang, Ewelina Rupnik et Marc Pierrot-Deseilligny. Appariement des caractéristiques pour des images aériennes historiques multi-époques. ISPRS Journal of Photogrammetry and Remote Sensing, 182, 176-189, 2021.
	\item Lulin Zhang, Ewelina Rupnik et Marc Pierrot-Deseilligny.	Appariement des caractéristiques guidé pour l'estimation de la pose de blocs d'images historiques multi-époques. Dans ISPRS Ann. Photogramm. Remote Sens. Spatial Inf. Sci., 2020.
\end{enumerate}
Nous fournissons également une vidéo \cite{HistoPVideo}, des diapositives \cite{HistoPSlides} et le site web du projet \cite{HistoPProj} pour améliorer la visibilité de notre travail.\\

\section{Organisation de la thèse}
Cette thèse présente des pipelines entièrement automatiques pour l'appariement d'images multi-époques. 
Une brève présentation de \textit{l'état de l'art} est donnée dans le \textbf{Chapitre}~\ref{chap:review}.\\

Dans le \textbf{Chapitre}~\ref{chap:ApplicationsAndDatasets}, les applications ainsi que 5 données représentatifs sont présentés, qui sont ensuite utilisés pour tester nos pipelines.\\

Dans le \textbf{Chapitre}~\ref{chap:RoughCoReg}, six variantes de co-enregistrement grossier sont élaborées pour aligner grossièrement le bloc entier en construisant un modèle de transformation globalement cohérent entre les époques différentes.\\

Dans le \textbf{Chapitre}~\ref{chap:Precisematching}, deux variantes d'appariement précis sont introduites pour obtenir des appariements exacts sous la direction des orientations et des \ac{DSM} qui sont grossièrement co-registrés.\\

Enfin, le \textbf{Chapitre} ~\ref{chap:conclusion} présente les conclusions et les perspectives.\\

\cleardoublepage
%!TEX root = Manuscript.tex

\chapter{Introduction}
\label{chap:intro}
\minitoc

\section{Motivation and objectives}
\subsection{Why are historical images interesting}
Historical (i.e., analogue or archival) aerial images play an important role in providing unique information about evolution of land-covers. 
They are valuable assets for a wide range of applications such as (1) analyzing of natural disasters (e.g., earthquake, landslide, volcano, flood, avalanche, etc), (2) eco-environmental monitoring (e.g., forest, atmosphere, glacier, water, coastline, etc), (3) urban expansion and (4) environmental pollution and protection and so on.
\par
Historical aerial images have been regularly acquired since the 1920’s by mapping, military or cadastral agencies all over the world. A mass amount of them have been digitized and made accessible through web services~\cite{sebastien2019archiving,earthexplorer,remonterletemps}. 
For example, according to a survey taken place at the beginning of 2017 in Europe~\cite{sebastien2019archiving}, there are approximately 50 million of aerial images archived in Europe, with around 37.8\% of them digitized. 
The images are of high spatial resolution, and are acquired in stereoscopic configuration, allowing for 3D restitution of territories. 
They are often accompanied by metadata, in most cases including the camera focal length, flight height, scale and the physical sensor size, which are usually saved or mentioned on the films. Other metadata such as flight plans, camera calibration certificates or orientations are not commonly available. 
\par
When the camera calibration parameters are unknown, they should be evaluated by a procedure called self-calibrating bundle adjustment. \ac{GCP}s are required, otherwise inaccurately estimated camera parameters will lead to systematic error surfaces called dome effect (i.e., bowl effect).
Generally, \ac{GCP}s originate from (1) field surveys \cite{micheletti2015application,walstra2004time,cardenal2006use}, (2) recent orthophotos and \ac{DSM} \cite{nurminen2015automation,ellis2006measuring,fox2008unlocking} and (3) recent satellite images \cite{ellis2006measuring,ford2013shoreline}. The most challenging part is to identify the \ac{GCP}s on the historical images, which is not easy due to inevitable scene changes. \ac{GCP}s are usually manually measured with the help of recent photos, however, it is still monotonous and time-consuming. 
There is an urgent need to automatically identify corresponding points (i.e., matches) on historical and recent images.\\
When users are only interested in comparing different historical epochs, the self-calibration can be accomplished without \ac{GCP}s. Matches between different epochs would serve as observations in bundle adjustment to eliminate the systematic errors in surfaces. In conclusion, the bottleneck of historical image self-calibration is recovering matches on images taken at different times (i.e., multi-epoch).

\subsection{How to match multi-epoch historical images}
However, matching multi-epoch historical images remains challenging, despite the fact that there exists a large number of image matching algorithms with their effectiveness proven on modern images. The reasons include:
\begin{enumerate}
	\item Multi-epoch images are often acquired at different times of day and in various weathers and seasons, which unavoidably leading to appearance differences.
	\item The scene changes over time due to anthropogenic phenomena (e.g., urban planning) or natural ones (e.g., earthquake), especially for large time gaps.
	\item Multi-epoch images often exhibit heterogeneous spatial resolutions, accompanied with different acquisition conditions (sensors, spectral channels, etc).
	\item Historical images are often facing low radiometric quality, including low contrast, image noise, deterioration due to the aging of films, or even scratches on the films.
\end{enumerate}
Simply applying \textit{state-of-the-art} feature matching methods (e.g., SIFT~\cite{lowe2004distinctive} or SuperGlue~\cite{sarlin2020superglue}) on multi-epoch image pair often leads to unsatisfactory results. An example is given in Figure~\ref{MultiEpochImgPair}. A pair of multi-epoch images are demonstrated with red rectangles indicating the overlapping area in Figure~\ref{MultiEpochImgPair}(a). The left and right images are taken at the same place in 1954 and 1970 individually. The scene changed significantly, a lot of new buildings arose, the color tones were very different. In Figure~\ref{MultiEpochImgPair}(b-d), the matching result of SIFT, SuperGlue and Ours are displayed for comparison. As can be seen, SIFT failed to find any matches. SuperGlue recovered 369 matches, most of which seem good, but at a closer look the details reveal poor localization precision. Our method found 1463 matches with high accuracy, thanks to the help of (1) 3D geometry and (2) the divide and conquer (i.e., rough-to-precise) strategy, which are elaborated in the following texts.
\begin{figure*}[htbp]
	\begin{center}
		\subfigure[Multi-epoch image pair]{
			\begin{minipage}[t]{0.45\linewidth}
				\centering
				\includegraphics[width=6.2cm]{images/Chapitre1/OIS-Reech_IGNF_PVA_1-0__1954-03-06__C3544-0211_1954_CDP866_0630_OIS-Reech_IGNF_PVA_1-0__1970__C3544-0221_1970_CDP6452_1409.png}
			\end{minipage}%
		}
		\subfigure[Result of SIFT (0 matches)]{
	\begin{minipage}[t]{0.45\linewidth}
		\centering
		\includegraphics[width=6.2cm]{images/Chapitre1/Homol-SIFT_OIS-Reech_IGNF_PVA_1-0__1954-03-06__C3544-0211_1954_CDP866_0630_OIS-Reech_IGNF_PVA_1-0__1970__C3544-0221_1970_CDP6452_1409.png}
	\end{minipage}%
}
		\subfigure[Result of SuperGlue]{
	\begin{minipage}[t]{0.45\linewidth}
		\centering
		\includegraphics[width=6.2cm]{images/Chapitre1/Homol-SuperGlue_OIS-Reech_IGNF_PVA_1-0__1954-03-06__C3544-0211_1954_CDP866_0630_OIS-Reech_IGNF_PVA_1-0__1970__C3544-0221_1970_CDP6452_1409.png}
	\end{minipage}%
}
		\subfigure[Result of Ours]{
	\begin{minipage}[t]{0.45\linewidth}
		\centering
		\includegraphics[width=6.2cm]{images/Chapitre1/Homol-Ours_OIS-Reech_IGNF_PVA_1-0__1954-03-06__C3544-0211_1954_CDP866_0630_OIS-Reech_IGNF_PVA_1-0__1970__C3544-0221_1970_CDP6452_1409.png}
	\end{minipage}%
}
		\caption{(a) A pair of multi-epoch images with red rectangles indicating the overlapping area. (b-d) Matching result of SIFT, SuperGlue and Ours.}
		\label{MultiEpochImgPair}
	\end{center}
\end{figure*}

%\subsubsection{Take advantage of 3D geometry}
\paragraph{Advantages of 3D geometry}
RGB images are widely used for image matching. However, they have the following shortcomings:\\
(1) Their appearances change over time (see Figure~\ref{AppearanceChange}), and over varying view angles on non-Lambertian surfaces (see Figure~\ref{PoorlyTextured}).\\
(2) Self similarities (e.g., repetitive patterns) favor false matches (see Figure~\ref{PoorlyTextured}).\\
Fortunately, 3D geometry such as \ac{DSM} makes up for these shortcomings perfectly. As can be seen in Figure~\ref{AppearanceChange}, the RGB images look very different because the scene changed a lot. However, the corresponding \ac{DSM}s look similar, which is reasonable, as the 3D landscape is more stable over time. Besides, \ac{DSM} is more distinctive than RGB image when it comes to non-Lambertian surfaces and repetitive patterns, as shown in Figure~\ref{PoorlyTextured}. 
Even though 3D geometry lacks textures and details compared to RGB image, it serves as an ideal supplement. Besides, it plays an important role in providing the 3D information for establishing 3D Helmert transformation model between epochs to (1) move different epochs into the same coordinate frame and (2) remove false matches in a RANSAC routine which is more reliable than 2D transformation models.

\begin{figure*}[htbp]
	\begin{center}
		\subfigure[RGB image 1971]{
			\begin{minipage}[t]{0.45\linewidth}
				\centering
				\includegraphics[width=6.2cm]{images/Chapitre1/AppearanceChangeRGBL.png}
			\end{minipage}%
		}
		\subfigure[RGB image 2015]{
			\begin{minipage}[t]{0.45\linewidth}
				\centering
				\includegraphics[width=6.2cm]{images/Chapitre1/AppearanceChangeRGBR.png}
			\end{minipage}%
		}
		\subfigure[\ac{DSM} 1971]{
			\begin{minipage}[t]{0.45\linewidth}
				\centering
				\includegraphics[width=6.2cm]{images/Chapitre1/AppearanceChangeDSML.png}
			\end{minipage}%
		}
		\subfigure[\ac{DSM} 2015]{
			\begin{minipage}[t]{0.45\linewidth}
				\centering
				\includegraphics[width=6.2cm]{images/Chapitre1/AppearanceChangeDSMR.png}
			\end{minipage}%
		}
		\caption{The same zone observed in different times. The RGB images changed a lot while the \ac{DSM}s stayed stable over time.}
		\label{AppearanceChange}
	\end{center}
\end{figure*} 


\begin{figure*}[htbp]
	\begin{center}
		\subfigure[RGB image 1971]{
			\begin{minipage}[t]{0.45\linewidth}
				\centering
				\includegraphics[width=6.2cm]{images/Chapitre1/PoorlyTexturedRGBL.png}
			\end{minipage}%
		}
		\subfigure[RGB image 2015]{
			\begin{minipage}[t]{0.45\linewidth}
				\centering
				\includegraphics[width=6.2cm]{images/Chapitre1/PoorlyTexturedRGBR.png}
			\end{minipage}%
		}
		\subfigure[\ac{DSM} 1971]{
			\begin{minipage}[t]{0.45\linewidth}
				\centering
				\includegraphics[width=6.2cm]{images/Chapitre1/PoorlyTexturedDSML.png}
			\end{minipage}%
		}
		\subfigure[\ac{DSM} 2015]{
			\begin{minipage}[t]{0.45\linewidth}
				\centering
				\includegraphics[width=6.2cm]{images/Chapitre1/PoorlyTexturedDSMR.png}
			\end{minipage}%
		}
		\caption{The same vegetation observed in different times. Non-Lambertian reflection and self similarities present in RGB images, while the \ac{DSM}s stay distinctive.}
		\label{PoorlyTextured}
	\end{center}
\end{figure*} 

%\subsubsection{Rough-to-precise strategy}
\paragraph{Divide and Conquer}
Since the task of recovering robust and precise matches on multi-epoch image pairs is difficult, we divide the task into two sub-tasks and conquer them individually with the rough-to-precise strategy. It is illustrated in Figure~\ref{rough-to-precise}. The two sub-tasks includes:\\
\begin{enumerate}
	\item Rough co-registration, as illustrated in Figure~\ref{rough-to-precise}(b). Its goal is to roughly align the multi-epoch image pairs by focusing on robustness and relaxing the requirement for accuracy.
	\item Precise matching, as illustrated in Figure~\ref{rough-to-precise}(c). It refines the matches predicted by the rough co-registration result by searching only the local neighborhood to reduce ambiguity.
\end{enumerate}

\begin{figure*}[htbp]
	\begin{center}
		\subfigure[Example of an inter-epoch image pair]{
			\begin{minipage}[t]{1\linewidth}
				\centering
				\includegraphics[width=1\columnwidth]{images/Chapitre1/imagepair.png}
			\end{minipage}%
		}
		\subfigure[Rough co-registration]{
			\begin{minipage}[t]{1\linewidth}
				\centering
				\includegraphics[width=0.68\columnwidth]{images/Chapitre1/CoReg.jpg}
			\end{minipage}%
		}
		\subfigure[Precise matching]{
			\begin{minipage}[t]{1\linewidth}
				\centering
				\includegraphics[width=0.55\columnwidth]{images/Chapitre1/Precise.png}
			\end{minipage}%
		}
		\caption{ Rough-to-precise strategy. (a) An example of an inter-epoch image pair to be matched. $I^{e_1}$ and $I^{e_2}$ represents images take at $epoch_1$ and $epoch_2$ individually. (b) Illustration of rough co-registration between $I^{e_1}$ and $I^{e_2}$. As a result, $I^{e_1}$ is roughly aligned with $I^{e_2}$. (c) Illustration of precise matching. For keypoints in $I^{e_1}$ (green cross), a location is predicted in $I^{e_2}$ (purple cross) based on rough co-registration, whose local neighborhood will be searched to find the precise match (yellow cross).}
		\label{rough-to-precise}
	\end{center}
\end{figure*}

\section{Contributions}
\label{sec:contributions}
In this thesis we present rough-to-precise pipelines for matching multi-epoch images. They are suitable for aerial, satellite and mixed images, which open the possibility of geo-referencing millions of historical images without requiring any \ac{GCP}s. 
Six variants are provided for the rough co-registration stage and two variants for the precise matching stage. Each variant has its own characteristic:\\
\begin{enumerate}
	\item For rough co-registration variants: (1) the ones based on the idea of matching \ac{DSM}s generally lead to the most robust results; (2) the ones that match orthophotos could serve as an alternates in rare scenarios of perfectly flat terrain where \ac{DSM}s fail to provide useful information; (3) the others that match original image pairs often lead to less satisfactory results, but they are the only options suitable for terrestrial images.
	\item For precise matching variants: (1) $Patch$ is based on learned matching methods, it generally results in more matches as it is more invariant over time. (2) $Guided$ is based on hand-crafted methods, it is more efficient in terms of the use of memory and CPU resources as it doesn't involve resampling patches, which is necessary for $Patch$. 
\end{enumerate}
\par
Our pipelines aim to unlock the potential of historical images for tracking environmental conditions. 
We are currently collaborating with several institutes to apply our pipelines in different applications, including:
\begin{enumerate}
	\item Institut de Physique du Globe de Paris (IPGP) and Korea Institute of Geoscience and Mineral Resources (KIGAM) for analyzing deformations of the earth crust to understand the seismic events.
	\item National Research Council, Research Institute for Hydrogeological Protection (CNR-IRPI) for analyzing landslide evolution in Italy.
	\item Department of Earth and Environmental Sciences in University of Pavia for analyzing badland evolution in Europe.
\end{enumerate}
\par
We also developed two thorough tutorials accompanied with test datasets to familiarize users with our pipelines implemented in MicMac\cite{HistoPcode} (more details are introduced in Appendix \ref{chap:appendixC}):
\begin{enumerate}
	\item Tutorial of matching aerial images \cite{tuto-aerial} 
	\item Tutorial of matching mixed images (i.e., aerial and satellite images) \cite{tuto-mixed} 
\end{enumerate}

\par
Publications of the author:
\begin{enumerate}
	\item Lulin Zhang, Ewelina Rupnik and Marc Pierrot-Deseilligny. Feature matching for multi-epoch historical aerial images. ISPRS Journal of Photogrammetry and Remote Sensing, 182, 176-189, 2021.
	\item Lulin Zhang, Ewelina Rupnik and Marc Pierrot-Deseilligny.	Guided feature matching for multi-epoch historical image blocks pose estimation. In ISPRS Ann. Photogramm. Remote Sens. Spatial Inf. Sci., 2020.
\end{enumerate}
We also provide video \cite{HistoPVideo}, slides \cite{HistoPSlides} and project website \cite{HistoPProj} to improve the visibility of our work.

\section{Organization of the thesis}
This thesis presents fully automatic pipelines to match multi-epoch images.
A brief presentation of the \textit{state-of-the-art} is given in \textbf{Chapter}~\ref{chap:review}. \\

In \textbf{Chapter}~\ref{chap:ApplicationsAndDatasets}, applications as well as 5 sets of representative datasets are introduced, which are latter used to test our pipelines.\\

In \textbf{Chapter}~\ref{chap:RoughCoReg}, six rough co-registration variants are elaborated to roughly align the whole block by building a globally consistent transformation model between different epochs.\\

In \textbf{Chapter}~\ref{chap:Precisematching}, two precise matching variants are introduced to get accurate matches under the guidance of roughly co-registered orientations and \ac{DSM}s.\\

Finally, in \textbf{Chapter} ~\ref{chap:conclusion} conclusion and perspective are given.\\


\cleardoublepage
%!TEX root = Manuscript.tex

\chapter{Literature review}
\label{chap:review}
\minitoc

\section{Local feature matching}
Local feature refers to a discriminative structure found in an image, such as a point, corner, blob, edge or image patch. It is often accompanied with a descriptor, which is a compact vector representing the local neighborhood.
\par
According to different data storage types, descriptors can be divided into two categories: floating-point and binary descriptors. The former is recorded in floating-point format, which has the advantage of being informative. It is widely used in various matching scenarios.
The latter is stored in binary type, which guarantees faster processing while demanding less memory. It is particularly suitable for real-time and/or smartphone applications.
Since our goal is to match multi-epoch images for high accuracy cartography, we are interested in floating-point descriptors rather than binary ones.
\par
According to whether machine learning techniques are applied, local features can be categorized as hand-crafted or learned. We subsequently elaborate on the two categories of approaches.
\subsection{Hand-crafted methods}
In the early stage, Moravec detects corner feature by measuring the sum-of-squared-differences (SSD) by applying a small shift in a number of directions to the patch around a candidate feature \cite{moravec1980obstacle}. Based on this, Harris computes an approximation to the second derivative of the SSD with respect to the shift \cite{harris1988combined}. Since both Moravec and Harris are sensitive to changes in image scale, algorithms invariant to scale and affine transformations based on Harris are presented \cite{mikolajczyk2004scale}. Other than corner feature, SIFT (Scale-invariant feature transform) \cite{lowe2004distinctive} detects blob feature in scale-space, which is an entire pipeline including detection and description. It uses a difference-of-Gaussian function to identify potential feature points that are invariant to scale and orientation. SIFT is a milestone among hand-crafted features, and is able to outperform machine learning alternatives when matching conditions are favorable. RootSIFT \cite{arandjelovic2012three} uses a square root (Hellinger) kernel instead of the standard Euclidean distance to measure the similarity between SIFT descriptors, which leads to a dramatic performance boost. Similar to SIFT, SURF \cite{bay2006surf} resorts to integral images and Haar filters to extract blob feature in a computationally efficient way. DAISY \cite{tola2009daisy} is a local image descriptor, which uses convolutions of gradients in specific directions with several Gaussian filters to make it very efficient to extract dense descriptors. KAZE \cite{alcantarilla2012kaze} is an algorithm that detects and describes multi-scale 2D feature in nonlinear scale spaces. AKAZE \cite{Alcantarilla13bmvc} is an accelerated version based on KAZE.
\subsection{Learned methods}
With the rise of machine learning, learned features have shown their feasibility in the image matching problem when enough ground truth data is available. 
FAST \cite{rosten2006machine} uses decision tree to speed up the process of finding corner feature. 
LIFT (Learned Invariant Feature Transform) \cite{yi2016lift} is a deep network architecture that implements a full pipeline including detection, orientation estimation and feature description. It is based on the previous work TILDE \cite{verdie2015tilde}, the method of \cite{moo2016learning} and DeepDesc \cite{simo2015discriminative}. 
Tian et al. introduce L2-Net~\cite{tian2017l2} to learn high performance descriptor in Euclidean space via the \ac{CNN}. 
Afterwards Mishchuk et al.~\cite{mishchuk2017working} introduce a compact descriptor named HardNet, by applying a novel loss to L2Net \cite{tian2017l2}. 
DELF \cite{noh2017DELF} is an attentive local feature descriptor based on \ac{CNN}, which works particularly well for illumination changes.
SuperPoint \cite{detone2018superpoint} is a self-supervised, fully-convolutional model that operates on full-sized images and jointly computes pixel-level feature point locations and associated descriptors in one forward pass. 
LF-Net \cite{ono2018lf} is a deep architecture that embeds the entire feature extraction pipeline, and can be trained end-to-end with just a collection of images. 
D2-Net \cite{dusmanu2019d2} is a single neural network that works as a \textit{dense} feature descriptor and a feature detector simultaneously, but their keypoints are less accurate compared to classical features since they are extracted on feature maps which have a resolution of 1/4 of the input image.
ASLFeat~\cite{luo2020aslfeat} improves shape-awareness and localization accuracy by applying light-weight yet effective modifications on improved D2-Net.
R2D2 \cite{revaud2019r2d2} is a \ac{CNN} architecture that learns \textit{dense} local descriptors (one for each pixel) as well as two associated repeatability and reliability confidence maps.
Contextdesc ~\cite{luo2019contextdesc} is a unified learning framework that leverages and aggregates the cross-modality contextual information.
D2D \cite{wiles2020d2d} allows \textit{dense} features to be modified based on the differences between the images by conditioning the feature maps on both images. 
Different than the aforementioned feature extraction methods, SuperGlue \cite{sarlin2020superglue} presents a new way of thinking about the feature matching problem. 
%Instead of learning better task-agnostic local features followed by simple matching heuristics and tricks, SuperGlue
%It learns the matching process from pre-existing local features 
It matches two sets of pre-existing local features by adopting a flexible context aggregation mechanism based on attention to jointly find matches and reject non-matchable points, leading to robust matching results even in challenging situations.
\par
Early learned methods (LIFT ~\cite{yi2016lift}, L2-Net~\cite{tian2017l2}, HardNet~\cite{mishchuk2017working}, DELF ~\cite{noh2017DELF}, SuperPoint ~\cite{detone2018superpoint}, LF-Net ~\cite{ono2018lf}) use only intermediate metrics (e.g., repeatability, matching score, mean matching accuracy, etc) to evaluate the matching performance. 
Even though they demonstrate better performance when compared to hand-crafted features on certain benchmarks, it does not necessarily imply a better performance in terms of subsequent processing steps. For example, in the context of \ac{SfM}, finding additional matches for image pairs where SIFT already provides enough matches does not necessarily result in more accurate or complete reconstructions \cite{schonberger2017comparative}.
Jin et al.~\cite{jin2020image} introduce a comprehensive benchmark for local features and robust estimation algorithms, focusing on the accuracy of the reconstructed camera pose as the primary metric. Using the new metric, SIFT ~\cite{lowe2004distinctive} and SuperGlue ~\cite{sarlin2020superglue} take the lead~\cite{imagematchingchallenge2020}.

\section{Robust matching}
The goal of robust matching is to tell apart true matches (i.e., inliers) from false matches (i.e., outliers), and eliminate the latter from further processing.
\par
Typically, an iterative sampling strategy based on RANSAC (Random Sample Consensus) \cite{fischler1981random} relying on some mathematical model, such as homography \cite{sonka2014image} or essential matrix \cite{sonka2014image} is carried out to remove outliers. 
This is an important issue which was often not given sufficient attention.
LMedS (Least Median of Squares) \cite{leroy1987robust} is a meaningful groundwork before RANSAC, which is also commonly used to replace RANSAC.
MLESAC (Maximum Likelihood SAC)  \cite{torr2000mlesac} adopts the same sampling strategy as RANSAC but chooses the solution that maximizes the likelihood instead of the number of inliers. PROSAC (Progressive Sample Consensus ) \cite{chum2005matching} chooses samples from progressively larger sets of top-ranked matches, which makes it significantly faster than RANSAC. DEGENSAC \cite{chum2005two} is an algorithm for epipolar geometry estimation unaffected by planar degeneracy. It is widely used in the 2020 image matching challenge \cite{imagematchingchallenge2020}.
USAC (Universal RANSAC) \cite{raguram2012usac} framework is a synthesis of the various optimizations and improvements that have been proposed to RANSAC.
GC-RANSAC (Graph-Cut RANSAC) \cite{barath2018graph} runs graph-cut algorithm in the local optimization step.
MAGSAC \cite{barath2019magsac} eliminates the need for a user-defined inlier-outlier threshold with marginalization.
\par
Various deep learning methods have also been developed to handle the erroneous matches.
DSAC (the differentiable counterpart of RANSAC) \cite{brachmann2017dsac} replaces the deterministic hypothesis selection by a probabilistic selection.
CNe (Context Networks) \cite{moo2018learning} trains deep networks in an end-to-end fashion to label the matches as inliers or outliers, known intrinsics are required as input, and a post-processing with RANSAC is often tasked. CNe was embedded into the framework of \cite{jin2020image} to remove outliers, paired with DEGENSAC, PyRANSAC (a variant of DEGENSAC by disabling the degeneracy check, introduced in \cite{jin2020image}) and MAGSAC. The results showed that with SIFT used to train CNe, about 80\% of the outliers were filtered out. Nearly all classical methods benefited from CNe, but not the learned ones. Jin et al. \cite{jin2020image} also state that RANSAC should be tuned to particular feature detector and descriptor, and specific settings should be selected for a particular RANSAC variant.
\par
In this research, we use RANSAC to estimate the 3D Helmert transformation between surfaces (i.e., \ac{DSM}s) calculated in different epochs. Compared to the classical essential/fundamental matrix filtering, with less data we impose stricter rules on the sets of points. Lastly, we eliminate the remaining false matches by looking at their cross-correlation.

\section{Pose estimation}
Pose estimation describes the intrinsic and extrinsic parameters of an image and is classically solved with the \ac{SfM} algorithms \cite{snavely2006photo,deseilligny2011apero,schonberger2016structure} based on local feature matches. 
%If the image pose should be known in a reference coordinate system, i.e., be georeferenced, a 7-parameter transformation followed by \textit{a posteriori} bundle block adjustment must be carried out.
The accuracy of matches plays an important role throughout the \ac{SfM} process, since small inaccuracies in their locations can result in large errors in the estimated poses. 
Good matches will lead to better results on image orientation, camera calibration and 3D reconstruction \cite{lindenberger2021pixel}, \cite{sarlin2021back}, \cite{truong2018second}.
%Sarlin et al. \cite{lindenberger2021pixel}, Lindenberger et al. \cite{lindenberger2021pixel} and Truong Giang et al. \cite{truong2018second} show.

\par
Unlike in modern images where the image coordinate system overlaps with the camera coordinate system, in historical images the overlap is not maintained due to the scanning procedure. To account for this, an additional 2D transformation is estimated in the \ac{SfM} procedure \cite{article}, which puts higher demands on the matches. %Inaccurate features 
\cite{giordano2018toward} demonstrates the importance of good matches in estimating the camera calibration and its great impact on the planimetric and altimetric accuracies of the resulted \ac{DSM}. 
Systematic errors expressed as \textit{dome} effect (i.e., a vertical doming of the surface) could appear in the \ac{DSM}s when camera models are poorly estimated {(i.e., inaccurately estimated lens distortion parameters)} \cite{wackrow2008minimising}, \cite{james2014mitigating}, which restricts the wider use of \ac{DSM}s. 
%It results from inaccurately estimated lens distortion parameters \cite{wackrow2008minimising}, \cite{james2014mitigating}.

%\cite{giordano2018toward} demonstrates the importance of the self-calibration of historical images and its impact on 3D accuracy. Poorly modelled intrinsic parameters result in the known \textit{dome}-like deformations that the authors eliminate with the help of automatic GCPs from exiting orthophotos. 


%The tie-point extraction step is the opening in the photogrammetric
%processing chain and therefore plays a key role in the quality of the subsequent image orientation,
%camera calibration and 3D reconstruction. Improving its precision will have an impact on the
%obtained 3D. In this research work we describe a method which aims at enhancing the accuracy of
%image orientation by adding a second iteration photogrammetric processing.

%\par 

%The uncertain internal geometry can be partially resolved by calibration. Research recent conducted at Loughborough University indicated the potential of consumer grade digital sensors to maintain their internal geometry but also identified residual systematic error surfaces, discernible in digital elevation models (DEM), which are caused by slightly inaccurately estimated lens distortion parameters.


%\par
%Unlike in modern images where the image coordinate system overlaps with the planimetric camera coordinate system, in historical images the overlap is not maintained due to the scanning procedure. To account for this, an additional 2D transformation is estimated in the course of the processing \cite{article}. \cite{giordano2018toward} demonstrates the importance of the self-calibration of historical images and its impact on 3D accuracy. Poorly modelled intrinsic parameters result in the known \textit{dome}-like deformations that the authors eliminate with the help of automatic GCPs from exiting orthophotos. 
%\par 
%Many cases reported in the literature calculate the image poses with \ac{SfM} using features matched exclusively within the same epoch \cite{nurminen2015automation,cardenal2006use,fox2008unlocking,walstra2004time}. In multi-epoch scenarios, the individual epochs should be defined in a common frame, be it the frame of a reference epoch or an absolute reference frame (i.e., a projection coordinate frame). Control points derived from recent orthophotos and DEM \cite{nurminen2015automation,ellis2006measuring,fox2008unlocking} or GPS survey \cite{micheletti2015application,walstra2004time,cardenal2006use} serve to transform the individual epochs to the common frame. Alternatively, a coarse flight plan may provide an approximate co-registration \cite{giordano2018toward}.
%\par  
%To increase relative accuracy between several epochs,  \cite{cardenal2006use,korpela2006geometrically,micheletti2015application} manually insert inter-epoch tie-points. \cite{giordano2018toward} extracts tie-points between analogue images and recent images based on the method of \cite{aubry2014painting}. \cite{feurer2018joining} joins multi-epoch images in a single \ac{SfM} block based on SIFT-like algorithm \cite{lowe2004distinctive,semyonov2011algorithms} by making the assumption that a sufficient number of feature points remain invariant across each time period. Identifying permanent points over a long time-span has not reached a maturity and research in this domain remains insignificant. Most of the mentioned approaches rely on manual effort, auxiliary input or limiting hypothesis.

\section{Historical image processing}
%\subsection{General processing pipeline}
%Archiving and geoprocessing of historical aerial images: current status in europe,official publication no 70. European Spatial Data Research~\cite{sebastien2019archiving}
%\subsection{Inter-epoch historical images alignment}
%When it comes to inter-epoch historical images, however, directly applying SIFT or SuperGlue often results in inferior results due to large radiometric differences.
\zll{Compared to modern digital images, historical images are accompanied with particular characteristics such as poor radiometric quality and deformation during scanning. Therefore, aligning multi-epoch historical images by directly applying \textit{state-of-the-art} feature matching methods often leads to unsatisfactory results. }
In Figure~\ref{MultiEpochImgPair} we showed an example where SIFT and SuperGlue failed to recover good matches on an inter-epoch image pair with drastic scene changes. It is understandable as (1) SIFT is not sufficiently invariant over time, while (2) SuperGlue is not invariant to rotations larger than 45$^\circ$ and it underperforms on larger images because it was presumably trained on small images.\\
Therefore, many previous researches bypassed the task of extracting inter-epoch matches by processing different epochs separately followed by an inter-epoch co-registration relying on \ac{GCP}s.
Between 10 and 169 \ac{GCP}s are required in \cite{pinto2019archived}, ~\cite{bozek2019analysis}, \cite{persia2020archival}, ~\cite{micheletti2015application} and \cite{molg2017structure}.
Manually measuring \ac{GCP}s are laboursome and tedious. Furthermore, it is difficult to find salient points that are stable over time.\\
Certain attempts were made to extract inter-epoch matches. Giordano et al.~\cite{giordano2018toward} extract feature matches between historical and recent images relying on HoG descriptors~\cite{dalal2005histograms}. The authors require flight plans as input, which are not commonly available as mentioned in Section \ref{chap:intro}. 
Feurer et al.~\cite{feurer2018joining} joins multi-epoch images in a single \ac{SfM} block based on SIFT-like algorithm by making the assumption that a sufficient number of feature points remain invariant across each time period. Their methods are widely used in the subsequent works~\cite{filhol2019time}, ~\cite{cook2019simple},~\cite{parente2021automated} and~\cite{blanch2021multi}. It remains questionable whether the method is capable of handling drastic scene changes.
%Zhang et al.~\cite{zhang2020guided} extract inter-epoch matches from SIFT-detected keypoints based on the hypothesis that points follow 2D and 3D spatial similarity model. This method works in simple cases with few scene changes. 
{Additionally, a stream of previous works focus on historical terrestrial images (~\cite{maiwald2021automatic}, ~\cite{beltrami20193d}, ~\cite{bevilacqua2019reconstruction}, ~\cite{maiwald2019generation}) and historical video recordings (~\cite{maiwald2019generation}). However, their algorithms are not suitable to the aerial case.}\\
%This work is an extention of~\cite{zhang2020guided}. Unlike in~\cite{zhang2020guided}, we introduce a rough co-registration between different epochs based on matching \ac{DSM}s with SuperGlue, and use it to guide a precise matching. Our rough co-registration is robust under extreme scene changes because (1) SuperGlue utilizes context to enhance feature descriptors and (2) \ac{DSM}s are generally stable over time. With the guidance of roughly co-registered orientations and \ac{DSM}s, both SIFT and SuperGlue achieved good performance, as shown in our experiments.
In this work, we propose a rough-to-precise strategy to recover inter-epoch matches, without requiring any \ac{GCP}s or auxiliary data.

\cleardoublepage
%!TEX root = Manuscript.tex

\chapter{Applications and Datasets}
\label{chap:ApplicationsAndDatasets}
%Before revealing the details of the rough-to-precise matching strategy, 
In this section, five sets of datasets categorized in four application scenarios are introduced (cf. Table ~\ref{application}), which will be used to test our pipeline.

\begin{table}[htbp]
	%\scriptsize %\footnotesize
	\centering
	\begin{tabular}{||l|l||c||}\hline
		 \multicolumn{2}{||c||}{Application scenario} & Dataset \\\hline\hline
		 \multirow{2}{*}{General case for change detection} & Urban area & Fr{\'e}jus \\
		 & Rural area & Pezenas \\\hline
		 \multicolumn{2}{||c||}{Earthquake case} & Kobe \\\hline
		 \multicolumn{2}{||c||}{Landslide case} & Alberona \\\hline
		 \multicolumn{2}{||c||}{Glacier case} & Hofsjökull \\\hline
	\end{tabular}
	\caption{Five sets of datasets categorized in four application scenarios.}
	\label{application}
\end{table}

Details of the datasets are listed in Table~\ref{AerialData}, ~\ref{SatelliteData}. 
Images demonstrations of each dataset are displayed in Figure~\ref{FrejusData},~\ref{PezenasData}, ~\ref{KobeData}, ~\ref{AlberonaData} and ~\ref{Hofsjökull}.

%General case for change detection
%	Rural area: Pezenas
%	Urban area: Fr{\'e}jus
%Earthquake case: Kobe
%Landslide case: Alberona
%Glacier case: Hofsjökull

\section{General case for change detection}
%Two sets of datasets are chosen to 
\subsection{Urban area: Fr{\'e}jus}
The dataset Fr{\'e}jus is an urban area, mainly covered with buildings along with scattered farmlands, except a half-moon-shaped bay located in south. It is a 15 $km^2$ rectangular area located in Fr{\'e}jus, a commune in southeastern France. We have four sets of aerial images acquired in 1954, 1966, 1970 and 2014. The epoch 2014 was acquired with the \ac{IGN}'s digital metric camera ~\cite{souchon2010ign}, its orientations are both in global reference frame and precise. Therefore it is treated as \ac{GT} during our processing (in other words, their parameters will be fixed during the \ac{BBA}). 
The area exhibits drastic scene changes in the 60-year period, as can be seen in Figure~\ref{FrejusEvolution}, where evolution of a subregion is displayed.\\

\subsection{Rural area: Pezenas}
The dataset Pezenas is a rural area, mainly covered with vegetation and several sparsely populated urban zones. It is a 420 $km^2$ rectangular area located in Pezenas in the Occitanie region in southern France. We have at our disposal three sets of aerial images acquired in 1971, 1981 and 2015, and one set of satellite images acquired in 2014. Both the epoch 2014 and 2015 are treated as \ac{GT}. In this dataset we are interested in matching historical epochs (1971 and 1981) with aerial \ac{GT} and satellite \ac{GT} individually. The area exhibits changes in scene appearance in the 44-year period.\\

\section{Earthquake case: Kobe}
The dataset Kobe witnessed the well-known Kobe earthquake in January 1995. It is a 90 $km^2$ area of irregular shape located in the north of Awaji Island, Japan. We have two sets of aerial images: pre-event acquired in 1991 and post-event acquired in 1995. It is mainly covered with mountain area and narrow urban zones along the sea. There are neither \ac{GT} epochs nor \ac{GCP}s, therefore we measured 2 points on Google map to scale the result to metric units. In this dataset we are interested in localizing the earthquake fault.

\section{Landslide case: Alberona}
The dataset Alberona is characterized by the diffuse presence of clay rich lithologies which confers to the landscape a typical smooth topography, with the wide presence of slow moving landslides of the slide and slide-earthflow type. It is a 90 $km^2$ rectangular area located in Southern Italy, near the village of Alberona (Puglia region). 
In the study area, the land use is typical of rural areas, poorly inhabited and mainly agricultural and wooded. A slow moving slide-earthflow has been detected there since the 1950s. 
Images were scanned with non photogrammetric scanner with 800 dpi. The films were poorly preserved before scanning, which present some scratches and dust, a typical feature for images that were not preserved for photogrammetric purposes.
There are no \ac{GT} epochs but 7 \ac{GCP}s which could be used to move the results in relative coordinate systems to the absolute one.
In this dataset we are interested in localizing the landslide.

\section{Glacier case: Hofsjökull}
The dataset Hofsjökull is a snow-covered area located in Hofsjökull in central Iceland. Unlike other datasets discribed previously, Hofsjökull consists of only one epoch, as in this dataset we are only interested in matching challenging intra-epoch image pairs. It contains several archival aerial images acquired in the year 1960, provided by the National Survey of Iceland. They were scanned with a photogrammetric scanner Wehrli RM-6, in 16micron/px and 12 bit, in order to digitize as much information as possible appearing in the films. In Figure~\ref{Hofsjökull} we displayed 6 consecutive images in the same flight strip, with snow-covered area gradually expanding. The most challenging image pair (i.e. image 5 and 6, as they are fully snow-covered with very limited context) are chosen for testing. Their common zone is labeled with red rectangles. 

\begin{table}[htbp]
	\scriptsize %\footnotesize
	\centering
	\begin{tabular}{||c|c||c|c|c|c|c|c|c|c|c||}\hline
& \multirow{2}{*}{Epoch} & F & Wid & Hei & GSD & \multirow{2}{*}{F. o.} & \multirow{2}{*}{S. o.} & H & \multirow{2}{*}{Nb} & \multirow{2}{*}{Flightline} \\
&  & [pix] & [mm] & [mm] & [m] & & & [m] & & \\\hline\hline

\multirow{4}{*}{Fr{\'e}jus} & 1954 & 23350 & 300 & 300 & \color{black}0.11 & 60\% & 20\% & 2500 & 19 & West-Est \\
& 1966 & 10230 & 180 & 180 & \color{black}0.17 & 60\% & 30\% & 1700 & 15& West-Est \\
& 1970 & 10230 & 180 & 180 & 0.17 & 60\% & 30\% & 1700 & 19& West-Est \\
& 2014 & \color{black}18281 & 99.28 & 72.42 & 0.35 & 60\% & 30\% & 6500 & 33& West-Est \\\hline\hline

\multirow{4}{*}{Pezenas} & 1971 & 7589 & 230 & 230 & 0.32 &    60\% &    20\% & 2400 & 57& West-Est \\
& 1981 & 7607 & 230 & 230 & 0.59 & 60\% & 20\% & 4500 & 27& West-Est \\
& \multirow{2}{*}{E2015} & 9967.5 & 47 & 35 & 0.46 & 60\% & 50\% & 4600 & 308& West-Est \\
&  & 9204.5 & 50 & 36 & 0.5 & 60\% & 50\% & 4600 & 74& West-Est \\\hline\hline

\multirow{2}{*}{Kobe}& 1991& 7662 & 212 & 212 & 0.5 &    65\% &    35\% & 3800 & 15 & Northeast-Southwest \\
& 1995& 7662 & 212 & 212 & 0.18 & 65\% & 65\% & 1400 & 83& Northeast-Southwest \\\hline\hline

\multirow{2}{*}{Alberona}& 1954 & 4760 & 230 & 230 & 1.0 & 65\% & / & 6000 & 3& North-South \\
& 2003 & 4650 & 230 & 230 & 0.85 & 65\% & 30\% & 4850 & 7 & West-Est \\\hline\hline

Hofsjökull & 1960 & 9656 & 224 & 224 & 0.57 & 60\% & / & 5480 & 6 & North-South \\\hline
	\end{tabular}
	\caption{Aerial dataset details of Fr{\'e}jus, Pezenas, Kobe, Alberona and Hofsjökull. The 2015 acquisition of Pezenas is obtained with two sets of camera. F means focal length, Wid and Hei are the width and height of image, GSD is the ground sampling distance, F.o. and S.o. are forward and side overlap, H is the flying height, Nb is the number of images.}
	\label{AerialData}
\end{table}

%\begin{table}[htbp]
%    \scriptsize %\footnotesize
%    \centering
%    \begin{tabular}{||l|c|c|c|c||c|c|c|c||c|c||}\hline
%        &\multicolumn{4}{c||}{Fr{\'e}jus}&\multicolumn{4}{c||}{Pezenas}&\multicolumn{2}{c||}{Kobe}\\\hline
%                &E1954&E1966&E1970&E2014&E1971&E1981&\multicolumn{2}{c||}{E2015}&E1991&E1995\\\hline\hline
%        F [pix]&23350&10230&10230&\color{black}18281&7589&7607&9967.5&9204.5&7662&7662\\
%        %Size [mm]&\color{black}300,300&\color{black}180,180&\color{black}180,180&99.28,72.42&230230&230230&47,35&50,36&212212&212,212\\
%        Wid [mm]&300&180&180&99.28&230&230&47&50&212&212\\
%        Hei [mm]&300&180&180&72.42&230&230&35&36&212&212\\
%        GSD [m]&\color{black}0.11&\color{black}0.17&0.17&0.35&0.32&0.59&0.46&0.5&0.5&0.18\\
%        F. o.&60\%&60\%&60\%&60\%&   60\%&60\%&60\%&60\%&   65\%&65\%\\
%        S. o.&20\%&30\%&30\%&30\%&   20\%&20\%&50\%&50\%&   35\%&65\%\\
%        H  [m]&2500&1700&1700&6500&2400&4500&4600&4600&3800&1400\\
%        Nb &19&15&19&33&57&27&308&74&15&83\\\hline
%    \end{tabular}
%    \caption{Aerial dataset details of Fr{\'e}jus, Pezenas and Kobe. The 2015 acquisition of Pezenas is obtained with two sets of camera. E stands for epoch, F means focal length, Wid and Hei are the width and height of image, GSD is the ground sampling distance, F.o. and S.o. are forward and side overlap, H is the flying height, Nb is the number of images.}
%    \label{AerialData}
%\end{table}

\begin{table}[htbp]
	\scriptsize %\footnotesize
	\centering
	\begin{tabular}{||l|c|c||}\hline
		& Master image & Secondary image\\\hline
		Constellation & Pleiades & Pleiades \\
		GSD [m] & 0.5 & 0.5\\
		Acquired date & 12/06/2014 & 12/06/2014 \\
		Number of lines & 38468 & 37710 \\
		Number of pixels per line & 34108 & 33392 \\
		Cloud cover & 3.9\% & 4.0\% \\
		Snow cover & 0\% & 0\% \\\hline
		%B/H?
	\end{tabular}
	\caption{Satellite dataset details of Pezenas. GSD means the ground sampling distance.}
	\label{SatelliteData}
\end{table}

\begin{figure*}[htbp]
    \begin{center}
        \subfigure[Fr{\'e}jus 1954 (19 images)]{
            \begin{minipage}[t]{0.4\linewidth}
                \centering
                \includegraphics[width=5.65cm]{images/Chapitre3/Frejus1954.png}
            \end{minipage}%
        }
        \subfigure[Fr{\'e}jus 1966 (15 images)]{
            \begin{minipage}[t]{0.56\linewidth}
                \centering
                \includegraphics[width=7.6cm]{images/Chapitre3/Frejus1966.png}
            \end{minipage}%
        }
        \subfigure[Fr{\'e}jus 1970 (19 images)]{
    \begin{minipage}[t]{0.5\linewidth}
        \centering
        \includegraphics[width=6.1cm]{images/Chapitre3/Frejus1970.png}
    \end{minipage}%
}
\subfigure[Fr{\'e}jus 2014 (36 images)]{
    \begin{minipage}[t]{0.46\linewidth}
        \centering
        \includegraphics[width=6.25cm]{images/Chapitre3/Frejus2014.png}
    \end{minipage}%
}
        \caption{Images demonstration of different aerial epochs in \textbf{Fr{\'e}jus}, image number of each epoch is displayed in the parenthesis of each sub headline. The common zone between all the epochs is indicated by the red rectangles. Graphic scale is demonstrated on epoch 2014 in (d).}
        \label{FrejusData}
    \end{center}
\end{figure*} 

\begin{figure*}[htbp]
	\begin{center}
		\subfigure[Subregion of Fr{\'e}jus 1954]{
			\begin{minipage}[t]{1\linewidth}
				\centering
				\includegraphics[width=13cm]{images/Chapitre3/Frejus1954Sub.png}
			\end{minipage}%
		}
		\subfigure[Subregion of Fr{\'e}jus 2014]{
			\begin{minipage}[t]{1\linewidth}
				\centering
				\includegraphics[width=13cm]{images/Chapitre3/Frejus2014Sub.png}
			\end{minipage}%
		}
		\caption{Evolution of a subregion in Fr{\'e}jus.}
		\label{FrejusEvolution}
	\end{center}
\end{figure*} 

\begin{figure*}[htbp]
    \begin{center}
        \subfigure[Pezenas 1971 (57 images)]{
            \begin{minipage}[t]{0.48\linewidth}
                \centering
                \includegraphics[width=6.45cm]{images/Chapitre3/Pezenas1971.png}
            \end{minipage}%
        }
        \subfigure[Pezenas 1981 (27 images)]{
            \begin{minipage}[t]{0.48\linewidth}
                \centering
                \includegraphics[width=6.68cm]{images/Chapitre3/Pezenas1981.png}
            \end{minipage}%
        }
            \subfigure[Pezenas 2014 (2 satellite images)]{
    	\begin{minipage}[t]{0.48\linewidth}
    		\centering
    		\includegraphics[width=6.6cm]{images/Chapitre3/Pezenas2014.png}
    	\end{minipage}%
    }
        \subfigure[Pezenas 2015 (382 images)]{
            \begin{minipage}[t]{0.48\linewidth}
                \centering
                \includegraphics[width=6.78cm]{images/Chapitre3/Pezenas2015.png}
            \end{minipage}%
        }
        \caption{Images demonstration of different aerial epochs as well as satellite epoch in \textbf{Pezenas}, image number of each epoch is displayed in the parenthesis of each sub headline. There are 2 historical aerial epochs (1971 and 1981) and 2 \ac{GT} epochs (2014 the satellite epoch and 2015 the aerial epoch) in this dataset. The common zone between the historical epochs and the 2014 satellite epoch is indicated by the blue rectangles, while that between historical epochs and the 2015 aerial epoch is in red rectangles. Graphic scales are demonstrated on epoch 2014 and 2015 in (c) and (d).}
        \label{PezenasData}
    \end{center}
\end{figure*} 


\begin{figure*}[htbp]
    \begin{center}
        \subfigure[Kobe 1991 (15 images)]{
            \begin{minipage}[t]{1\linewidth}
                \centering
                \includegraphics[height=13cm,angle=90]{images/Chapitre3/Kobe1991.png}
            \end{minipage}%
        }
        \subfigure[Kobe 1995 (83 images)]{
            \begin{minipage}[t]{1\linewidth}
                \centering
                \includegraphics[height=13cm,angle=90]{images/Chapitre3/Kobe1995.png}
            \end{minipage}%
        }
        \caption{Images demonstration of different aerial epochs in \textbf{Kobe}, image number of each epoch is displayed in the parenthesis of each sub headline. The common zone between all the epochs is indicated by the red rectangles. Graphic scale is demonstrated on epoch 1995 in (b).}
        \label{KobeData}
    \end{center}
\end{figure*} 

\begin{figure*}[htbp]
	\begin{center}
		\subfigure[Alberona 1954 (3 images)]{
			\begin{minipage}[t]{0.48\linewidth}
				\centering
				\includegraphics[width=6.2cm]{images/ChapitreNew/Ortho-MEC-Malt_Tapas_1954.png}
			\end{minipage}%
		}
		\subfigure[Alberona 2003 (7 images)]{
			\begin{minipage}[t]{0.48\linewidth}
				\centering
				\includegraphics[width=6.2cm]{images/ChapitreNew/Ortho-MEC-Malt_Tapas_2003.png}
			\end{minipage}%
		}
		\caption{Images demonstration of different aerial epochs in \textbf{Alberona}, image number of each epoch is displayed in the parenthesis of each sub headline. The common zone between all the epochs is indicated by the red rectangles. Graphic scale is demonstrated on epoch 2003 in (b).}
		\label{AlberonaData}
	\end{center}
\end{figure*} 

\begin{figure*}[htbp]
	\begin{center}
		\subfigure[Image 1]{
			\begin{minipage}[t]{0.31\linewidth}
				\centering
				\includegraphics[width=4.3cm]{images/ChapitreNew/3-6366_crp_8Bits_Zoom8.png}
			\end{minipage}%
		}
		\subfigure[Image 2]{
			\begin{minipage}[t]{0.31\linewidth}
				\centering
				\includegraphics[width=4.3cm]{images/ChapitreNew/3-6367_crp_8Bits_Zoom8.png}
			\end{minipage}%
		}
		\subfigure[Image 3]{
			\begin{minipage}[t]{0.31\linewidth}
				\centering
				\includegraphics[width=4.3cm]{images/ChapitreNew/3-6368_crp_8Bits_Zoom8.png}
			\end{minipage}%
		}
		\subfigure[Image 4]{
			\begin{minipage}[t]{0.31\linewidth}
				\centering
				\includegraphics[width=4.3cm]{images/ChapitreNew/3-6369_crp_8Bits_Zoom8.png}
			\end{minipage}%
		}
		\subfigure[Image 5]{
			\begin{minipage}[t]{0.31\linewidth}
				\centering
				\includegraphics[width=4.3cm]{images/ChapitreNew/3-6370_crp_8Bits_Zoom8.png}
			\end{minipage}%
		}
		\subfigure[Image 6]{
			\begin{minipage}[t]{0.31\linewidth}
				\centering
				\includegraphics[width=4.3cm]{images/ChapitreNew/3-6371_crp_8Bits_Zoom8.png}
			\end{minipage}%
		}
		\caption{Images demonstration of epoch 1960 in \textbf{Hofsjökull}.}
		\label{Hofsjökull}
	\end{center}
\end{figure*} 

\cleardoublepage
%!TEX root = Manuscript.tex

\chapter{Rough co-registration}
\label{chap:intro}
\minitoc

\section{Introduction}
\subsection{Motivation}
\textit{state-of-the-art} feature matching methods (SIFT and SuperGlue) fail on inter-epoch image pairs.
\subsection{Contribution}

\section{Methodology}
Our goal is to improve robustness by building globally consistent transformation model over the whole block.
(1) matching each potential image pair followed with global filtering based on 3D RANSAC
(2) get a global image for each epoch first, apply matching and 2D RANSAC
%零碎匹配,整体inlier
%整体匹配

\subsection{Strategy 1: Match image pairs}
%\subsection{Strategy 1: global filtering}
CoReg-R3D
\subsubsection{SIFT}
\subsubsection{SuperGlue}
\subsection{Strategy 2: Match DSM/Orthophoto}
%\subsection{trategy 2: global matching}
CoReg-DSM
CoReg-Ortho
\subsubsection{SIFT}
\subsubsection{SuperGlue}

\section{Experiments}
\subsection{Datasets}
Frejus, Pezenas, Kobe\\
Show piled image of each epoch?
\subsection{Evaluation}
\subsubsection{Quantitative evaluation}
(1)inlier ratio (RANSAC and GT)
(2)Check point diff
(3)DoD (picture and statistics)
\subsubsection{Qualitative evaluation}
inlier tie points with orthophotos as background
\subsection{Comparison}
%分三部分分别展示3组数据的结果
3*2 methods:
%1张表表示6种方法的(1)和(2),1张大图放6种方法的DoD, inlier tie pt图(R3D, DSM, ortho各3张(无点图,SIFT点图,SpG点图))

\section{Conclusion}

\section{Discussion}

\cleardoublepage
%!TEX root = Manuscript.tex

\chapter{Precise matching}
\label{chap:intro}
\minitoc

\section{Introduction}
\subsection{Motivation}
\subsection{Contribution}
check SIFT scale and rotation\\
\textit{one-to-one tiling scheme}\\
3D ransac?

\section{Methodology}

\begin{figure*}[htbp]
	\begin{center}
		\subfigure[Workflow]{
			\begin{minipage}[t]{1\linewidth}
				\centering
				\includegraphics[width=1\columnwidth]{images/Chapitre4/precisematching.png}
			\end{minipage}%
		}
		\subfigure[Patch matching]{
			\begin{minipage}[t]{0.35\linewidth}
				\centering
				\includegraphics[width=4.5cm]{images/Chapitre4/patchmatching.png}
			\end{minipage}%
		}
		\subfigure[Buffer zone of tiles]{
	\begin{minipage}[t]{0.25\linewidth}
		\centering
		\includegraphics[width=3cm]{images/Chapitre4/tilingScheme.png}
	\end{minipage}%
}
		\subfigure[Guided matching]{
	\begin{minipage}[t]{0.35\linewidth}
		\centering
		\includegraphics[width=4.5cm]{images/Chapitre4/guidedmatching.png}
	\end{minipage}%
}
		\caption{(a) Workflow of precise matching. (b) and (d)   illustrate toy-examples of the patch matching and guided matching, respectively, (c) displays the feature correspondences where $\mathbf{K}^{e_1}$ exceeds the original tile size (dark green area) and therefore will be abandoned.}
		\label{WorkflowPatch}
	\end{center}
\end{figure*}

\begin{figure*}[htbp]
	\begin{center}
			\subfigure[Example of an image pair]{
		\begin{minipage}[t]{1\linewidth}
			\centering
			\includegraphics[width=1\columnwidth]{images/Chapitre4/example.png}
		\end{minipage}%
	}
		\subfigure[Example of patch pairs]{
			\begin{minipage}[t]{1\linewidth}
				\centering
				\includegraphics[width=0.5\columnwidth]{images/Chapitre4/patchexample.png}
			\end{minipage}%
		}
		\caption{(a) Example demonstration of an image pair, the master image ($I^{e_1}$) and secondary image ($I^{e_2}$) are taken at Fr{\'e}jus in 1954 and 2014 individually. (b) Patch pairs resulted from (a).}
		\label{WorkflowPatch}
	\end{center}
\end{figure*}

\begin{figure*}[htbp]
	\begin{center}
		\subfigure[Example of keypoint prediction]{
			\begin{minipage}[t]{1\linewidth}
				\centering
				\includegraphics[width=1\columnwidth]{images/Chapitre4/guidedexample.png}
			\end{minipage}%
		}
		\caption{Example demonstration of keypoint prediction (cross symbols) accompanied with search space (circles) on an image pair, the master image ($I^{e_1}$) and secondary image ($I^{e_2}$) are taken at Fr{\'e}jus in 1954 and 2014 individually.}
		\label{WorkflowPatch}
	\end{center}
\end{figure*}



\section{Experiments}

\section{Conclusion}

\section{Discussion}

\cleardoublepage
%!TEX root = Manuscript.tex

\chapter{Conclusion and Perspective}
\label{chap:conclusion}
\minitoc



\appendix

\cleardoublepage
\mtcaddpart[Appendices]
\part*{Appendices}
%!TEX root = Manuscript.tex

\chapter{Result of rough co-registration}
\label{chap:appendixA}

\section{Matches visualization}
\label{sec:matchViz}
\begin{enumerate}
	\item For Fr{\'e}jus, the reference epoch $E_r$ is 2014, the matches visualizations between free epochs $E_f$ (i.e. epoch 1954, 1966 and 1970) and $E_r$ are displayed in Figure~\ref{MatchVizFrejus1954DSM}, ~\ref{MatchVizFrejus1966DSM} and ~\ref{MatchVizFrejus1970DSM} individually. As can be seen, for all 3 free epochs:\\
	\begin{itemize}
		\item[-] $SIFT_{ImgPairs}$ and $SIFT_{Ortho}$ failed to recover correct matches. It is reasonable as extreme scene changes are present in Fr{\'e}jus, yet SIFT is not sufficiently invariant over time by its very nature.
		\item[-] $SIFT_{DSM}$ found several good matches after RANSAC, thanks to stable information on DSMs. However, the inlier ratio is dangerously low (around 1\%), which makes the RANSAC procedure unstable.
		\item[-] $SuperGlue_{ImgPairs}$, $SuperGlue_{Ortho}$ and $SuperGlue_{DSM}$ succeeded to find enough good matches, with the inlier ratios ranging from 3.3\% to 46.5\%. In general, the inlier ratios raise from $SuperGlue_{ImgPairs}$ to $SuperGlue_{Ortho}$ then to $SuperGlue_{DSM}$.
	\end{itemize}
	%$SIFT_{ImgPairs}$, $SuperGlue_{ImgPairs}$, $SIFT_{Ortho}$, $SuperGlue_{Ortho}$, $SIFT_{DSM}$ and $SuperGlue_{DSM}$
	\item For Pezenas, there are 2 reference epochs $E_r$:\\
	\begin{itemize}
		\item For aerial $E_r$ (i.e. epoch 2015), the matches visualizations between free epochs $E_f$ (i.e. epoch 1971 and 1981) and $E_r$ are displayed in Figure~\ref{MatchVizPezenas1971DSM} and ~\ref{MatchVizPezenas1981DSM}. As can be seen, for all 2 free epochs:\\
		\begin{itemize}
			\item[-] $SIFT_{ImgPairs}$, $SIFT_{Ortho}$ and $SIFT_{DSM}$ successfully recovered enough correct matches (with inlier ratios above 43\%, 1\% and 14\% individually). $SIFT_{Ortho}$ has obviously lower inlier ratio than $SIFT_{ImgPairs}$, because higher resolution is possessed by \textit{ImgPairs}, which improves the result for dataset with moderate scene changes like Pezenas.
			\item[-] $SuperGlue_{ImgPairs}$, $SuperGlue_{Ortho}$ and $SuperGlue_{DSM}$ recovered more total matches along with generally higher inlier ratio than SIFT. The inlier ratio of $SuperGlue_{Ortho}$ is above 76\%, significantly larger than $SIFT_{Ortho}$, which we attribute to 2 reasons: (1) SuperGlue is more invariant over time as it is trained on multi-epoch images and (2) our tiling scheme. 
		\end{itemize}
		\item For satellite $E_r$ (i.e. epoch 2014), the matches visualizations between free epochs $E_f$ (i.e. epoch 1971 and 1981) and $E_r$ are shown in Figure~\ref{MatchVizPezenas-Satellite1971DSM} and ~\ref{MatchVizPezenas-Satellite1981DSM}. As can be seen:\\
		\begin{itemize}
			\item[-] The results are inferior compared to the ones on aerial $E_r$, since satellite epoch not only has more limited zone overlapped with the free epochs, especially for epoch 1971, but also is covered with clouds.
			\item[-] $SIFT_{Ortho}$ failed on both free epochs, while $SIFT_{DSM}$ found enough good matches with inlier ratio around 4\%, as landscapes are more informative.
			\item[-] $SuperGlue_{Ortho}$ failed on epoch 1971 yet succeeded on epoch 1981, as larger common zone provides more clues in context to ensure the matching performance. $SuperGlue_{DSM}$ recovered more good matches with larger inlier ratio than $SIFT_{DSM}$ on both epochs.
		\end{itemize}
	\end{itemize}
	\item For Kobe, the reference epoch $E_r$ is 1995, the matches visualizations between free epoch $E_f$ (i.e. epoch 1991) and $E_r$ are displayed in Figure~\ref{MatchVizKobe1991DSM}.
	\begin{itemize}
		\item[-] $SIFT_{ImgPairs}$ and $SIFT_{Ortho}$ failed while $SIFT_{DSM}$ performs well.
		\item[-] $SuperGlue_{ImgPairs}$, $SuperGlue_{Ortho}$ and $SuperGlue_{DSM}$ successfully found good matches with inlier ratio over 32\% .
	\end{itemize}
\end{enumerate}

\begin{figure*}[htbp]
	\begin{center}
		\subfigure[Image pairs (19$\times$36 pairs)]{
			\begin{minipage}[t]{0.48\linewidth}
				\centering
				\includegraphics[width=6.8cm]{images/Chapitre3/Pseudo-Ortho-MEC-Malt_Tapas_1954_Ortho-MEC-Malt_2014.png}
			\end{minipage}%
		}
		\subfigure[Number of recovered matches(\textit{ImgPairs})]{
			\begin{minipage}[t]{0.48\linewidth}
				\centering
				\includegraphics[width=3.5cm,trim=0 20 0 38,clip]{images/Chapitre3/PlotCurves_Pseudo-Ortho-MEC-Malt_Tapas_1954_Ortho-MEC-Malt_2014.png}
			\end{minipage}%
		}
		\subfigure[$SIFT_{ImgPairs}^{RANSAC Inliers}$]{
			\begin{minipage}[t]{0.48\linewidth}
				\centering
				\includegraphics[width=6.8cm]{images/Chapitre3/Pseudo-Homol-SIFT2Step_1954-2014-Rough-2DRANSAC-GlobalR3D-PileImg_Ortho-MEC-Malt_Tapas_1954_Ortho-MEC-Malt_2014.png}
			\end{minipage}%
		}
		\subfigure[$SuperGlue_{ImgPairs}^{RANSAC Inliers}$]{
			\begin{minipage}[t]{0.48\linewidth}
				\centering
				\includegraphics[width=6.8cm]{images/Chapitre3/Pseudo-Homol-SuperGlue_1954-2014-GlobalR3D-PileImg_Ortho-MEC-Malt_Tapas_1954_Ortho-MEC-Malt_2014.png}
			\end{minipage}%
		}
		%        \caption{Result of matching image pairs (i.e. \textit{ImgPairs}) of Fr{\'e}jus 1954 and 2014. (a) Image pairs to be matched, with red rectangles indicating the common zone. (b) Numbers of total matches and RANSAC inliers of both SIFT and SuperGlue. (c) Visualization of RANSAC inliers based on SIFT. (d) Visualization of RANSAC inliers based on SuperGlue.}
		%        \label{MatchVizFrejus1954ImgPairs}
		%    \end{center}
		%\end{figure*} 
		%
		%
		%
		%\begin{figure*}[htbp]
		%    \begin{center}
		\subfigure[Orthophotos]{
			\begin{minipage}[t]{0.48\linewidth}
				\centering
				\includegraphics[width=6.8cm]{images/Chapitre3/Ortho-MEC-Malt_Tapas_1954_Ortho-MEC-Malt_2014.png}
			\end{minipage}%
		}
		\subfigure[Number of recovered matches(\textit{Ortho})]{
			\begin{minipage}[t]{0.48\linewidth}
				\centering
				\includegraphics[width=3.5cm,trim=0 20 0 38,clip]{images/Chapitre3/PlotCurves_Ortho-MEC-Malt_Tapas_1954_Ortho-MEC-Malt_2014.png}
			\end{minipage}%
		}
		\subfigure[$SIFT_{Ortho}^{RANSAC Inliers}$]{
			\begin{minipage}[t]{0.48\linewidth}
				\centering
				\includegraphics[width=6.8cm]{images/Chapitre3/Homol-SIFT2Step-Rough-2DRANSAC_Ortho-MEC-Malt_Tapas_1954_Ortho-MEC-Malt_2014.png}
			\end{minipage}%
		}
		\subfigure[$SuperGlue_{Ortho}^{RANSAC Inliers}$]{
			\begin{minipage}[t]{0.48\linewidth}
				\centering
				\includegraphics[width=6.8cm]{images/Chapitre3/Homol-SubPatch_R270-2DRANSAC_Ortho-MEC-Malt_Tapas_1954_Ortho-MEC-Malt_2014.png}
			\end{minipage}%
		}
		%        \caption{Result of matching orthophotos (i.e. \textit{Ortho}) of Fr{\'e}jus 1954 and 2014. (a) Orthophotos to be matched, with red rectangles indicating the common zone. (b) Numbers of total matches and RANSAC inliers of both SIFT and SuperGlue. (c) Visualization of RANSAC inliers based on SIFT. (d) Visualization of RANSAC inliers based on SuperGlue.}
		%        \label{MatchVizFrejus1954Ortho}
		%    \end{center}
		%\end{figure*} 
		%
		%\begin{figure*}[htbp]
		%    \begin{center}
		\subfigure[DSMs]{
			\begin{minipage}[t]{0.48\linewidth}
				\centering
				\includegraphics[width=6.8cm]{images/Chapitre3/MEC-Malt_Tapas_1954_MEC-Malt_2014.png}
			\end{minipage}%
		}
		\subfigure[Number of recovered matches(\textit{DSM})]{
			\begin{minipage}[t]{0.48\linewidth}
				\centering
				\includegraphics[width=3.5cm,trim=0 20 0 38,clip]{images/Chapitre3/PlotCurves_MEC-Malt_Tapas_1954_MEC-Malt_2014.png}
			\end{minipage}%
		}
		\subfigure[$SIFT_{DSM}^{RANSAC Inliers}$]{
			\begin{minipage}[t]{0.48\linewidth}
				\centering
				\includegraphics[width=6.8cm]{images/Chapitre3/Homol-SIFT2Step-Rough-2DRANSAC_MEC-Malt_Tapas_1954_MEC-Malt_2014.png}
			\end{minipage}%
		}
		\subfigure[$SuperGlue_{DSM}^{RANSAC Inliers}$]{
			\begin{minipage}[t]{0.48\linewidth}
				\centering
				\includegraphics[width=6.8cm]{images/Chapitre3/Homol-SubPatch_R270-2DRANSAC_MEC-Malt_Tapas_1954_MEC-Malt_2014.png}
			\end{minipage}%
		}
		%\caption{Result of matching DSMs (i.e. \textit{DSM}) of Fr{\'e}jus 1954 and 2014. (a) DSMs to be matched, with red rectangles indicating the common zone. (b) Numbers of total matches and RANSAC inliers of both SIFT and SuperGlue. (c) Visualization of RANSAC inliers based on SIFT. (d) Visualization of RANSAC inliers based on SuperGlue.}
		\caption{{\scriptsize Result of \textit{ImgPairs} (a-d), \textit{Ortho} (e-h) and \textit{DSM} (i-l) on matching \textbf{Fr{\'e}jus 1954 and 2014}. (a, e, i) Image pairs/orthophotos/DSMs to be matched, with red rectangles indicating the common zone. (b, f, j) Numbers of total matches and RANSAC inliers of both SIFT and SuperGlue on methods \textit{ImgPairs}, \textit{Ortho} and \textit{DSM} individually. (c, g, k) Visualization of RANSAC inliers based on $SIFT_{ImgPairs}$, $SIFT_{Ortho}$ and $SIFT_{DSM}$. (d, h, l) Visualization of RANSAC inliers based on $SuperGlue_{ImgPairs}$, $SuperGlue_{Ortho}$ and $SuperGlue_{DSM}$.}}
		\label{MatchVizFrejus1954DSM}
	\end{center}
\end{figure*} 



\begin{figure*}[htbp]
	\begin{center}
		\subfigure[Image pairs (15$\times$36 pairs)]{
			\begin{minipage}[t]{0.48\linewidth}
				\centering
				\includegraphics[width=6.1cm]{images/Chapitre3/Pseudo-Ortho-MEC-Malt_Tapas_1966_Ortho-MEC-Malt_2014.png}
			\end{minipage}%
		}
		\subfigure[Number of recovered matches(\textit{ImgPairs})]{
			\begin{minipage}[t]{0.48\linewidth}
				\centering
				\includegraphics[width=3.5cm,trim=0 20 0 38,clip]{images/Chapitre3/PlotCurves_Pseudo-Ortho-MEC-Malt_Tapas_1966_Ortho-MEC-Malt_2014.png}
			\end{minipage}%
		}
		\subfigure[$SIFT_{ImgPairs}^{RANSAC Inliers}$]{
			\begin{minipage}[t]{0.48\linewidth}
				\centering
				\includegraphics[width=6cm]{images/Chapitre3/Pseudo-Homol-SIFT2Step_1966-2014-Rough-2DRANSAC-GlobalR3D-PileImg_Ortho-MEC-Malt_Tapas_1966_Ortho-MEC-Malt_2014.png}
			\end{minipage}%
		}
		\subfigure[$SuperGlue_{ImgPairs}^{RANSAC Inliers}$]{
			\begin{minipage}[t]{0.48\linewidth}
				\centering
				\includegraphics[width=6cm]{images/Chapitre3/Pseudo-Homol-SuperGlue_1966-2014-GlobalR3D-PileImg_Ortho-MEC-Malt_Tapas_1966_Ortho-MEC-Malt_2014.png}
			\end{minipage}%
		}
		%        \caption{Result of matching image pairs (i.e. \textit{ImgPairs}) of Fr{\'e}jus 1966 and 2014. (a) Image pairs to be matched, with red rectangles indicating the common zone. (b) Numbers of total matches and RANSAC inliers of both SIFT and SuperGlue. (c) Visualization of RANSAC inliers based on SIFT. (d) Visualization of RANSAC inliers based on SuperGlue.}
		%        \label{MatchVizFrejus1966ImgPairs}
		%    \end{center}
		%\end{figure*} 
		%
		%
		%\begin{figure*}[htbp]
		%    \begin{center}
		\subfigure[Orthophotos]{
			\begin{minipage}[t]{0.48\linewidth}
				\centering
				\includegraphics[width=6.1cm]{images/Chapitre3/Ortho-MEC-Malt_Tapas_1966_Ortho-MEC-Malt_2014.png}
			\end{minipage}%
		}
		\subfigure[Number of recovered matches(\textit{Ortho})]{
			\begin{minipage}[t]{0.48\linewidth}
				\centering
				\includegraphics[width=3.5cm,trim=0 20 0 38,clip]{images/Chapitre3/PlotCurves_Ortho-MEC-Malt_Tapas_1966_Ortho-MEC-Malt_2014.png}
			\end{minipage}%
		}
		\subfigure[$SIFT_{Ortho}^{RANSAC Inliers}$]{
			\begin{minipage}[t]{0.48\linewidth}
				\centering
				\includegraphics[width=6cm]{images/Chapitre3/Homol-SIFT2Step-Rough-2DRANSAC_Ortho-MEC-Malt_Tapas_1966_Ortho-MEC-Malt_2014.png}
			\end{minipage}%
		}
		\subfigure[$SuperGlue_{Ortho}^{RANSAC Inliers}$]{
			\begin{minipage}[t]{0.48\linewidth}
				\centering
				\includegraphics[width=6cm]{images/Chapitre3/Homol-SubPatch_R270-2DRANSAC_Ortho-MEC-Malt_Tapas_1966_Ortho-MEC-Malt_2014.png}
			\end{minipage}%
		}
		%        \caption{Result of matching orthophotos (i.e. \textit{Ortho}) of Fr{\'e}jus 1966 and 2014. (a) Orthophotos to be matched, with red rectangles indicating the common zone. (b) Numbers of total matches and RANSAC inliers of both SIFT and SuperGlue. (c) Visualization of RANSAC inliers based on SIFT. (d) Visualization of RANSAC inliers based on SuperGlue.}
		%        \label{MatchVizFrejus1966Ortho}
		%    \end{center}
		%\end{figure*} 
		%
		%\begin{figure*}[htbp]
		%    \begin{center}
		\subfigure[DSMs]{
			\begin{minipage}[t]{0.48\linewidth}
				\centering
				\includegraphics[width=6.1cm]{images/Chapitre3/MEC-Malt_Tapas_1966_MEC-Malt_2014.png}
			\end{minipage}%
		}
		\subfigure[Number of recovered matches(\textit{DSM})]{
			\begin{minipage}[t]{0.48\linewidth}
				\centering
				\includegraphics[width=3.5cm,trim=0 20 0 38,clip]{images/Chapitre3/PlotCurves_MEC-Malt_Tapas_1966_MEC-Malt_2014.png}
			\end{minipage}%
		}
		\subfigure[$SIFT_{DSM}^{RANSAC Inliers}$]{
			\begin{minipage}[t]{0.48\linewidth}
				\centering
				\includegraphics[width=6cm]{images/Chapitre3/Homol-SIFT2Step-Rough-2DRANSAC_MEC-Malt_Tapas_1966_MEC-Malt_2014.png}
			\end{minipage}%
		}
		\subfigure[$SuperGlue_{DSM}^{RANSAC Inliers}$]{
			\begin{minipage}[t]{0.48\linewidth}
				\centering
				\includegraphics[width=6cm]{images/Chapitre3/Homol-SubPatch_R270-2DRANSAC_MEC-Malt_Tapas_1966_MEC-Malt_2014.png}
			\end{minipage}%
		}
		%\caption{Result of matching DSMs (i.e. \textit{DSM}) of Fr{\'e}jus 1966 and 2014. (a) DSMs to be matched, with red rectangles indicating the common zone. (b) Numbers of total matches and RANSAC inliers of both SIFT and SuperGlue. (c) Visualization of RANSAC inliers based on SIFT. (d) Visualization of RANSAC inliers based on SuperGlue.}
		%\caption{{\small Result of \textit{ImgPairs}, \textit{Ortho} and \textit{DSM} on Fr{\'e}jus 1966 and 2014. (a, e, i) Image pairs/orthophotos/DSMs to be matched, with red rectangles indicating the common zone. (b, f, j) Numbers of total matches and RANSAC inliers of both SIFT and SuperGlue on methods \textit{ImgPairs}, \textit{Ortho} and \textit{DSM} individually. (c, g, k) Visualization of RANSAC inliers based on $SIFT_{ImgPairs}$, $SIFT_{Ortho}$ and $SIFT_{DSM}$. (d, h, l) Visualization of RANSAC inliers based on $SuperGlue_{ImgPairs}$, $SuperGlue_{Ortho}$ and $SuperGlue_{DSM}$.}}
		\caption{{\scriptsize Result of \textit{ImgPairs} (a-d), \textit{Ortho} (e-h) and \textit{DSM} (i-l) on matching \textbf{Fr{\'e}jus 1966 and 2014}. (a, e, i) Image pairs/orthophotos/DSMs to be matched, with red rectangles indicating the common zone. (b, f, j) Numbers of total matches and RANSAC inliers of both SIFT and SuperGlue on methods \textit{ImgPairs}, \textit{Ortho} and \textit{DSM} individually. (c, g, k) Visualization of RANSAC inliers based on $SIFT_{ImgPairs}$, $SIFT_{Ortho}$ and $SIFT_{DSM}$. (d, h, l) Visualization of RANSAC inliers based on $SuperGlue_{ImgPairs}$, $SuperGlue_{Ortho}$ and $SuperGlue_{DSM}$.}}
		\label{MatchVizFrejus1966DSM}
	\end{center}
\end{figure*} 



\begin{figure*}[htbp]
	\begin{center}
		\subfigure[Image pairs (19$\times$36 pairs)]{
			\begin{minipage}[t]{0.48\linewidth}
				\centering
				\includegraphics[width=5.2cm]{images/Chapitre3/Pseudo-Ortho-MEC-Malt_Tapas_1970_Ortho-MEC-Malt_2014.png}
			\end{minipage}%
		}
		\subfigure[Number of recovered matches(\textit{ImgPairs})]{
			\begin{minipage}[t]{0.48\linewidth}
				\centering
				\includegraphics[width=3.5cm,trim=0 20 0 38,clip]{images/Chapitre3/PlotCurves_Pseudo-Ortho-MEC-Malt_Tapas_1970_Ortho-MEC-Malt_2014.png}
			\end{minipage}%
		}
		\subfigure[$SIFT_{ImgPairs}^{RANSAC Inliers}$]{
			\begin{minipage}[t]{0.48\linewidth}
				\centering
				\includegraphics[width=5.7cm]{images/Chapitre3/Pseudo-Homol-SIFT2Step_1970-2014-Rough-2DRANSAC-GlobalR3D-PileImg_Ortho-MEC-Malt_Tapas_1970_Ortho-MEC-Malt_2014.png}
			\end{minipage}%
		}
		\subfigure[$SuperGlue_{ImgPairs}^{RANSAC Inliers}$]{
			\begin{minipage}[t]{0.48\linewidth}
				\centering
				\includegraphics[width=5.7cm]{images/Chapitre3/Pseudo-Homol-SuperGlue_1970-2014-GlobalR3D-PileImg_Ortho-MEC-Malt_Tapas_1970_Ortho-MEC-Malt_2014.png}
			\end{minipage}%
		}
		%        \caption{Result of matching image pairs (i.e. \textit{ImgPairs}) of Fr{\'e}jus 1970 and 2014. (a) Image pairs to be matched, with red rectangles indicating the common zone. (b) Numbers of total matches and RANSAC inliers of both SIFT and SuperGlue. (c) Visualization of RANSAC inliers based on SIFT. (d) Visualization of RANSAC inliers based on SuperGlue.}
		%        \label{MatchVizFrejus1970ImgPairs}
		%    \end{center}
		%\end{figure*} 
		%
		%\begin{figure*}[htbp]
		%    \begin{center}
		\subfigure[Orthophotos]{
			\begin{minipage}[t]{0.48\linewidth}
				\centering
				\includegraphics[width=5.2cm]{images/Chapitre3/Ortho-MEC-Malt_Tapas_1970_Ortho-MEC-Malt_2014.png}
			\end{minipage}%
		}
		\subfigure[Number of recovered matches(\textit{Ortho})]{
			\begin{minipage}[t]{0.48\linewidth}
				\centering
				\includegraphics[width=3.5cm,trim=0 20 0 38,clip]{images/Chapitre3/PlotCurves_Ortho-MEC-Malt_Tapas_1970_Ortho-MEC-Malt_2014.png}
			\end{minipage}%
		}
		\subfigure[$SIFT_{Ortho}^{RANSAC Inliers}$]{
			\begin{minipage}[t]{0.48\linewidth}
				\centering
				\includegraphics[width=5.7cm]{images/Chapitre3/Homol-SIFT2Step-Rough-2DRANSAC_Ortho-MEC-Malt_Tapas_1970_Ortho-MEC-Malt_2014.png}
			\end{minipage}%
		}
		\subfigure[$SuperGlue_{Ortho}^{RANSAC Inliers}$]{
			\begin{minipage}[t]{0.48\linewidth}
				\centering
				\includegraphics[width=5.7cm]{images/Chapitre3/Homol-SubPatch_R270-2DRANSAC_Ortho-MEC-Malt_Tapas_1970_Ortho-MEC-Malt_2014.png}
			\end{minipage}%
		}
		%        \caption{Result of matching orthophotos (i.e. \textit{Ortho}) of Fr{\'e}jus 1970 and 2014. (a) Orthophotos to be matched, with red rectangles indicating the common zone. (b) Numbers of total matches and RANSAC inliers of both SIFT and SuperGlue. (c) Visualization of RANSAC inliers based on SIFT. (d) Visualization of RANSAC inliers based on SuperGlue.}
		%        \label{MatchVizFrejus1970Ortho}
		%    \end{center}
		%\end{figure*} 
		%
		%\begin{figure*}[htbp]
		%    \begin{center}
		\subfigure[DSMs]{
			\begin{minipage}[t]{0.48\linewidth}
				\centering
				\includegraphics[width=5.2cm]{images/Chapitre3/MEC-Malt_Tapas_1970_MEC-Malt_2014.png}
			\end{minipage}%
		}
		\subfigure[Number of recovered matches(\textit{DSM})]{
			\begin{minipage}[t]{0.48\linewidth}
				\centering
				\includegraphics[width=3.5cm,trim=0 20 0 38,clip]{images/Chapitre3/PlotCurves_MEC-Malt_Tapas_1970_MEC-Malt_2014.png}
			\end{minipage}%
		}
		\subfigure[$SIFT_{DSM}^{RANSAC Inliers}$]{
			\begin{minipage}[t]{0.48\linewidth}
				\centering
				\includegraphics[width=5.7cm]{images/Chapitre3/Homol-SIFT2Step-Rough-2DRANSAC_MEC-Malt_Tapas_1970_MEC-Malt_2014.png}
			\end{minipage}%
		}
		\subfigure[$SuperGlue_{DSM}^{RANSAC Inliers}$]{
			\begin{minipage}[t]{0.48\linewidth}
				\centering
				\includegraphics[width=5.7cm]{images/Chapitre3/Homol-SubPatch_R270-2DRANSAC_MEC-Malt_Tapas_1970_MEC-Malt_2014.png}
			\end{minipage}%
		}
		%\caption{Result of matching DSMs (i.e. \textit{DSM}) of Fr{\'e}jus 1970 and 2014. (a) DSMs to be matched, with red rectangles indicating the common zone. (b) Numbers of total matches and RANSAC inliers of both SIFT and SuperGlue. (c) Visualization of RANSAC inliers based on SIFT. (d) Visualization of RANSAC inliers based on SuperGlue.}
		%\caption{{\small Result of \textit{ImgPairs}, \textit{Ortho} and \textit{DSM} on Fr{\'e}jus 1970 and 2014. (a, e, i) image pairs/orthophotos/DSMs to be matched, with red rectangles indicating the common zone. (b, f, j) Numbers of total matches and RANSAC inliers of both SIFT and SuperGlue on methods \textit{ImgPairs}, \textit{Ortho} and \textit{DSM} individually. (c, g, k) Visualization of RANSAC inliers based on $SIFT_{ImgPairs}$, $SIFT_{Ortho}$ and $SIFT_{DSM}$. (d, h, l) Visualization of RANSAC inliers based on $SuperGlue_{ImgPairs}$, $SuperGlue_{Ortho}$ and $SuperGlue_{DSM}$.}}
		\caption{{\scriptsize Result of \textit{ImgPairs} (a-d), \textit{Ortho} (e-h) and \textit{DSM} (i-l) on matching \textbf{Fr{\'e}jus 1970 and 2014}. (a, e, i) Image pairs/orthophotos/DSMs to be matched, with red rectangles indicating the common zone. (b, f, j) Numbers of total matches and RANSAC inliers of both SIFT and SuperGlue on methods \textit{ImgPairs}, \textit{Ortho} and \textit{DSM} individually. (c, g, k) Visualization of RANSAC inliers based on $SIFT_{ImgPairs}$, $SIFT_{Ortho}$ and $SIFT_{DSM}$. (d, h, l) Visualization of RANSAC inliers based on $SuperGlue_{ImgPairs}$, $SuperGlue_{Ortho}$ and $SuperGlue_{DSM}$.}}
		\label{MatchVizFrejus1970DSM}
	\end{center}
\end{figure*} 


%%%%%%%%%%%%%%%%%%%%%%%%%%%%%%%%%%%%%%Pezenas Aerial
\begin{figure*}[htbp]
	\begin{center}
		\subfigure[Image pairs (57$\times$382 pairs)]{
			\begin{minipage}[t]{0.48\linewidth}
				\centering
				\includegraphics[width=5.1cm]{images/Chapitre3/Pseudo-Ortho-MEC-Malt_Tapas_1971_Ortho-MEC-Malt_2015.png}
			\end{minipage}%
		}
		\subfigure[Number of recovered matches(\textit{ImgPairs})]{
			\begin{minipage}[t]{0.48\linewidth}
				\centering
				\includegraphics[width=3.5cm,trim=0 20 0 38,clip]{images/Chapitre3/PlotCurves_Pseudo-Ortho-MEC-Malt_Tapas_1971_Ortho-MEC-Malt_2015.png}
			\end{minipage}%
		}
		\subfigure[$SIFT_{ImgPairs}^{RANSAC Inliers}$]{
			\begin{minipage}[t]{0.48\linewidth}
				\centering
				\includegraphics[width=5.2cm]{images/Chapitre3/Pseudo-Homol-SIFT2Step_1971-2015-Rough-2DRANSAC-GlobalR3D-PileImg_Ortho-MEC-Malt_Tapas_1971_Ortho-MEC-Malt_2015.png}
			\end{minipage}%
		}
		\subfigure[$SuperGlue_{ImgPairs}^{RANSAC Inliers}$]{
			\begin{minipage}[t]{0.48\linewidth}
				\centering
				\includegraphics[width=5.2cm]{images/Chapitre3/Pseudo-Homol-SuperGlue_1971-2015-GlobalR3D-PileImg_Ortho-MEC-Malt_Tapas_1971_Ortho-MEC-Malt_2015.png}
			\end{minipage}%
		}
		%        \caption{Result of matching image pairs (i.e. \textit{ImgPairs}) of Pezenas 1971 and 2015. (a) Image pairs to be matched, with red rectangles indicating the common zone. (b) Numbers of total matches and RANSAC inliers of both SIFT and SuperGlue. (c) Visualization of RANSAC inliers based on SIFT. (d) Visualization of RANSAC inliers based on SuperGlue.}
		%        \label{MatchVizPezenas1971ImgPairs}
		%    \end{center}
		%\end{figure*} 
		%
		%\begin{figure*}[htbp]
		%    \begin{center}
		\subfigure[Orthophotos]{
			\begin{minipage}[t]{0.48\linewidth}
				\centering
				\includegraphics[width=5.1cm]{images/Chapitre3/Ortho-MEC-Malt_Tapas_1971_Ortho-MEC-Malt_2015.png}
			\end{minipage}%
		}
		\subfigure[Number of recovered matches(\textit{Ortho})]{
			\begin{minipage}[t]{0.48\linewidth}
				\centering
				\includegraphics[width=3.5cm,trim=0 20 0 38,clip]{images/Chapitre3/PlotCurves_Ortho-MEC-Malt_Tapas_1971_Ortho-MEC-Malt_2015.png}
			\end{minipage}%
		}
		\subfigure[$SIFT_{Ortho}^{RANSAC Inliers}$]{
			\begin{minipage}[t]{0.48\linewidth}
				\centering
				\includegraphics[width=5.2cm]{images/Chapitre3/Homol-SIFT2Step-Rough-2DRANSAC_Ortho-MEC-Malt_Tapas_1971_Ortho-MEC-Malt_2015.png}
			\end{minipage}%
		}
		\subfigure[$SuperGlue_{Ortho}^{RANSAC Inliers}$]{
			\begin{minipage}[t]{0.48\linewidth}
				\centering
				\includegraphics[width=5.2cm]{images/Chapitre3/Homol-SubPatch_R270-2DRANSAC_Ortho-MEC-Malt_Tapas_1971_Ortho-MEC-Malt_2015.png}
			\end{minipage}%
		}
		%        \caption{Result of matching orthophotos (i.e. \textit{Ortho}) of Pezenas 1971 and 2015. (a) Orthophotos to be matched, with red rectangles indicating the common zone. (b) Numbers of total matches and RANSAC inliers of both SIFT and SuperGlue. (c) Visualization of RANSAC inliers based on SIFT. (d) Visualization of RANSAC inliers based on SuperGlue.}
		%        \label{MatchVizPezenas1971Ortho}
		%    \end{center}
		%\end{figure*} 
		%
		%\begin{figure*}[htbp]
		%    \begin{center}
		\subfigure[DSMs]{
			\begin{minipage}[t]{0.48\linewidth}
				\centering
				\includegraphics[width=5.1cm]{images/Chapitre3/MEC-Malt_Tapas_1971_MEC-Malt_2015.png}
			\end{minipage}%
		}
		\subfigure[Number of recovered matches(\textit{DSM})]{
			\begin{minipage}[t]{0.48\linewidth}
				\centering
				\includegraphics[width=3.5cm,trim=0 20 0 38,clip]{images/Chapitre3/PlotCurves_MEC-Malt_Tapas_1971_MEC-Malt_2015.png}
			\end{minipage}%
		}
		\subfigure[$SIFT_{DSM}^{RANSAC Inliers}$]{
			\begin{minipage}[t]{0.48\linewidth}
				\centering
				\includegraphics[width=5.2cm]{images/Chapitre3/Homol-SIFT2Step-Rough-2DRANSAC_MEC-Malt_Tapas_1971_MEC-Malt_2015.png}
			\end{minipage}%
		}
		\subfigure[$SuperGlue_{DSM}^{RANSAC Inliers}$]{
			\begin{minipage}[t]{0.48\linewidth}
				\centering
				\includegraphics[width=5.2cm]{images/Chapitre3/Homol-SubPatch_R270-2DRANSAC_MEC-Malt_Tapas_1971_MEC-Malt_2015.png}
			\end{minipage}%
		}
		%\caption{Result of matching DSMs (i.e. \textit{DSM}) of Pezenas 1971 and 2015. (a) DSMs to be matched, with red rectangles indicating the common zone. (b) Numbers of total matches and RANSAC inliers of both SIFT and SuperGlue. (c) Visualization of RANSAC inliers based on SIFT. (d) Visualization of RANSAC inliers based on SuperGlue.}
		%\caption{{\small Result of \textit{ImgPairs}, \textit{Ortho} and \textit{DSM} on Pezenas 1971 and 2015. (a, e, i) image pairs/orthophotos/DSMs to be matched, with red rectangles indicating the common zone. (b, f, j) Numbers of total matches and RANSAC inliers of both SIFT and SuperGlue on methods \textit{ImgPairs}, \textit{Ortho} and \textit{DSM} individually. (c, g, k) Visualization of RANSAC inliers based on $SIFT_{ImgPairs}$, $SIFT_{Ortho}$ and $SIFT_{DSM}$. (d, h, l) Visualization of RANSAC inliers based on $SuperGlue_{ImgPairs}$, $SuperGlue_{Ortho}$ and $SuperGlue_{DSM}$.}}
		\caption{{\scriptsize Result of \textit{ImgPairs} (a-d), \textit{Ortho} (e-h) and \textit{DSM} (i-l) on matching \textbf{Pezenas 1971 and 2015}. (a, e, i) Image pairs/orthophotos/DSMs to be matched, with red rectangles indicating the common zone. (b, f, j) Numbers of total matches and RANSAC inliers of both SIFT and SuperGlue on methods \textit{ImgPairs}, \textit{Ortho} and \textit{DSM} individually. (c, g, k) Visualization of RANSAC inliers based on $SIFT_{ImgPairs}$, $SIFT_{Ortho}$ and $SIFT_{DSM}$. (d, h, l) Visualization of RANSAC inliers based on $SuperGlue_{ImgPairs}$, $SuperGlue_{Ortho}$ and $SuperGlue_{DSM}$.}}        
		\label{MatchVizPezenas1971DSM}
	\end{center}
\end{figure*} 



\begin{figure*}[htbp]
	\begin{center}
		\subfigure[Image pairs (27$\times$382 pairs)]{
			\begin{minipage}[t]{0.48\linewidth}
				\centering
				\includegraphics[width=5.1cm]{images/Chapitre3/Pseudo-Ortho-MEC-Malt_Tapas_1981_Ortho-MEC-Malt_2015.png}
			\end{minipage}%
		}
		\subfigure[Number of recovered matches(\textit{ImgPairs})]{
			\begin{minipage}[t]{0.48\linewidth}
				\centering
				\includegraphics[width=3.5cm,trim=0 20 0 38,clip]{images/Chapitre3/PlotCurves_Pseudo-Ortho-MEC-Malt_Tapas_1981_Ortho-MEC-Malt_2015.png}
			\end{minipage}%
		}
		\subfigure[$SIFT_{ImgPairs}^{RANSAC Inliers}$]{
			\begin{minipage}[t]{0.48\linewidth}
				\centering
				\includegraphics[width=5.2cm]{images/Chapitre3/Pseudo-Homol-SIFT2Step_1981-2015-Rough-2DRANSAC-GlobalR3D-PileImg_Ortho-MEC-Malt_Tapas_1981_Ortho-MEC-Malt_2015.png}
			\end{minipage}%
		}
		\subfigure[$SuperGlue_{ImgPairs}^{RANSAC Inliers}$]{
			\begin{minipage}[t]{0.48\linewidth}
				\centering
				\includegraphics[width=5.2cm]{images/Chapitre3/Pseudo-Homol-SuperGlue_1981-2015-GlobalR3D-PileImg_Ortho-MEC-Malt_Tapas_1981_Ortho-MEC-Malt_2015.png}
			\end{minipage}%
		}
		%        \caption{Result of matching image pairs (i.e. \textit{ImgPairs}) of Pezenas 1981 and 2015. (a) Image pairs to be matched, with red rectangles indicating the common zone. (b) Numbers of total matches and RANSAC inliers of both SIFT and SuperGlue. (c) Visualization of RANSAC inliers based on SIFT. (d) Visualization of RANSAC inliers based on SuperGlue.}
		%        \label{MatchVizPezenas1981ImgPairs}
		%    \end{center}
		%\end{figure*} 
		%
		%\begin{figure*}[htbp]
		%    \begin{center}
		\subfigure[Orthophotos]{
			\begin{minipage}[t]{0.48\linewidth}
				\centering
				\includegraphics[width=5.1cm]{images/Chapitre3/Ortho-MEC-Malt_Tapas_1981_Ortho-MEC-Malt_2015.png}
			\end{minipage}%
		}
		\subfigure[Number of recovered matches(\textit{Ortho})]{
			\begin{minipage}[t]{0.48\linewidth}
				\centering
				\includegraphics[width=3.5cm,trim=0 20 0 38,clip]{images/Chapitre3/PlotCurves_Ortho-MEC-Malt_Tapas_1981_Ortho-MEC-Malt_2015.png}
			\end{minipage}%
		}
		\subfigure[$SIFT_{Ortho}^{RANSAC Inliers}$]{
			\begin{minipage}[t]{0.48\linewidth}
				\centering
				\includegraphics[width=5.2cm]{images/Chapitre3/Homol-SIFT2Step-Rough-2DRANSAC_Ortho-MEC-Malt_Tapas_1981_Ortho-MEC-Malt_2015.png}
			\end{minipage}%
		}
		\subfigure[$SuperGlue_{Ortho}^{RANSAC Inliers}$]{
			\begin{minipage}[t]{0.48\linewidth}
				\centering
				\includegraphics[width=5.2cm]{images/Chapitre3/Homol-SubPatch_R90-2DRANSAC_Ortho-MEC-Malt_Tapas_1981_Ortho-MEC-Malt_2015.png}
			\end{minipage}%
		}
		%        \caption{Result of matching orthophotos (i.e. \textit{Ortho}) of Pezenas 1981 and 2015. (a) Orthophotos to be matched, with red rectangles indicating the common zone. (b) Numbers of total matches and RANSAC inliers of both SIFT and SuperGlue. (c) Visualization of RANSAC inliers based on SIFT. (d) Visualization of RANSAC inliers based on SuperGlue.}
		%        \label{MatchVizPezenas1981Ortho}
		%    \end{center}
		%\end{figure*} 
		%
		%\begin{figure*}[htbp]
		%    \begin{center}
		\subfigure[DSMs]{
			\begin{minipage}[t]{0.48\linewidth}
				\centering
				\includegraphics[width=5.1cm]{images/Chapitre3/MEC-Malt_Tapas_1981_MEC-Malt_2015.png}
			\end{minipage}%
		}
		\subfigure[Number of recovered matches(\textit{DSM})]{
			\begin{minipage}[t]{0.48\linewidth}
				\centering
				\includegraphics[width=3.5cm,trim=0 20 0 38,clip]{images/Chapitre3/PlotCurves_MEC-Malt_Tapas_1981_MEC-Malt_2015.png}
			\end{minipage}%
		}
		\subfigure[$SIFT_{DSM}^{RANSAC Inliers}$]{
			\begin{minipage}[t]{0.48\linewidth}
				\centering
				\includegraphics[width=5.2cm]{images/Chapitre3/Homol-SIFT2Step-Rough-2DRANSAC_MEC-Malt_Tapas_1981_MEC-Malt_2015.png}
			\end{minipage}%
		}
		\subfigure[$SuperGlue_{DSM}^{RANSAC Inliers}$]{
			\begin{minipage}[t]{0.48\linewidth}
				\centering
				\includegraphics[width=5.2cm]{images/Chapitre3/Homol-SubPatch_R90-2DRANSAC_MEC-Malt_Tapas_1981_MEC-Malt_2015.png}
			\end{minipage}%
		}
		%\caption{Result of matching DSMs (i.e. \textit{DSM}) of Pezenas 1981 and 2015. (a) DSMs to be matched, with red rectangles indicating the common zone. (b) Numbers of total matches and RANSAC inliers of both SIFT and SuperGlue. (c) Visualization of RANSAC inliers based on SIFT. (d) Visualization of RANSAC inliers based on SuperGlue.}
		%\caption{{\small Result of \textit{ImgPairs}, \textit{Ortho} and \textit{DSM} on Pezenas 1981 and 2015. (a, e, i) image pairs/orthophotos/DSMs to be matched, with red rectangles indicating the common zone. (b, f, j) Numbers of total matches and RANSAC inliers of both SIFT and SuperGlue on methods \textit{ImgPairs}, \textit{Ortho} and \textit{DSM} individually. (c, g, k) Visualization of RANSAC inliers based on $SIFT_{ImgPairs}$, $SIFT_{Ortho}$ and $SIFT_{DSM}$. (d, h, l) Visualization of RANSAC inliers based on $SuperGlue_{ImgPairs}$, $SuperGlue_{Ortho}$ and $SuperGlue_{DSM}$.}}
		\caption{{\scriptsize Result of \textit{ImgPairs} (a-d), \textit{Ortho} (e-h) and \textit{DSM} (i-l) on matching \textbf{Pezenas 1981 and 2015}. (a, e, i) Image pairs/orthophotos/DSMs to be matched, with red rectangles indicating the common zone. (b, f, j) Numbers of total matches and RANSAC inliers of both SIFT and SuperGlue on methods \textit{ImgPairs}, \textit{Ortho} and \textit{DSM} individually. (c, g, k) Visualization of RANSAC inliers based on $SIFT_{ImgPairs}$, $SIFT_{Ortho}$ and $SIFT_{DSM}$. (d, h, l) Visualization of RANSAC inliers based on $SuperGlue_{ImgPairs}$, $SuperGlue_{Ortho}$ and $SuperGlue_{DSM}$.}}        
		\label{MatchVizPezenas1981DSM}
	\end{center}
\end{figure*} 



%%%%%%%%%%%%%%%%%%%%%%%%%%%%%%%%%%%%%%Pezenas Satellite
\begin{figure*}[htbp]
	\begin{center}
		\subfigure[Orthophotos]{
			\begin{minipage}[t]{0.48\linewidth}
				\centering
				\includegraphics[width=6.8cm]{images/Chapitre3/Ortho-MEC-Malt_Tapas_1971_Ortho-MEC-Malt_Satellite.png}
			\end{minipage}%
		}
		\subfigure[Number of recovered matches(\textit{Ortho})]{
			\begin{minipage}[t]{0.48\linewidth}
				\centering
				\includegraphics[width=4.9cm,trim=0 20 0 38,clip]{images/Chapitre3/PlotCurves_Ortho-MEC-Malt_Tapas_1971_Ortho-MEC-Malt_Satellite.png}
			\end{minipage}%
		}
		\subfigure[$SIFT_{Ortho}^{RANSAC Inliers}$]{
			\begin{minipage}[t]{0.48\linewidth}
				\centering
				\includegraphics[width=6.8cm]{images/Chapitre3/Homol-SIFT2Step-Rough-2DRANSAC_Ortho-MEC-Malt_Tapas_1971_Ortho-MEC-Malt_Satellite.png}
			\end{minipage}%
		}
		\subfigure[$SuperGlue_{Ortho}^{RANSAC Inliers}$]{
			\begin{minipage}[t]{0.48\linewidth}
				\centering
				\includegraphics[width=6.8cm]{images/Chapitre3/Homol-SubPatch_R270-2DRANSAC_Ortho-MEC-Malt_Tapas_1971_Ortho-MEC-Malt_Satellite.png}
			\end{minipage}%
		}
		%        \caption{Result of matching orthophotos (i.e. \textit{Ortho}) of Pezenas 1971 and Satellite. (a) Orthophotos to be matched, with red rectangles indicating the common zone. (b) Numbers of total matches and RANSAC inliers of both SIFT and SuperGlue. (c) Visualization of RANSAC inliers based on SIFT. (d) Visualization of RANSAC inliers based on SuperGlue.}
		%        \label{MatchVizPezenas-Satellite1971Ortho}
		%    \end{center}
		%\end{figure*} 
		%
		%\begin{figure*}[htbp]
		%    \begin{center}
		\subfigure[DSMs]{
			\begin{minipage}[t]{0.48\linewidth}
				\centering
				\includegraphics[width=6.8cm]{images/Chapitre3/MEC-Malt_Tapas_1971_MEC-Malt_Satellite.png}
			\end{minipage}%
		}
		\subfigure[Number of recovered matches(\textit{DSM})]{
			\begin{minipage}[t]{0.48\linewidth}
				\centering
				\includegraphics[width=4.9cm,trim=0 20 0 38,clip]{images/Chapitre3/PlotCurves_MEC-Malt_Tapas_1971_MEC-Malt_Satellite.png}
			\end{minipage}%
		}
		\subfigure[$SIFT_{DSM}^{RANSAC Inliers}$]{
			\begin{minipage}[t]{0.48\linewidth}
				\centering
				\includegraphics[width=6.8cm]{images/Chapitre3/Homol-SIFT2Step-Rough-2DRANSAC_MEC-Malt_Tapas_1971_MEC-Malt_Satellite.png}
			\end{minipage}%
		}
		\subfigure[$SuperGlue_{DSM}^{RANSAC Inliers}$]{
			\begin{minipage}[t]{0.48\linewidth}
				\centering
				\includegraphics[width=6.8cm]{images/Chapitre3/Homol-SubPatch_R270-2DRANSAC_MEC-Malt_Tapas_1971_MEC-Malt_Satellite.png}
			\end{minipage}%
		}
		%\caption{Result of matching DSMs (i.e. \textit{DSM}) of Pezenas 1971 and Satellite. (a) DSMs to be matched, with red rectangles indicating the common zone. (b) Numbers of total matches and RANSAC inliers of both SIFT and SuperGlue. (c) Visualization of RANSAC inliers based on SIFT. (d) Visualization of RANSAC inliers based on SuperGlue.}
		%\caption{{\small Result of \textit{ImgPairs}, \textit{Ortho} and \textit{DSM} on Pezenas 1971 and Satellite. (a, e, i) image pairs/orthophotos/DSMs to be matched, with red rectangles indicating the common zone. (b, f, j) Numbers of total matches and RANSAC inliers of both SIFT and SuperGlue on methods \textit{ImgPairs}, \textit{Ortho} and \textit{DSM} individually. (c, g, k) Visualization of RANSAC inliers based on $SIFT_{ImgPairs}$, $SIFT_{Ortho}$ and $SIFT_{DSM}$. (d, h, l) Visualization of RANSAC inliers based on $SuperGlue_{ImgPairs}$, $SuperGlue_{Ortho}$ and $SuperGlue_{DSM}$.}}
		\caption{{\scriptsize Result of \textit{ImgPairs} (a-d), \textit{Ortho} (e-h) and \textit{DSM} (i-l) on matching \textbf{Pezenas 1971 and 2014 (Satellite)}. (a, e, i) Image pairs/orthophotos/DSMs to be matched, with red rectangles indicating the common zone. (b, f, j) Numbers of total matches and RANSAC inliers of both SIFT and SuperGlue on methods \textit{ImgPairs}, \textit{Ortho} and \textit{DSM} individually. (c, g, k) Visualization of RANSAC inliers based on $SIFT_{ImgPairs}$, $SIFT_{Ortho}$ and $SIFT_{DSM}$. (d, h, l) Visualization of RANSAC inliers based on $SuperGlue_{ImgPairs}$, $SuperGlue_{Ortho}$ and $SuperGlue_{DSM}$.}}        
		\label{MatchVizPezenas-Satellite1971DSM}
	\end{center}
\end{figure*} 



\begin{figure*}[htbp]
	\begin{center}
		\subfigure[Orthophotos]{
			\begin{minipage}[t]{0.48\linewidth}
				\centering
				\includegraphics[width=6.8cm]{images/Chapitre3/Ortho-MEC-Malt_Tapas_1981_Ortho-MEC-Malt_Satellite.png}
			\end{minipage}%
		}
		\subfigure[Number of recovered matches(\textit{Ortho})]{
			\begin{minipage}[t]{0.48\linewidth}
				\centering
				\includegraphics[width=4.9cm,trim=0 20 0 38,clip]{images/Chapitre3/PlotCurves_Ortho-MEC-Malt_Tapas_1981_Ortho-MEC-Malt_Satellite.png}
			\end{minipage}%
		}
		\subfigure[$SIFT_{Ortho}^{RANSAC Inliers}$]{
			\begin{minipage}[t]{0.48\linewidth}
				\centering
				\includegraphics[width=6.8cm]{images/Chapitre3/Homol-SIFT2Step-Rough-2DRANSAC_Ortho-MEC-Malt_Tapas_1981_Ortho-MEC-Malt_Satellite.png}
			\end{minipage}%
		}
		\subfigure[$SuperGlue_{Ortho}^{RANSAC Inliers}$]{
			\begin{minipage}[t]{0.48\linewidth}
				\centering
				\includegraphics[width=6.8cm]{images/Chapitre3/Homol-SubPatch_R90-2DRANSAC_Ortho-MEC-Malt_Tapas_1981_Ortho-MEC-Malt_Satellite.png}
			\end{minipage}%
		}
		%        \caption{Result of matching orthophotos (i.e. \textit{Ortho}) of Pezenas 1981 and Satellite. (a) Orthophotos to be matched, with red rectangles indicating the common zone. (b) Numbers of total matches and RANSAC inliers of both SIFT and SuperGlue. (c) Visualization of RANSAC inliers based on SIFT. (d) Visualization of RANSAC inliers based on SuperGlue.}
		%        \label{MatchVizPezenas-Satellite1981Ortho}
		%    \end{center}
		%\end{figure*} 
		%
		%\begin{figure*}[htbp]
		%    \begin{center}
		\subfigure[DSMs]{
			\begin{minipage}[t]{0.48\linewidth}
				\centering
				\includegraphics[width=6.8cm]{images/Chapitre3/MEC-Malt_Tapas_1981_MEC-Malt_Satellite.png}
			\end{minipage}%
		}
		\subfigure[Number of recovered matches(\textit{DSM})]{
			\begin{minipage}[t]{0.48\linewidth}
				\centering
				\includegraphics[width=4.9cm,trim=0 20 0 38,clip]{images/Chapitre3/PlotCurves_MEC-Malt_Tapas_1981_MEC-Malt_Satellite.png}
			\end{minipage}%
		}
		\subfigure[$SIFT_{DSM}^{RANSAC Inliers}$]{
			\begin{minipage}[t]{0.48\linewidth}
				\centering
				\includegraphics[width=6.8cm]{images/Chapitre3/Homol-SIFT2Step-Rough-2DRANSAC_MEC-Malt_Tapas_1981_MEC-Malt_Satellite.png}
			\end{minipage}%
		}
		\subfigure[$SuperGlue_{DSM}^{RANSAC Inliers}$]{
			\begin{minipage}[t]{0.48\linewidth}
				\centering
				\includegraphics[width=6.8cm]{images/Chapitre3/Homol-SubPatch_R90-2DRANSAC_MEC-Malt_Tapas_1981_MEC-Malt_Satellite.png}
			\end{minipage}%
		}
		%\caption{Result of matching DSMs (i.e. \textit{DSM}) of Pezenas 1981 and Satellite. (a) DSMs to be matched, with red rectangles indicating the common zone. (b) Numbers of total matches and RANSAC inliers of both SIFT and SuperGlue. (c) Visualization of RANSAC inliers based on SIFT. (d) Visualization of RANSAC inliers based on SuperGlue.}
		%\caption{{\small Result of \textit{ImgPairs}, \textit{Ortho} and \textit{DSM} on Pezenas 1981 and Satellite. (a, e, i) image pairs/orthophotos/DSMs to be matched, with red rectangles indicating the common zone. (b, f, j) Numbers of total matches and RANSAC inliers of both SIFT and SuperGlue on methods \textit{ImgPairs}, \textit{Ortho} and \textit{DSM} individually. (c, g, k) Visualization of RANSAC inliers based on $SIFT_{ImgPairs}$, $SIFT_{Ortho}$ and $SIFT_{DSM}$. (d, h, l) Visualization of RANSAC inliers based on $SuperGlue_{ImgPairs}$, $SuperGlue_{Ortho}$ and $SuperGlue_{DSM}$.}}
		\caption{{\scriptsize Result of \textit{ImgPairs} (a-d), \textit{Ortho} (e-h) and \textit{DSM} (i-l) on matching \textbf{Pezenas 1981 and 2014 (Satellite)}. (a, e, i) Image pairs/orthophotos/DSMs to be matched, with red rectangles indicating the common zone. (b, f, j) Numbers of total matches and RANSAC inliers of both SIFT and SuperGlue on methods \textit{ImgPairs}, \textit{Ortho} and \textit{DSM} individually. (c, g, k) Visualization of RANSAC inliers based on $SIFT_{ImgPairs}$, $SIFT_{Ortho}$ and $SIFT_{DSM}$. (d, h, l) Visualization of RANSAC inliers based on $SuperGlue_{ImgPairs}$, $SuperGlue_{Ortho}$ and $SuperGlue_{DSM}$.}}        
		\label{MatchVizPezenas-Satellite1981DSM}
	\end{center}
\end{figure*} 


%%%%%%%%%%%%%%%%%%%%%%%%%%%%%%%%%%%%%%Kobe

\begin{figure*}[htbp]
	\begin{center}
		\subfigure[Image pairs (15$\times$83 pairs)]{
			\begin{minipage}[t]{0.48\linewidth}
				\centering
				\includegraphics[width=2.3cm,angle=90]{images/Chapitre3/Pseudo-Ortho-MEC-Malt_Tapas_1991_Ortho-MEC-Malt_Tapas_1994.png}
			\end{minipage}%
		}
		\subfigure[Number of recovered matches(\textit{ImgPairs})]{
			\begin{minipage}[t]{0.48\linewidth}
				\centering
				\includegraphics[width=3.5cm,trim=0 20 0 38,clip]{images/Chapitre3/PlotCurves_Pseudo-Ortho-MEC-Malt_Tapas_1991_Ortho-MEC-Malt_Tapas_1994.png}
			\end{minipage}%
		}   
		\subfigure[$SIFT_{ImgPairs}^{RANSAC Inliers}$]{
			\begin{minipage}[t]{0.48\linewidth}
				\centering
				\includegraphics[width=2.6cm,angle=90]{images/Chapitre3/Pseudo-Homol-SIFT2Step_1991-1994-Rough-2DRANSAC-GlobalR3D-PileImg_Ortho-MEC-Malt_Tapas_1991_Ortho-MEC-Malt_Tapas_1994.png}
			\end{minipage}%
		}
		\subfigure[$SuperGlue_{ImgPairs}^{RANSAC Inliers}$]{
			\begin{minipage}[t]{0.48\linewidth}
				\centering
				\includegraphics[width=2.6cm,angle=90]{images/Chapitre3/Pseudo-Homol-SuperGlue_1991-1994-GlobalR3D-PileImg_Ortho-MEC-Malt_Tapas_1991_Ortho-MEC-Malt_Tapas_1994.png}
			\end{minipage}%
		}
		%        \caption{Result of matching image pairs (i.e. \textit{ImgPairs}) of Kobe 1991 and 1995. (a) Image pairs to be matched, with red rectangles indicating the common zone. (b) Numbers of total matches and RANSAC inliers of both SIFT and SuperGlue. (c) Visualization of RANSAC inliers based on SIFT. (d) Visualization of RANSAC inliers based on SuperGlue.}
		%        \label{MatchVizKobe1991ImgPairs}
		%    \end{center}
		%\end{figure*} 
		%
		%\begin{figure*}[htbp]
		%    \begin{center}
		\subfigure[Orthophotos]{
			\begin{minipage}[t]{0.48\linewidth}
				\centering
				\includegraphics[width=2.3cm,angle=90]{images/Chapitre3/Ortho-MEC-Malt_Tapas_1991_Ortho-MEC-Malt_Tapas_1994.png}
			\end{minipage}%
		}
		\subfigure[Number of recovered matches(\textit{Ortho})]{
			\begin{minipage}[t]{0.48\linewidth}
				\centering
				\includegraphics[width=3.5cm,trim=0 20 0 38,clip]{images/Chapitre3/PlotCurves_Ortho-MEC-Malt_Tapas_1991_Ortho-MEC-Malt_Tapas_1994.png}
			\end{minipage}%
		}   
		\subfigure[$SIFT_{Ortho}^{RANSAC Inliers}$]{
			\begin{minipage}[t]{0.48\linewidth}
				\centering
				\includegraphics[width=2.6cm,angle=90]{images/Chapitre3/Homol-SIFT2Step-Rough-2DRANSAC_Ortho-MEC-Malt_Tapas_1991_Ortho-MEC-Malt_Tapas_1994.png}
			\end{minipage}%
		}
		\subfigure[$SuperGlue_{Ortho}^{RANSAC Inliers}$]{
			\begin{minipage}[t]{0.48\linewidth}
				\centering
				\includegraphics[width=2.6cm,angle=90]{images/Chapitre3/Homol-SubPatch-2DRANSAC_Ortho-MEC-Malt_Tapas_1991_Ortho-MEC-Malt_Tapas_1994.png}
			\end{minipage}%
		}
		%        \caption{Result of matching orthophotos (i.e. \textit{Ortho}) of Kobe 1991 and 1995. (a) Orthophotos to be matched, with red rectangles indicating the common zone. (b) Numbers of total matches and RANSAC inliers of both SIFT and SuperGlue. (c) Visualization of RANSAC inliers based on SIFT. (d) Visualization of RANSAC inliers based on SuperGlue.}
		%        \label{MatchVizKobe1991Ortho}
		%    \end{center}
		%\end{figure*} 
		%
		%\begin{figure*}[htbp]
		%    \begin{center}
		\subfigure[DSMs]{
			\begin{minipage}[t]{0.48\linewidth}
				\centering
				\includegraphics[width=2.3cm,angle=90]{images/Chapitre3/MEC-Malt_Tapas_1991_MEC-Malt_Tapas_1994.png}
			\end{minipage}%
		}
		\subfigure[Number of recovered matches(\textit{DSM})]{
			\begin{minipage}[t]{0.48\linewidth}
				\centering
				\includegraphics[width=3.5cm,trim=0 20 0 38,clip]{images/Chapitre3/PlotCurves_MEC-Malt_Tapas_1991_MEC-Malt_Tapas_1994.png}
			\end{minipage}%
		}
		\subfigure[$SIFT_{DSM}^{RANSAC Inliers}$]{
			\begin{minipage}[t]{0.48\linewidth}
				\centering
				\includegraphics[width=2.6cm,angle=90]{images/Chapitre3/Homol-SIFT2Step-Rough-2DRANSAC_MEC-Malt_Tapas_1991_MEC-Malt_Tapas_1994.png}
			\end{minipage}%
		}
		\subfigure[$SuperGlue_{DSM}^{RANSAC Inliers}$]{
			\begin{minipage}[t]{0.48\linewidth}
				\centering
				\includegraphics[width=2.6cm,angle=90]{images/Chapitre3/Homol-SubPatch-2DRANSAC_MEC-Malt_Tapas_1991_MEC-Malt_Tapas_1994.png}
			\end{minipage}%
		}
		%\caption{Result of matching DSMs (i.e. \textit{DSM}) of Kobe 1991 and 1995. (a) DSMs to be matched, with red rectangles indicating the common zone. (b) Numbers of total matches and RANSAC inliers of both SIFT and SuperGlue. (c) Visualization of RANSAC inliers based on SIFT. (d) Visualization of RANSAC inliers based on SuperGlue.}
		%\caption{{\small Result of \textit{ImgPairs}, \textit{Ortho} and \textit{DSM} on Kobe 1991 and 1995. (a, e, i) image pairs/orthophotos/DSMs to be matched, with red rectangles indicating the common zone. (b, f, j) Numbers of total matches and RANSAC inliers of both SIFT and SuperGlue on methods \textit{ImgPairs}, \textit{Ortho} and \textit{DSM} individually. (c, g, k) Visualization of RANSAC inliers based on $SIFT_{ImgPairs}$, $SIFT_{Ortho}$ and $SIFT_{DSM}$. (d, h, l) Visualization of RANSAC inliers based on $SuperGlue_{ImgPairs}$, $SuperGlue_{Ortho}$ and $SuperGlue_{DSM}$.}}
		\caption{{\scriptsize Result of \textit{ImgPairs} (a-d), \textit{Ortho} (e-h) and \textit{DSM} (i-l) on matching \textbf{Kobe 1991 and 1995}. (a, e, i) Image pairs/orthophotos/DSMs to be matched, with red rectangles indicating the common zone. (b, f, j) Numbers of total matches and RANSAC inliers of both SIFT and SuperGlue on methods \textit{ImgPairs}, \textit{Ortho} and \textit{DSM} individually. (c, g, k) Visualization of RANSAC inliers based on $SIFT_{ImgPairs}$, $SIFT_{Ortho}$ and $SIFT_{DSM}$. (d, h, l) Visualization of RANSAC inliers based on $SuperGlue_{ImgPairs}$, $SuperGlue_{Ortho}$ and $SuperGlue_{DSM}$.}}        
		\label{MatchVizKobe1991DSM}
	\end{center}
\end{figure*} 

\section{\ac{DoD}}
\label{sec:matchViz}
Here we display the visualization of \ac{DoD}s for dataset Fr{\'e}jus, Pezenas and Kobe in Figure~\ref{DoDFrejus}, ~\ref{DoDPezenas} and ~\ref{DoDKobe}. The corresponding statistical information is given in Table~\ref{DoDStatistic}.\\
As can be seen, dome effect is present in all the \ac{DoD}s because of poorly estimated camera parameters.\\
\begin{figure*}[htbp]
	\begin{center}
		\subfigure[\ac{DoD}$_{Frejus1954}^{SuperGlue_{ImgPairs}}$]{
			\begin{minipage}[t]{0.31\linewidth}
				\centering
				%left, lower, right, up
				\includegraphics[width=3.3cm,trim=680 80 110 220,clip]{images/Chapitre3/DoD1954R3D-SuperGlue.png}
			\end{minipage}%
		}
		\subfigure[\ac{DoD}$_{Frejus1954}^{SuperGlue_{Ortho}}$]{
			\begin{minipage}[t]{0.31\linewidth}
				\centering
				\includegraphics[width=3.3cm,trim=680 80 110 220,clip]{images/Chapitre3/DoD1954Ortho-SuperGlue.png}
			\end{minipage}%
		}
		\subfigure[\ac{DoD}$_{Frejus1954}^{SuperGlue_{DSM}}$]{
			\begin{minipage}[t]{0.31\linewidth}
				\centering
				\includegraphics[width=3.3cm,trim=680 80 110 220,clip]{images/Chapitre3/DoD1954DSM-SuperGlue.png}
			\end{minipage}%
		}
		
		\subfigure[\ac{DoD}$_{Frejus1954}^{SIFT_{ImgPairs}}$]{
			\begin{minipage}[t]{0.31\linewidth}
				\centering
				%left, lower, right, up
				\includegraphics[width=2cm]{images/Chapitre3/NoDoD.png}
			\end{minipage}%
		}
		\subfigure[\ac{DoD}$_{Frejus1954}^{SIFT_{Ortho}}$]{
			\begin{minipage}[t]{0.31\linewidth}
				\centering
				\includegraphics[width=2cm]{images/Chapitre3/NoDoD.png}
			\end{minipage}%
		}
		\subfigure[\ac{DoD}$_{Frejus1954}^{SIFT_{DSM}}$]{
			\begin{minipage}[t]{0.31\linewidth}
				\centering
				\includegraphics[width=3.3cm,trim=680 80 110 220,clip]{images/Chapitre3/DoD1954DSM-SIFT.png}
			\end{minipage}%
		}
		%       \caption{DoDs of SuperGlue and SIFT on dataset Fr{\'e}jus (epoch 1954). The prohibition sign means the corresponding method failed.}
		%       \label{Frejus DoD of SuperGlue}
		%   \end{center}
		%\end{figure*} 
		%
		%\begin{figure*}[htbp]
		%   \begin{center}
		\subfigure[\ac{DoD}$_{Frejus1966}^{SuperGlue_{ImgPairs}}$]{
			\begin{minipage}[t]{0.31\linewidth}
				\centering
				%left, lower, right, up
				\includegraphics[width=3.3cm,trim=740 230 50 380,clip]{images/Chapitre3/DoD1966R3D-SuperGlue.png}
			\end{minipage}%
		}
		\subfigure[\ac{DoD}$_{Frejus1966}^{SuperGlue_{Ortho}}$]{
			\begin{minipage}[t]{0.31\linewidth}
				\centering
				\includegraphics[width=3.3cm,trim=740 230 50 380,clip]{images/Chapitre3/DoD1966Ortho-SuperGlue.png}
			\end{minipage}%
		}
		\subfigure[\ac{DoD}$_{Frejus1966}^{SuperGlue_{DSM}}$]{
			\begin{minipage}[t]{0.31\linewidth}
				\centering
				\includegraphics[width=3.3cm,trim=740 230 50 380,clip]{images/Chapitre3/DoD1966DSM-SuperGlue.png}
			\end{minipage}%
		}
		
		
		\subfigure[\ac{DoD}$_{Frejus1966}^{SIFT_{ImgPairs}}$]{
			\begin{minipage}[t]{0.31\linewidth}
				\centering
				%left, lower, right, up
				\includegraphics[width=2cm]{images/Chapitre3/NoDoD.png}
			\end{minipage}%
		}
		\subfigure[\ac{DoD}$_{Frejus1966}^{SIFT_{Ortho}}$]{
			\begin{minipage}[t]{0.31\linewidth}
				\centering
				\includegraphics[width=2cm]{images/Chapitre3/NoDoD.png}
			\end{minipage}%
		}
		\subfigure[\ac{DoD}$_{Frejus1966}^{SIFT_{DSM}}$]{
			\begin{minipage}[t]{0.31\linewidth}
				\centering
				\includegraphics[width=3.3cm,trim=740 230 50 380,clip]{images/Chapitre3/DoD1966DSM-SIFT.png}
			\end{minipage}%
		}       
		%       \caption{DoDs of SuperGlue and SIFT on dataset Fr{\'e}jus (epoch 1966). The prohibition sign means the corresponding method failed.}
		%\label{Frejus DoD of SuperGlue}
		%\end{center}
		%\end{figure*} 
		%
		%\begin{figure*}[htbp]
		%\begin{center}     
		\subfigure[\ac{DoD}$_{Frejus1970}^{SuperGlue_{ImgPairs}}$]{
			\begin{minipage}[t]{0.31\linewidth}
				\centering
				%left, lower, right, up
				\includegraphics[width=3.3cm,trim=680 180 50 260,clip]{images/Chapitre3/DoD1970R3D-SuperGlue.png}
			\end{minipage}%
		}
		\subfigure[\ac{DoD}$_{Frejus1970}^{SuperGlue_{Ortho}}$]{
			\begin{minipage}[t]{0.31\linewidth}
				\centering
				\includegraphics[width=3.3cm,trim=680 180 50 260,clip]{images/Chapitre3/DoD1970Ortho-SuperGlue.png}
			\end{minipage}%
		}
		\subfigure[\ac{DoD}$_{Frejus1970}^{SuperGlue_{DSM}}$]{
			\begin{minipage}[t]{0.31\linewidth}
				\centering
				\includegraphics[width=3.3cm,trim=680 180 50 260,clip]{images/Chapitre3/DoD1970DSM-SuperGlue.png}
			\end{minipage}%
		}
		
		\subfigure[\ac{DoD}$_{Frejus1970}^{SIFT_{ImgPairs}}$]{
			\begin{minipage}[t]{0.31\linewidth}
				\centering
				%left, lower, right, up
				\includegraphics[width=2cm]{images/Chapitre3/NoDoD.png}
			\end{minipage}%
		}
		\subfigure[\ac{DoD}$_{Frejus1970}^{SIFT_{Ortho}}$]{
			\begin{minipage}[t]{0.31\linewidth}
				\centering
				\includegraphics[width=2cm]{images/Chapitre3/NoDoD.png}
			\end{minipage}%
		}
		\subfigure[\ac{DoD}$_{Frejus1970}^{SIFT_{DSM}}$]{
			\begin{minipage}[t]{0.31\linewidth}
				\centering
				\includegraphics[width=3.3cm,trim=680 180 50 260,clip]{images/Chapitre3/DoD1970DSM-SIFT.png}
			\end{minipage}%
		}
		
		\subfigure[\ac{DoD} legend]{
			\begin{minipage}[t]{1\linewidth}
				\centering
				\includegraphics[width=11cm]{images/Chapitre3/LegendDoD.png}
			\end{minipage}%
		}
		%\caption{DoDs of SuperGlue and SIFT on dataset Fr{\'e}jus (epoch 1970). The prohibition sign means the corresponding method failed.}
		\caption{{\scriptsize \ac{DoD}s between free epoch \textbf{Fr{\'e}jus 1954, 1966, 1970} and reference epoch \textbf{2014} with methods $SuperGlue_{ImgPairs}$ (a, g, m), $SuperGlue_{Ortho}$ (b, h, n), $SuperGlue_{DSM}$ (c, i, o), $SIFT_{ImgPairs}$ (d, j, p), $SIFT_{Ortho}$ (e, k, q) and $SIFT_{DSM}$ (f, l, r). The prohibition sign means the corresponding method failed.}}
		\label{DoDFrejus}
	\end{center}
\end{figure*} 



\begin{figure*}[htbp]
	\begin{center}
		\subfigure[\ac{DoD}$_{Pezenas1971}^{SuperGlue_{ImgPairs}}$]{
			\begin{minipage}[t]{0.31\linewidth}
				\centering
				%left, lower, right, up
				\includegraphics[width=4.2cm,trim=680 80 50 230,clip]{images/Chapitre3/DoD1971R3D-SuperGlue.png}
			\end{minipage}%
		}
		\subfigure[\ac{DoD}$_{Pezenas1971}^{SuperGlue_{Ortho}}$]{
			\begin{minipage}[t]{0.31\linewidth}
				\centering
				\includegraphics[width=4.2cm,trim=680 80 50 230,clip]{images/Chapitre3/DoD1971Ortho-SuperGlue.png}
			\end{minipage}%
		}
		\subfigure[\ac{DoD}$_{Pezenas1971}^{SuperGlue_{DSM}}$]{
			\begin{minipage}[t]{0.31\linewidth}
				\centering
				\includegraphics[width=4.2cm,trim=680 80 50 230,clip]{images/Chapitre3/DoD1971DSM-SuperGlue.png}
			\end{minipage}%
		}
		
		\subfigure[\ac{DoD}$_{Pezenas1971}^{SIFT_{ImgPairs}}$]{
			\begin{minipage}[t]{0.31\linewidth}
				\centering
				%left, lower, right, up
				\includegraphics[width=4.2cm,trim=680 80 50 230,clip]{images/Chapitre3/DoD1971R3D-SIFT.png}
			\end{minipage}%
		}
		\subfigure[\ac{DoD}$_{Pezenas1971}^{SIFT_{Ortho}}$]{
			\begin{minipage}[t]{0.31\linewidth}
				\centering
				\includegraphics[width=4.2cm,trim=680 80 50 230,clip]{images/Chapitre3/DoD1971Ortho-SIFT.png}
			\end{minipage}%
		}
		\subfigure[\ac{DoD}$_{Pezenas1971}^{SIFT_{DSM}}$]{
			\begin{minipage}[t]{0.31\linewidth}
				\centering
				\includegraphics[width=4.2cm,trim=680 80 50 230,clip]{images/Chapitre3/DoD1971DSM-SIFT.png}
			\end{minipage}%
		}
		
		\subfigure[\ac{DoD}$_{Pezenas1981}^{SuperGlue_{ImgPairs}}$]{
			\begin{minipage}[t]{0.31\linewidth}
				\centering
				%left, lower, right, up
				\includegraphics[width=4.2cm,trim=720 100 50 200,clip]{images/Chapitre3/DoD1981R3D-SuperGlue.png}
			\end{minipage}%
		}
		\subfigure[\ac{DoD}$_{Pezenas1981}^{SuperGlue_{Ortho}}$]{
			\begin{minipage}[t]{0.31\linewidth}
				\centering
				\includegraphics[width=4.2cm,trim=720 100 50 200,clip]{images/Chapitre3/DoD1981Ortho-SuperGlue.png}
			\end{minipage}%
		}
		\subfigure[\ac{DoD}$_{Pezenas1981}^{SuperGlue_{DSM}}$]{
			\begin{minipage}[t]{0.31\linewidth}
				\centering
				\includegraphics[width=4.2cm,trim=720 100 50 200,clip]{images/Chapitre3/DoD1981DSM-SuperGlue.png}
			\end{minipage}%
		}
		
		\subfigure[\ac{DoD}$_{Pezenas1981}^{SIFT_{ImgPairs}}$]{
			\begin{minipage}[t]{0.31\linewidth}
				\centering
				%left, lower, right, up
				\includegraphics[width=4.2cm,trim=720 100 50 200,clip]{images/Chapitre3/DoD1981R3D-SIFT.png}
			\end{minipage}%
		}
		\subfigure[\ac{DoD}$_{Pezenas1981}^{SIFT_{Ortho}}$]{
			\begin{minipage}[t]{0.31\linewidth}
				\centering
				\includegraphics[width=4.2cm,trim=720 100 50 200,clip]{images/Chapitre3/DoD1981Ortho-SIFT.png}
			\end{minipage}%
		}
		\subfigure[\ac{DoD}$_{Pezenas1981}^{SIFT_{DSM}}$]{
			\begin{minipage}[t]{0.31\linewidth}
				\centering
				\includegraphics[width=4.2cm,trim=720 100 50 200,clip]{images/Chapitre3/DoD1981DSM-SIFT.png}
			\end{minipage}%
		}
		\subfigure[\ac{DoD} legend]{
			\begin{minipage}[t]{1\linewidth}
				\centering
				\includegraphics[width=11cm]{images/Chapitre3/LegendDoD.png}
			\end{minipage}%
		}
		\caption{{\scriptsize \ac{DoD}s between free epoch \textbf{Pezenas 1971, 1981} and reference aerial epoch \textbf{2015} with methods $SuperGlue_{ImgPairs}$ (a, g), $SuperGlue_{Ortho}$ (b, h), $SuperGlue_{DSM}$ (c, i), $SIFT_{ImgPairs}$ (d, j), $SIFT_{Ortho}$ (e, k) and $SIFT_{DSM}$ (f, l). The prohibition sign means the corresponding method failed.}}
		\label{DoDPezenas}
	\end{center}
\end{figure*} 


\begin{figure*}[htbp]
	\begin{center}
		\subfigure[\ac{DoD}$_{Pezenas1971}^{SuperGlue_{Ortho}}$]{
			\begin{minipage}[t]{0.31\linewidth}
				\centering
				%\includegraphics[width=4.5cm,trim=680 80 50 230,clip]{images/Chapitre3/DoD1971Ortho-SuperGlue-Satellite.png}
				\includegraphics[width=2cm]{images/Chapitre3/NoDoD.png}
			\end{minipage}%
		}
		\subfigure[\ac{DoD}$_{Pezenas1971}^{SuperGlue_{DSM}}$]{
			\begin{minipage}[t]{0.31\linewidth}
				\centering
				\includegraphics[width=4.5cm,trim=680 80 50 230,clip]{images/Chapitre3/DoD1971DSM-SuperGlue-Satellite.png}
			\end{minipage}%
		}\\
		
		\subfigure[\ac{DoD}$_{Pezenas1971}^{SIFT_{Ortho}}$]{
			\begin{minipage}[t]{0.31\linewidth}
				\centering
				\includegraphics[width=2cm]{images/Chapitre3/NoDoD.png}
			\end{minipage}%
		}
		\subfigure[\ac{DoD}$_{Pezenas1971}^{SIFT_{DSM}}$]{
			\begin{minipage}[t]{0.31\linewidth}
				\centering
				\includegraphics[width=4.2cm,trim=720 100 50 200,clip]{images/Chapitre3/DoD1971DSM-SIFT-Satellite.png}
			\end{minipage}%
		}\\
		
		\subfigure[\ac{DoD}$_{Pezenas1981}^{SuperGlue_{Ortho}}$]{
			\begin{minipage}[t]{0.31\linewidth}
				\centering
				\includegraphics[width=4.2cm,trim=900 100 250 200,clip]{images/Chapitre3/DoD1981Ortho-SuperGlue-Satellite.png}
			\end{minipage}%
		}
		\subfigure[\ac{DoD}$_{Pezenas1981}^{SuperGlue_{DSM}}$]{
			\begin{minipage}[t]{0.31\linewidth}
				\centering
				\includegraphics[width=4.2cm,trim=900 100 250 200,clip]{images/Chapitre3/DoD1981DSM-SuperGlue-Satellite.png}
			\end{minipage}%
		}\\
		
		
		\subfigure[\ac{DoD}$_{Pezenas1981}^{SIFT_{Ortho}}$]{
			\begin{minipage}[t]{0.31\linewidth}
				\centering
				\includegraphics[width=2cm]{images/Chapitre3/NoDoD.png}
			\end{minipage}%
		}
		\subfigure[\ac{DoD}$_{Pezenas1981}^{SIFT_{DSM}}$]{
			\begin{minipage}[t]{0.31\linewidth}
				\centering
				\includegraphics[width=4.2cm,trim=900 100 250 200,clip]{images/Chapitre3/DoD1981DSM-SIFT-Satellite.png}
			\end{minipage}%
		}\\
		
		
		\subfigure[\ac{DoD} legend]{
			\begin{minipage}[t]{1\linewidth}
				\centering
				\includegraphics[width=11cm]{images/Chapitre3/LegendDoD.png}
			\end{minipage}%
		}
		\caption{{\scriptsize \ac{DoD}s between free epoch \textbf{Pezenas 1971, 1981} and reference satellite epoch \textbf{2014} with methods $SuperGlue_{Ortho}$ (a, e), $SuperGlue_{DSM}$ (b, f), $SIFT_{Ortho}$ (c, g) and $SIFT_{DSM}$ (d, h). The holes among them are areas covered with clouds which are masked out. The prohibition sign means the corresponding method failed.}}
		\label{DoDPezenas-Satellite}
	\end{center}
\end{figure*} 


\begin{figure*}[htbp]
	\begin{center}
		\subfigure[\ac{DoD}$_{Kobe}^{SuperGlue_{ImgPairs}}$]{
			\begin{minipage}[t]{1\linewidth}
				\centering
				%left, lower, right, up
				\includegraphics[width=12cm,trim=700 450 180 560,clip]{images/Chapitre3/DoD1991R3D-SuperGlue.png}
			\end{minipage}%
		}
		\subfigure[\ac{DoD}$_{Kobe}^{SuperGlue_{Ortho}}$]{
			\begin{minipage}[t]{1\linewidth}
				\centering
				\includegraphics[width=12cm,trim=700 450 180 560,clip]{images/Chapitre3/DoD1991Ortho-SuperGlue.png}
			\end{minipage}%
		}
		\subfigure[\ac{DoD}$_{Kobe}^{SuperGlue_{DSM}}$]{
			\begin{minipage}[t]{1\linewidth}
				\centering
				\includegraphics[width=12cm,trim=700 450 180 560,clip]{images/Chapitre3/DoD1991DSM-SuperGlue.png}
			\end{minipage}%
		}
		
		\subfigure[\ac{DoD}$_{Kobe}^{SIFT_{ImgPairs}}$]{
			\begin{minipage}[t]{1\linewidth}
				\centering
				%left, lower, right, up
				\includegraphics[width=2cm]{images/Chapitre3/NoDoD.png}
			\end{minipage}%
		}
		\subfigure[\ac{DoD}$_{Kobe}^{SIFT_{Ortho}}$]{
			\begin{minipage}[t]{1\linewidth}
				\centering
				\includegraphics[width=2cm]{images/Chapitre3/NoDoD.png}
			\end{minipage}%
		}
		\subfigure[\ac{DoD}$_{Kobe}^{SIFT_{DSM}}$]{
			\begin{minipage}[t]{1\linewidth}
				\centering
				\includegraphics[width=12cm,trim=700 450 180 560,clip]{images/Chapitre3/DoD1991DSM-SIFT.png}
			\end{minipage}%
		}
		
		\subfigure[\ac{DoD} legend]{
			\begin{minipage}[t]{1\linewidth}
				\centering
				\includegraphics[width=11cm]{images/Chapitre3/LegendDoD.png}
			\end{minipage}%
		}
		\caption{{\scriptsize \ac{DoD}s between free epoch \textbf{Kobe 1991} and reference epoch \textbf{1995} with methods $SuperGlue_{ImgPairs}$ (a), $SuperGlue_{Ortho}$ (b), $SuperGlue_{DSM}$ (c), $SIFT_{ImgPairs}$ (d), $SIFT_{Ortho}$ (e) and $SIFT_{DSM}$ (f). The prohibition sign means the corresponding method failed.}}
		\label{DoDKobe}
	\end{center}
\end{figure*} 

\begin{table}%[H]
	\footnotesize
	\centering
	\begin{tabular}{||l|l|c|c|c||}\hline
		& &$\mu$ [m]&$\sigma$ [m]&$|\mu|$ [m]\\\hline\hline
		\multirow{6}{*}{$DoD^{Frejus}_{1954-2014}$}
		&${SuperGlue_{ImgPairs}}$ & 5.70 & 6.32 & 6.62\\
		&${SuperGlue_{Ortho}}$ & 2.19 & 6.46 & 4.55\\
		&${SuperGlue_{DSM}}$ & 2.07 & 4.87 & \textbf{3.83} \\
		&${SIFT_{ImgPairs}}$ & / & / & / \\
		&${SIFT_{Ortho}}$ & / & / & / \\
		&${SIFT_{DSM}}$ & 6.92 & 8.47 & 8.04\\\hline
		
		\multirow{6}{*}{$DoD^{Frejus}_{1966-2014}$}
		&${SuperGlue_{ImgPairs}}$ & -1.36 & 3.82 & 2.90\\
		&${SuperGlue_{Ortho}}$ & -0.37 & 4.22 & 3.01\\
		&${SuperGlue_{DSM}}$ & -0.46 & 3.77 & \textbf{2.68}\\
		&${SIFT_{ImgPairs}}$ & / & / & / \\
		&${SIFT_{Ortho}}$ & / & / & / \\
		&${SIFT_{DSM}}$ & -1.72 & 4.92 & 3.75\\\hline
		
		\multirow{6}{*}{$DoD^{Frejus}_{1970-2014}$}
		&${SuperGlue_{ImgPairs}}$ & -5.04 & 5.09 & 5.70\\
		&${SuperGlue_{Ortho}}$ & -2.63 & 5.18 & \textbf{4.39}\\
		&${SuperGlue_{DSM}}$ & -1.71 & 5.75 & 4.61\\
		&${SIFT_{ImgPairs}}$ & / & / & / \\
		&${SIFT_{Ortho}}$ & / & / & / \\
		&${SIFT_{DSM}}$ & -3.17 & 5.44 & 4.71\\\hline
		
		
		\multirow{6}{*}{$DoD^{Pezenas}_{1971-2015}$}
		&${SuperGlue_{ImgPairs}}$ & -7.42 & 16.60 & 13.78\\
		&${SuperGlue_{Ortho}}$ & -8.46 & 22.36 & 16.68\\
		&${SuperGlue_{DSM}}$ & 4.71 & 16.85 & 14.20\\
		&${SIFT_{ImgPairs}}$ & 2.77 & 16.58 & 13.98\\
		&${SIFT_{Ortho}}$ & -8.05 & 21.71 & 17.07\\
		&${SIFT_{DSM}}$ & -0.75 & 17.75 & \textbf{13.74}\\\hline
		
		\multirow{6}{*}{$DoD^{Pezenas}_{1981-2015}$}
		&${SuperGlue_{ImgPairs}}$ & -0.39 & 9.12 & \textbf{7.10}\\
		&${SuperGlue_{Ortho}}$ & -2.60 & 9.93 & 7.76\\
		&${SuperGlue_{DSM}}$ & 0.64 & 9.15 & 7.24\\
		&${SIFT_{ImgPairs}}$ & -2.01 & 10.07 & 7.80\\
		&${SIFT_{Ortho}}$ & -4.80 & 13.25 & 10.33\\
		&${SIFT_{DSM}}$ & -0.73 & 9.67 & 7.42\\\hline
		
		%%%%%%%%%%%%%%%%%%%%%%%%satellite
		\multirow{4}{*}{$DoD^{Pezenas}_{1971-2014(Satellite)}$}
		&${SuperGlue_{Ortho}}$ & / & / & /\\
		&${SuperGlue_{DSM}}$ & -3.70 & 10.65 & 8.29\\
		%&${SuperGlue_{DSM}}$ & -1.85 & 13.24 & 9.15\\
		&${SIFT_{Ortho}}$ & / & / & /\\
		&${SIFT_{DSM}}$ & -0.68 & 8.11 & \textbf{5.80} \\\hline
		%&${SIFT_{DSM}}$ & 0.58 & 7.82 & \textbf{3.18}\\\hline
		
		\multirow{4}{*}{$DoD^{Pezenas}_{1981-2014(Satellite)}$}
		&${SuperGlue_{Ortho}}$ & -0.64 & 6.18 & 4.48\\
		%&${SuperGlue_{Ortho}}$ & 0.59 & 9.87 & 5.66\\
		&${SuperGlue_{DSM}}$ & -1.15 & 6.07 & 4.55\\
		%&${SuperGlue_{DSM}}$ & 0.31 & 9.88 & 5.73\\
		&${SIFT_{Ortho}}$ & / & / & /\\
		&${SIFT_{DSM}}$ & -1.41 & 6.14 & \textbf{4.45} \\\hline
		%&${SIFT_{DSM}}$ & 0.46 & 5.96 & \textbf{1.92}\\\hline
		
		
		\multirow{6}{*}{$DoD^{Kobe}_{1991-1995}$}
		&${SuperGlue_{ImgPairs}}$ & -1.63 & 13.85 & \textbf{7.24}\\
		%&${SIFT_{ImgPairs}}$ & 167.46 & 121.26 & 168.47\\
		&${SuperGlue_{Ortho}}$ & -0.54 & 14.83 & 7.78\\
		%&${SIFT_{Ortho}}$ & 4.29 & 41.89 & 26.01\\
		&${SuperGlue_{DSM}}$ & -0.75 & 14.62 & 7.95\\
		&${SIFT_{ImgPairs}}$ & / & / & / \\
		&${SIFT_{Ortho}}$ & / & / & / \\
		&${SIFT_{DSM}}$ & 0.27 & 14.40 & 7.57\\\hline
		
	\end{tabular}
	\caption{Average value $\mu$, standard deviation $\sigma$, and absolute average value $|\mu|$ of all the \ac{DoD}s in Figure~\ref{DoDFrejus}, ~\ref{DoDPezenas}, ~\ref{DoDPezenas-Satellite} and ~\ref{DoDKobe}.}
	\label{DoDStatistic}
\end{table}

\cleardoublepage
%!TEX root = Manuscript.tex

\chapter{Result of precise matching}
\label{chap:appendixB}
In this section we demonstrate the matches visualizations and \ac{DoD}s of precise matching results on datasets Pezenas and Kobe.\\

\section{Matches visualization}
\label{sec:PrecisematchViz}
For both datasets, there exist 2 epochs, leading to 1 set of epoch combination for precise matching. Four methods (i.e. \ding{172} $Patch_{SpGDSM}$, \ding{173} $Guided_{SpGDSM}$, \ding{174} $Patch_{SIFTDSM}$ and \ding{175} $Guided_{SIFTDSM}$) are tested on both datasets, the resulted matches are visualized in Figure \ref{MatchVizPezenas} and \ref{MatchVizKobe}. 

%For each dataset, every possible combination of 2 epochs are matched with 4 methods (\ding{172} $Patch_{SpGDSM}$, \ding{173} $Guided_{SpGDSM}$, \ding{174} $Patch_{SIFTDSM}$ and \ding{175} $Guided_{SIFTDSM}$). 

%\begin{enumerate}
%	\item For Pezenas, there exist 2 epochs, leading to 1 set of epoch combination, the matches visualization of which is displayed in Figure \ref{MatchVizPezenas}.
%	\item For Kobe, there exist 2 epochs, leading to 1 set of epoch combination, the matches visualization of which is displayed in Figure \ref{MatchVizKobe}.
%\end{enumerate}

Patterns similar to Fr{\'e}jus and Alberona (elaborated in Section \ref{matchVizMainBody}) are present in Pezenas and Kobe:\\
\begin{enumerate}
	\item $Patch$ recovered considerably more matches than $Guided$, which is understandable as SuperGlue is more invariant over time than SIFT.\\
	\item 3D-RANSAC filter and cross correlation removed considerable number of matches, yet enough matches survived.
\end{enumerate}


%%%%%%%%%%%%%%%%%Pezenas
\begin{figure*}[htbp]
	\begin{center}
		\subfigure[Common zone]{
			\begin{minipage}[t]{0.48\linewidth}
				\centering
				\includegraphics[width=6.8cm]{images/Chapitre4/Pseudo-Ortho-MEC-Malt_Tapas_1971_Ortho-MEC-Malt_Satellite.png}
			\end{minipage}%
		}
		\subfigure[Number of recovered matches]{
			\begin{minipage}[t]{0.48\linewidth}
				\centering
				\includegraphics[width=5.8cm]{images/Chapitre4/PlotBarH-Pezenas1971-2014.png}
			\end{minipage}%
		}
		\subfigure[$Patch_{SpGDSM}$]{
			\begin{minipage}[t]{0.48\linewidth}
				\centering
				\includegraphics[width=6.8cm]{images/Chapitre4/Precise-SpGDSMHomol-SuperGlue-3DRANSAC-CrossCorrelation-PileImg_Ortho-MEC-Malt_Tapas_1971_Ortho-MEC-Malt_Satellite.png}
			\end{minipage}%
		}
		\subfigure[$Guided_{SpGDSM}$]{
			\begin{minipage}[t]{0.48\linewidth}
				\centering
				\includegraphics[width=6.8cm]{images/Chapitre4/Precise-SpGDSMHomol-GuidedSIFT-3DRANSAC-CrossCorrelation-PileImg_Ortho-MEC-Malt_Tapas_1971_Ortho-MEC-Malt_Satellite.png}
			\end{minipage}%
		}
		\subfigure[$Patch_{SIFTDSM}$]{
			\begin{minipage}[t]{0.48\linewidth}
				\centering
				\includegraphics[width=6.8cm]{images/Chapitre4/Precise-SIFTDSMHomol-SuperGlue-3DRANSAC-CrossCorrelation-PileImg_Ortho-MEC-Malt_Tapas_1971_Ortho-MEC-Malt_Satellite.png}
			\end{minipage}%
		}
		\subfigure[$Guided_{SIFTDSM}$]{
			\begin{minipage}[t]{0.48\linewidth}
				\centering
				\includegraphics[width=6.8cm]{images/Chapitre4/Precise-SIFTDSMHomol-GuidedSIFT-3DRANSAC-CrossCorrelation-PileImg_Ortho-MEC-Malt_Tapas_1971_Ortho-MEC-Malt_Satellite.png}
			\end{minipage}%
		}
		\caption{Precise matching visualization of \textbf{Pezenas 1971 and 2014 (Satellite)}. (a) Image pairs to be matched, with red rectangles indicating the common zone. (b) Numbers of tentative, enhanced and final matches recovered with $Patch_{SpGDSM}$, $Guided_{SpGDSM}$, $Patch_{SIFTDSM}$ and $Guided_{SIFTDSM}$ individually. (c-f) Visualization of final matches recovered with $Patch_{SpGDSM}$, $Guided_{SpGDSM}$, $Patch_{SIFTDSM}$ and $Guided_{SIFTDSM}$ individually.}
		\label{MatchVizPezenas}
	\end{center}
\end{figure*} 

%Pseudo-Ortho-MEC-Malt_Tapas_1991_Ortho-MEC-Malt_Tapas_1994

\begin{figure*}[htbp]
	\begin{center}
		\subfigure[Common zone]{
			\begin{minipage}[t]{0.48\linewidth}
				\centering
				\includegraphics[width=3.8cm,angle=90]{images/Chapitre3/Pseudo-Ortho-MEC-Malt_Tapas_1991_Ortho-MEC-Malt_Tapas_1994.png}
			\end{minipage}%
		}
		\subfigure[Number of recovered matches]{
			\begin{minipage}[t]{0.48\linewidth}
				\centering
				\includegraphics[width=5.8cm]{images/Chapitre4/PlotBarH-Kobe1991-1995.png}
			\end{minipage}%
		}
		\subfigure[$Patch_{SpGDSM}$]{
			\begin{minipage}[t]{0.48\linewidth}
				\centering
				\includegraphics[width=3.8cm,angle=90]{images/Chapitre4/Precise-SpGDSMHomol-SuperGlue-3DRANSAC-CrossCorrelation-PileImg_Ortho-MEC-Malt_Tapas_1991_Ortho-MEC-Malt_Tapas_1994.png}
			\end{minipage}%
		}
		\subfigure[$Guided_{SpGDSM}$]{
			\begin{minipage}[t]{0.48\linewidth}
				\centering
				\includegraphics[width=3.8cm,angle=90]{images/Chapitre4/Precise-SpGDSMHomol-GuidedSIFT-3DRANSAC-CrossCorrelation-PileImg_Ortho-MEC-Malt_Tapas_1991_Ortho-MEC-Malt_Tapas_1994.png}
			\end{minipage}%
		}
		\subfigure[$Patch_{SIFTDSM}$]{
			\begin{minipage}[t]{0.48\linewidth}
				\centering
				\includegraphics[width=3.8cm,angle=90]{images/Chapitre4/Precise-SIFTDSMHomol-SuperGlue-3DRANSAC-CrossCorrelation-PileImg_Ortho-MEC-Malt_Tapas_1991_Ortho-MEC-Malt_Tapas_1994.png}
			\end{minipage}%
		}
		\subfigure[$Guided_{SIFTDSM}$]{
			\begin{minipage}[t]{0.48\linewidth}
				\centering
				\includegraphics[width=3.8cm,angle=90]{images/Chapitre4/Precise-SIFTDSMHomol-GuidedSIFT-3DRANSAC-CrossCorrelation-PileImg_Ortho-MEC-Malt_Tapas_1991_Ortho-MEC-Malt_Tapas_1994.png}
			\end{minipage}%
		}
		\caption{Precise matching visualization of \textbf{Kobe 1991 and 1995}. (a) Image pairs to be matched, with red rectangles indicating the common zone. (b) Numbers of tentative, enhanced and final matches recovered with $Patch_{SpGDSM}$, $Guided_{SpGDSM}$, $Patch_{SIFTDSM}$ and $Guided_{SIFTDSM}$ individually. (c-f) Visualization of final matches recovered with $Patch_{SpGDSM}$, $Guided_{SpGDSM}$, $Patch_{SIFTDSM}$ and $Guided_{SIFTDSM}$ individually.}
		\label{MatchVizKobe}
	\end{center}
\end{figure*} 

\section{DoD}
\label{sec:PreciseDoD}
The \ac{DoD}s for Pezenas and Kobe are demonstrated in Figure~\ref{PreciseDoDPezenas-Satellite} and ~\ref{PreciseDoDKobe}. 
%In each figure, the \ac{DoD}s resulted from rough co-registered orientations using methods $SuperGlue_{DSM}$ and $SIFT_{DSM}$ (elaborated in Chapter~\ref{chap:RoughCoReg}, hereinafter referred to as DoD$^{SpGDSM}$ and DoD$^{SIFTDSM}$) are displayed as bases, and the \ac{DoD}s resulted from refined orientations using methods $Patch_{SpGDSM}$, $Guided_{SpGDSM}$, $Patch_{SIFTDSM}$ and $Guided_{SIFTDSM}$ (hereinafter termed as DoD$^{Patch_{SpGDSM}}$, DoD$^{Guided_{SpGDSM}}$, DoD$^Patch_{SIFTDSM}$ and DoD$^Guided_{SIFTDSM}$) are given for comparison. 
The corresponding statistical information is displayed in Table~\ref{PreciseDoDStatistic1}.\\

For both datasets, the dome effect presented in DoD$^{SpGDSM}$ and DoD$^{SIFTDSM}$ (the first column of the subgraphs) is mitigated in DoD$^{Patch_{SpGDSM}}$, DoD$^{Guided_{SpGDSM}}$, DoD$^Patch_{SIFTDSM}$ and DoD$^Guided_{SIFTDSM}$ (the second and third column of subgraphs). 
According to the absolute average value $|\mu|$ of all the \ac{DoD}s in Figure \ref{PreciseDoDPezenas-Satellite} and \ref{PreciseDoDKobe} (cf. Talbe \ref{PreciseDoDStatistic1}), $Guided$ leads to \ac{DoD}s with slightly better accuracy than $Patch$.\\

\begin{figure*}[htbp]
	\begin{center}
		\subfigure[\ac{DoD}$_{Pezenas1971}^{{SpGDSM}}$]{
			\begin{minipage}[t]{0.31\linewidth}
				\centering
				\includegraphics[width=4.5cm,trim=740 80 50 230,clip]{images/Chapitre3/DoD1971DSM-SuperGlue-Satellite.png}
			\end{minipage}%
		}
		\subfigure[\ac{DoD}$_{Pezenas1971}^{Patch_{SpGDSM}}$]{
			\begin{minipage}[t]{0.31\linewidth}
				\centering
				\includegraphics[width=4.5cm,trim=840 70 290 340,clip]{images/Chapitre4/DoD1971_Patch_SpGDSM.png}
			\end{minipage}%
		}
		\subfigure[\ac{DoD}$_{Pezenas1971}^{Guided_{SpGDSM}}$]{
			\begin{minipage}[t]{0.31\linewidth}
				\centering
				\includegraphics[width=4.5cm,trim=840 70 290 340,clip]{images/Chapitre4/DoD1971_Guided_SpGDSM.png}
			\end{minipage}%
		}\\
		\subfigure[\ac{DoD}$_{Pezenas1971}^{{SIFTDSM}}$]{
			\begin{minipage}[t]{0.31\linewidth}
				\centering
				\includegraphics[width=4.5cm,trim=740 100 50 200,clip]{images/Chapitre3/DoD1971DSM-SIFT-Satellite.png}
			\end{minipage}%
		}
		\subfigure[\ac{DoD}$_{Pezenas1971}^{Patch_{SIFTDSM}}$]{
			\begin{minipage}[t]{0.31\linewidth}
				\centering
				\includegraphics[width=4.5cm,trim=840 70 290 340,clip]{images/Chapitre4/DoD1971_Patch_SIFTDSM.png}
			\end{minipage}%
		}
		\subfigure[\ac{DoD}$_{Pezenas1971}^{Guided_{SIFTDSM}}$]{
			\begin{minipage}[t]{0.31\linewidth}
				\centering
				\includegraphics[width=4.5cm,trim=840 70 290 340,clip]{images/Chapitre4/DoD1971_Guided_SIFTDSM.png}
			\end{minipage}%
		}\\
		
		\subfigure[\ac{DoD} legend]{
			\begin{minipage}[t]{1\linewidth}
				\centering
				\includegraphics[width=11cm]{images/Chapitre4/LegendDoD.png}
			\end{minipage}%
		}
		\caption{{\scriptsize \ac{DoD}s between free epoch \textbf{Pezenas 1971} and reference satellite epoch \textbf{2014}. (a) and (d) are \ac{DoD}s resulted from roughly co-registered orientations using methods $SuperGlue_{DSM}$ and $SIFT_{DSM}$ (elaborated in Chapter 3). (b, c, e, f) are \ac{DoD}s resulted from refined orientations using methods $Patch_{SpGDSM}$, $Guided_{SpGDSM}$, $Patch_{SIFTDSM}$ and $Guided_{SIFTDSM}$ individually. The holes among them are areas covered with clouds which are masked out.}}
		\label{PreciseDoDPezenas-Satellite}
	\end{center}
\end{figure*} 


\begin{figure*}[htbp]
	\begin{center}
		\subfigure[\ac{DoD}$_{Kobe}^{{SpGDSM}}$]{
			\begin{minipage}[t]{1\linewidth}
				\centering
				\includegraphics[width=12cm,trim=700 450 180 560,clip]{images/Chapitre3/DoD1991DSM-SuperGlue.png}
			\end{minipage}%
		}
		\subfigure[\ac{DoD}$_{Kobe}^{Patch_{SpGDSM}}$]{
			\begin{minipage}[t]{1\linewidth}
				\centering
				\includegraphics[width=12cm,trim=680 330 100 570,clip]{images/Chapitre4/DoD1991_Patch_SpGDSM.png}
			\end{minipage}%
		}
		\subfigure[\ac{DoD}$_{Kobe}^{Guided_{SpGDSM}}$]{
			\begin{minipage}[t]{1\linewidth}
				\centering
				\includegraphics[width=12cm,trim=680 330 100 570,clip]{images/Chapitre4/DoD1991_Guided_SpGDSM.png}
			\end{minipage}%
		}\\
		\subfigure[\ac{DoD}$_{Kobe}^{{SIFTDSM}}$]{
			\begin{minipage}[t]{1\linewidth}
				\centering
				\includegraphics[width=12cm,trim=700 450 180 560,clip]{images/Chapitre3/DoD1991DSM-SIFT.png}
			\end{minipage}%
		}
		\subfigure[\ac{DoD}$_{Kobe}^{Patch_{SIFTDSM}}$]{
			\begin{minipage}[t]{1\linewidth}
				\centering
				\includegraphics[width=12cm,trim=680 330 100 570,clip]{images/Chapitre4/DoD1991_Patch_SIFTDSM.png}
			\end{minipage}%
		}
		\subfigure[\ac{DoD}$_{Kobe}^{Guided_{SIFTDSM}}$]{
			\begin{minipage}[t]{1\linewidth}
				\centering
				\includegraphics[width=12cm,trim=680 330 100 570,clip]{images/Chapitre4/DoD1991_Guided_SIFTDSM.png}
			\end{minipage}%
		}
		
		\subfigure[\ac{DoD} legend]{
			\begin{minipage}[t]{1\linewidth}
				\centering
				\includegraphics[width=11cm]{images/Chapitre4/LegendDoD.png}
			\end{minipage}%
		}
		\caption{{\scriptsize \ac{DoD}s between free epoch \textbf{Kobe 1991} and reference epoch \textbf{1995}. (a) and (d) are \ac{DoD}s resulted from roughly co-registered orientations using methods $SuperGlue_{DSM}$ and $SIFT_{DSM}$ (elaborated in Chapter~\ref{chap:RoughCoReg}). (b, c, e, f) are \ac{DoD}s resulted from refined orientations using methods $Patch_{SpGDSM}$, $Guided_{SpGDSM}$, $Patch_{SIFTDSM}$ and $Guided_{SIFTDSM}$ individually.}}
		\label{PreciseDoDKobe}
	\end{center}
\end{figure*} 



\begin{table}%[H]
	\footnotesize
	\centering
	\begin{tabular}{||l|l|c|c|c||}\hline
		& &$\mu$ [m]&$\sigma$ [m]&$|\mu|$ [m]\\\hline\hline
				%%%%%%%%%%%%%%%%%%%%%%%%satellite
		\multirow{4}{*}{$DoD^{Pezenas}_{1971-2014(Satellite)}$}
		&${{SpGDSM}}$ & -3.70 & 10.65 & 8.29\\
		&${Patch_{SpGDSM}}$ & -0.34 & 4.39 & 2.28\\
		&${Guided_{SpGDSM}}$ & -0.65 & 4.46 & 2.45\\
		&${{SIFTDSM}}$ & -0.68 & 8.11 & 5.80\\
		&${Patch_{SIFTDSM}}$ & -0.49 & 4.41 & 2.29\\
		&${Guided_{SIFTDSM}}$ & -0.57 & 4.38 & \textbf{2.27}\\\hline
		
		\multirow{6}{*}{$DoD^{Kobe}_{1991-1995}$}
		&${{SpGDSM}}$ & -0.75 & 14.62 & 7.95\\
		&${Patch_{SpGDSM}}$ & 1.93 & 10.26 & 3.99\\
		&${Guided_{SpGDSM}}$ & 2.03 & 11.74 & 4.30\\
		&${{SIFTDSM}}$ & 0.27 & 14.40 & 7.57\\
		&${Patch_{SIFTDSM}}$ & 1.80 & 10.36 & 4.00\\
		&${Guided_{SIFTDSM}}$ & 1.84 & 9.48 & \textbf{3.87}\\\hline
				
	\end{tabular}
	\caption{Average value $\mu$, standard deviation $\sigma$, and absolute average value $|\mu|$ of all the \ac{DoD}s in Figure~\ref{PreciseDoDPezenas-Satellite} and ~\ref{PreciseDoDKobe}.}
	\label{PreciseDoDStatistic1}
\end{table}



\cleardoublepage
%!TEX root = Manuscript.tex

\chapter{Tutorial of our pipeline}
\label{chap:appendixC}

We provide two thorough tutorials \cite{tuto-aerial}, \cite{tuto-mixed} with test datasets to familiarize users with our pipelines. 
%In this tutorial we will introduce you to tie-points extraction in diachronic images (only aerial images) in MicMac. (For processing both satellite and aerial images, please refer to Historical_Satellite_TiePtExtraction_pipeline.ipynb) 
The goal of the tutorials is to recover matches for multi-epoch images. %We refer to images obtained within one epoch as intra-epoch images, and images obtained at different epochs as inter-epoch images.
The tutorial performs an intra-epoch processing, followed by an inter-epoch processing. The latter consists of 2 main steps: rough co-registration and precise matching. At the end, an evaluation part is presented to generate and display the resulted \ac{DoD}s.
The structure of the tutorial is as follows:
\begin{itemize}
	\item[-] Intra-epoch processing:
	\begin{enumerate}
		\item \textbf{Feature matching}. Apply feature matching based on SIFT on images within the same epoch.
		\item \textbf{Relative orientation}. Compute relative orientations for each epoch.
		\item \textbf{DSM generation}. Compute \ac{DSM} of each epoch based on relative orientations.
	\end{enumerate}
	\item[-] Inter-epoch processing:
	\begin{enumerate}
		\item \textbf{Automated pipeline}. The automated pipeline will launch the whole inter-epoch processing pipeline by calling several subcommands.
		\item \textbf{Deep-dive in submodules}. We also provide deep-dive to explain all the submodules used in the automated pipeline. It consists of: (1) rough co-registration, which roughly co-register the \ac{DSM}s and image orientations from different epochs; (2) precise matching, which obtains precise matches under the guidance of rough co-registration.
	\end{enumerate}	
	\item[-] Evaluation:
	\begin{enumerate}
	\item \textbf{Roughly co-registered DoD}.
	\item \textbf{Refined DoD based on SuperGlue}.
	\item \textbf{Refined DoD based on SIFT}.
\end{enumerate}	
\end{itemize}

Take one tutorial (i.e., \cite{tuto-aerial}) as example, in the following we display the commands used in the tutorial. The dataset used in the tutorial consists of 2 epochs (i.e., 1971 and 1981).

\section{Intra-epoch processing}
In this section, both epochs 1971 and 1981 go through the same commands individually. For the sake of simplicity, we take only epoch 1981 as an example to demonstrate the commands.\\
\subsection{Feature matching}

\begin{enumerate}
\item{Recover tie-points with command \textit{Tapioca}:}

\begin{verbatim}
mm3d Tapioca MulScale OIS-Reech_IGNF_PVA_1-0__1981.*tif 500 -1 PostFix=_1981
\end{verbatim}

\item{Remove tie-points on the fiducial marks with command \textit{HomolFilterMasq}:}
\begin{verbatim}
mm3d HomolFilterMasq OIS-Reech_IGNF_PVA_1-0__1981.*tif
 GlobalMasq=Fiducial_marks_masq-1981-3.tif PostIn=_1981 PostOut=_1981-Masq
\end{verbatim}

\item{Tie-points reduction with command \textit{Ratafia}:}
\begin{verbatim}
mm3d TestLib NO_AllOri2Im OIS-Reech_IGNF_PVA_1-0__1981.*tif SH=_1981-Masq

mm3d Ratafia OIS-Reech_IGNF_PVA_1-0__1981.*tif SH=_1981-Masq Out=_1981-Ratafia
\end{verbatim}
\end{enumerate}

\subsection{Relative orientation}
Recover relative orientation with command \textit{Tapas}:\\
\begin{verbatim}
mm3d Tapas FraserBasic OIS-Reech_IGNF_PVA_1-0__1981.*tif Out=1981 SH=_1981-Masq
\end{verbatim}

\subsection{DSM generation}
Calculate \ac{DSM} with command \textit{Malt}:\\
\begin{verbatim}
mm3d Malt Ortho OIS-Reech_IGNF_PVA_1-0__1981.*tif 1981 NbVI=2 
MasqImGlob=Fiducial_marks_masq-1981-3.tif DirMEC=MEC-Malt_1981 EZA=1 ZoomF=2 
DoOrtho=0
\end{verbatim}

\section{Inter-epoch processing}
\subsection{Automated pipeline with command \textit{TiePHistoP}}

\begin{enumerate}

\item{Option 1: SuperGlue:}
\begin{verbatim}
mm3d TiePHistoP Ori-1971 Ori-1981 ImgList1971all.txt ImgList1981all.txt 
 MEC-Malt_1971 MEC-Malt_1981 CoRegPatchLSz=[1280,960]
 CoRegPatchRSz=[1280,960] PrecisePatchSz=[1280,960] Feature=SuperGlue
\end{verbatim}

\item{Option 2: SIFT:}
\begin{verbatim}
mm3d TiePHistoP Ori-1971 Ori-1981 ImgList1971all.txt ImgList1981all.txt 
 MEC-Malt_1971 MEC-Malt_1981 PrecisePatchSz=[1280,960] Feature=SIFT 
 SkipCoReg=1 CoRegOri1=1971_CoReg_SuperGlue
\end{verbatim}
\end{enumerate}

\subsection{Deep-dive in the pipeline's submodules}
\begin{enumerate}

\item{Rough co-registration}
 
(1) \ac{DSM} Equalization for each epoch with command \textit{TestLib DSM\_Equalization}:\\
\begin{verbatim}
mm3d TestLib DSM_Equalization MEC-Malt_1981 DSMFile=MMLastNuage.xml
 OutImg=DSM1981-gray.tif
 
mm3d TestLib DSM_Equalization MEC-Malt_1971 DSMFile=MMLastNuage.xml
 OutImg=DSM1971-gray.tif
\end{verbatim}

(2) \ac{DSM} Wallis filter for each epoch with command \textit{TestLib Wallis}:\\
\begin{verbatim}
mm3d TestLib Wallis DSM1981-gray.tif Dir=MEC-Malt_1981
 OutImg=DSM1981-gray.tif_sfs.tif
 
mm3d TestLib Wallis DSM1971-gray.tif Dir=MEC-Malt_1971
 OutImg=DSM1971-gray.tif_sfs.tif
\end{verbatim}
(3) Matching \ac{DSM} based on SuperGlue with 4 rotation hypotheses.\\

 (3.1) Rotate the secondary \ac{DSM} four times and split \ac{DSM} pairs into patch pairs with command \textit{TestLib GetPatchPair}:\\
\begin{verbatim}
mm3d TestLib GetPatchPair BruteForce MEC-Malt_1971/DSM1971-gray.tif_sfs.tif 
 MEC-Malt_1981/DSM1981-gray.tif_sfs.tif  OutDir=./Tmp_Patches-CoReg
 Rotate=1 PatchLSz=[1280,960] PatchRSz=[1280,960]
\end{verbatim}
 (3.2) Hypothesis 0 $^\circ$:\\
\begin{verbatim}
mm3d TestLib SuperGlue SuperGlueInput.txt  InDir=./Tmp_Patches-CoReg/ 
 OutDir=./Tmp_Patches-CoReg/ SpGOutSH=-SuperGlue

mm3d TestLib MergeTiePt ./Tmp_Patches-CoReg/  HomoXml=SubPatch.xml 
 MergeInSH=-SuperGlue MergeOutSH=-SubPatch PatchSz=[1280,960]

mm3d TestLib RANSAC R2D MEC-Malt_1971.tif MEC-Malt_1981.tif
 Dir=./Tmp_Patches-CoReg/ 2DRANInSH=-SubPatch
 2DRANOutSH=-SubPatch-2DRANSAC
\end{verbatim}
 (3.3) Hypothesis 90 $^\circ$:\\
\begin{verbatim}
mm3d TestLib SuperGlue SuperGlueInput_R90.txt  InDir=./Tmp_Patches-CoReg/ 
 OutDir=./Tmp_Patches-CoReg/ SpGOutSH=-SuperGlue

mm3d TestLib MergeTiePt ./Tmp_Patches-CoReg/  HomoXml=SubPatch_R90.xml 
MergeInSH=-SuperGlue MergeOutSH=-SubPatch_R90
 PatchSz=[1280,960]

mm3d TestLib RANSAC R2D MEC-Malt_1971.tif MEC-Malt_1981.tif
 Dir=./Tmp_Patches-CoReg/ 2DRANInSH=-SubPatch_R90 
 2DRANOutSH=-SubPatch_R90-2DRANSAC
\end{verbatim}
 (3.4) Hypothesis 180 $^\circ$:\\
\begin{verbatim}
mm3d TestLib SuperGlue SuperGlueInput_R180.txt  InDir=./Tmp_Patches-CoReg/ 
 OutDir=./Tmp_Patches-CoReg/ SpGOutSH=-SuperGlue

mm3d TestLib MergeTiePt ./Tmp_Patches-CoReg/  HomoXml=SubPatch_R180.xml 
 MergeInSH=-SuperGlue MergeOutSH=-SubPatch_R180 PatchSz=[1280,960]

mm3d TestLib RANSAC R2D MEC-Malt_1971.tif MEC-Malt_1981.tif
 Dir=./Tmp_Patches-CoReg/ 2DRANInSH=-SubPatch_R180 
 2DRANOutSH=-SubPatch_R180-2DRANSAC
\end{verbatim}
 (3.5) Hypothesis 270 $^\circ$:\\
\begin{verbatim}
mm3d TestLib SuperGlue SuperGlueInput_R270.txt  InDir=./Tmp_Patches-CoReg/ 
 OutDir=./Tmp_Patches-CoReg/ SpGOutSH=-SuperGlue

mm3d TestLib MergeTiePt ./Tmp_Patches-CoReg/  HomoXml=SubPatch_R270.xml 
 MergeInSH=-SuperGlue MergeOutSH=-SubPatch_R270 PatchSz=[1280,960]

mm3d TestLib RANSAC R2D MEC-Malt_1971.tif MEC-Malt_1981.tif
 Dir=./Tmp_Patches-CoReg/ 2DRANInSH=-SubPatch_R270 
 2DRANOutSH=-SubPatch_R270-2DRANSAC
\end{verbatim}
(4) Create GCPs with command \textit{TestLib CreateGCPs}:\\
\begin{verbatim}
mm3d TestLib CreateGCPs ./Tmp_Patches-CoReg MEC-Malt_1971.tif
 MEC-Malt_1981.tif ./ ImgList1971all.txt ImgList1981all.txt
 Ori-1971 Ori-1981 MEC-Malt_1971 MEC-Malt_1981 
 CreateGCPsInSH=-SubPatch_R180-2DRANSAC Out2DXml1=OutGCP2D_epoch1971.xml
 Out3DXml1=OutGCP3D_epoch1971.xml Out2DXml2=OutGCP2D_epoch1981.xml 
 Out3DXml2=OutGCP3D_epoch1981.xml
\end{verbatim}
(5) 3D Helmert transformation with command \textit{GCPBascule}:\\
\begin{verbatim}
mm3d GCPBascule "OIS-Reech_IGNF_PVA_1-0__1971.*tif" 1971 1981
 OutGCP3D_epoch1981.xml OutGCP2D_epoch1971.xml
\end{verbatim}
\item{Precise matching}

(1) Get overlapped images with command \textit{TestLib GetOverlappedImages}:\\
\begin{verbatim}
mm3d TestLib GetOverlappedImages 1971 1981 ImgList1971all.txt
 ImgList1981all.txt Para3DH=Basc-1971-2-1981.xml
\end{verbatim}
(2) Get Patch Pair with command \textit{TestLib GetPatchPair Guided}:\\
\begin{verbatim}
mm3d TestLib GetPatchPair Guided
 OIS-Reech_IGNF_PVA_1-0__1971-06-21__C2844-0141_1971_FR2117_0974.tif 
 OIS-Reech_IGNF_PVA_1-0__1981-06-16__C2544-0021_1981_F2544-2644_0064.tif 
 Ori-1971 Ori-1981 OutDir=./Tmp_Patches-Precise 
 SubPXml=OIS-Reech_IGNF_PVA_1-0__1971-06-21__C2844-0141_1971_FR2117_0974_
OIS-Reech_IGNF_PVA_1-0__1981-06-16__C2544-0021_1981_F2544-2644_0064_SubPatch.xml 
 ImgPair=OIS-Reech_IGNF_PVA_1-0__1971-06-21__C2844-0141_1971_FR2117_0974_
OIS-Reech_IGNF_PVA_1-0__1981-06-16__C2544-0021_1981_F2544-2644_0064
_SuperGlueInput.txt
 PatchSz=[1280,960] Para3DH=Basc-1971-2-1981.xml DSMDirL=MEC-Malt_1971
\end{verbatim}

(3) Get tentative tie-points (option1: SuperGlue) with command \textit{TestLib SuperGlue} and \textit{TestLib MergeTiePt}:\\
\begin{verbatim}
mm3d TestLib SuperGlue
 OIS-Reech_IGNF_PVA_1-0__1971-06-21__C2844-0141_1971_FR2117_0974_
OIS-Reech_IGNF_PVA_1-0__1981-06-16__C2544-0021_1981_F2544-2644_0064
_SuperGlueInput.txt  
 InDir=./Tmp_Patches-Precise/ OutDir=./Tmp_Patches-Precise/ 
 SpGOutSH=-SuperGlue CheckNb=100

mm3d TestLib MergeTiePt ./Tmp_Patches-Precise/ 
 HomoXml=OIS-Reech_IGNF_PVA_1-0__1971-06-21__C2844-0141_1971_FR2117_0974_
OIS-Reech_IGNF_PVA_1-0__1981-06-16__C2544-0021_1981_F2544-2644_0064_SubPatch.xml 
 MergeInSH=-SuperGlue MergeOutSH=-SuperGlue  OutDir=./ PatchSz=[1280,960] 
 BufferSz=[128,96]
\end{verbatim}

(4) Get tentative tie-points (option1: SIFT) with command \textit{TestLib GuidedSIFTMatch}:\\
\begin{verbatim}
mm3d TestLib GuidedSIFTMatch
 OIS-Reech_IGNF_PVA_1-0__1971-06-21__C2844-0141_1971_FR2117_0974.tif     
 OIS-Reech_IGNF_PVA_1-0__1981-06-16__C2544-0021_1981_F2544-2644_0064.tif 
 Ori-1971 Ori-1981  SkipSIFT=false DSMDirL=MEC-Malt_1971 DSMDirR=MEC-Malt_1981 
 Para3DH=Basc-1971-2-1981.xml
\end{verbatim}
(5) 3D-RANSAC with command \textit{TestLib RANSAC R3D}:\\
\begin{verbatim}
mm3d TestLib RANSAC R3D
 OIS-Reech_IGNF_PVA_1-0__1971-06-21__C2844-0141_1971_FR2117_0974.tif  
 OIS-Reech_IGNF_PVA_1-0__1981-06-16__C2544-0021_1981_F2544-2644_0064.tif 
 Ori-1971 Ori-1981 Dir=./  DSMDirL=MEC-Malt_1971 DSMDirR=MEC-Malt_1981 
 DSMFileL=MMLastNuage.xml DSMFileR=MMLastNuage.xml 3DRANInSH=-SuperGlue 
 3DRANOutSH=-SuperGlue-3DRANSAC
\end{verbatim}
(6) Cross correlation with command \textit{TestLib CrossCorrelation}:\\
\begin{verbatim}
mm3d TestLib CrossCorrelation
 OIS-Reech_IGNF_PVA_1-0__1971-06-21__C2844-0141_1971_FR2117_0974.tif 
 OIS-Reech_IGNF_PVA_1-0__1981-06-16__C2544-0021_1981_F2544-2644_0064.tif
 CCInSH=-SuperGlue-3DRANSAC CCOutSH=-SuperGlue-3DRANSAC-CrossCorrelation
 SzW=32 CCTh=0.6 PatchSz=[1280,960] BufferSz=[30,60] 
 PatchDir=./Tmp_Patches-Precise
 SubPXml=OIS-Reech_IGNF_PVA_1-0__1971-06-21__C2844-0141_1971_FR2117_0974_
OIS-Reech_IGNF_PVA_1-0__1981-06-16__C2544-0021_1981_F2544-2644_0064_SubPatch.xml
\end{verbatim}
\end{enumerate}

\section{Evaluation}
\subsection{Roughly co-registered DoD}
\begin{enumerate}

\item{Get \ac{DSM} of epoch 1971:}
\begin{verbatim}
mm3d Malt Ortho OIS-Reech_IGNF_PVA_1-0__1971.*tif 1981 NbVI=2 
 DirMEC=MEC-Malt_1971_CoReg EZA=1 MasqImGlob=Fiducial_marks_masq-1971-3.tif 
 ZoomF=4 DoOrtho=0
\end{verbatim}
\item{Calculate DoD with command \textit{CmpIm}:}
\begin{verbatim}
mm3d CmpIm MEC-Malt_1971_CoReg/Z_Num7_DeZoom4_STD-MALT.tif 
 MEC-Malt_1981/Z_Num8_DeZoom2_STD-MALT.tif UseFOM=1 FileDiff=DoD-CoReg.tif 
 16Bit=1
\end{verbatim}
\end{enumerate}

\subsection{Refined DoD based on SuperGlue}
\begin{enumerate}

\item{Set weight of inter-epoch tie-points with command \textit{TestLib TiePtAddWeight }:}
\begin{verbatim}
mm3d TestLib TiePtAddWeight 10 InSH=-SuperGlue-3DRANSAC-CrossCorrelation
\end{verbatim}
\item{Txt to binary conversion with command \textit{HomolFilterMasq}:}
\begin{verbatim}
mm3d HomolFilterMasq "O.*tif" PostIn=-SuperGlue-3DRANSAC-CrossCorrelation-W10 
 PostOut=-SuperGlue-3DRANSAC-CrossCorrelation-W10-dat ANM=1 ExpTxt=1 
 ExpTxtOut=0
\end{verbatim}
\item{Merge intra- and inter-epoch tie-points with command \textit{MergeHomol}:}
\begin{verbatim}
mm3d MergeHomol "Homol_1971-Ratafia|Homol_1981-Ratafia
|Homol-SuperGlue-3DRANSAC-CrossCorrelation-W10-dat"
 Homol_Merged-SuperGlue
\end{verbatim}
\item{Run bundle adjustment with command \textit{Campari}:}
\begin{verbatim}
mm3d Campari "O.*tif" 1981 Campari_Refined-SuperGlue SH=_Merged-SuperGlue 
AllFree=1 NbIterEnd=20 SigmaTieP=0.25
\end{verbatim}
\item{Get \ac{DSM} of epoch 1981:}
\begin{verbatim}
mm3d Malt Ortho OIS-Reech_IGNF_PVA_1-0__1981.*tif Campari_Refined-SuperGlue
 NbVI=2 DirMEC=MEC-Malt_1981_Refined-SuperGlue EZA=1 
 MasqImGlob=Fiducial_marks_masq-1981-3.tif ZoomF=2 DoOrtho=0
\end{verbatim}
\item{Get \ac{DSM} of epoch 1971:}
\begin{verbatim}
mm3d Malt Ortho OIS-Reech_IGNF_PVA_1-0__1971.*tif Campari_Refined-SuperGlue
 NbVI=2 DirMEC=MEC-Malt_1971_Refined-SuperGlue EZA=1 
 MasqImGlob=Fiducial_marks_masq-1971-3.tif ZoomF=4 DoOrtho=0
\end{verbatim}
\item{Calculate DoD:}
\begin{verbatim}
mm3d CmpIm MEC-Malt_1971_Refined-SuperGlue/Z_Num7_DeZoom4_STD-MALT.tif 
 MEC-Malt_1981_Refined-SuperGlue/Z_Num8_DeZoom2_STD-MALT.tif UseFOM=1 
 FileDiff=DoD-Refined-SuperGlue.tif 16Bit=1
\end{verbatim}
\end{enumerate}

\subsection{Refined DoD based on SIFT}
\begin{enumerate}

\item{Set weight of inter-epoch tie-points:}
\begin{verbatim}
mm3d TestLib TiePtAddWeight 10 InSH=-GuidedSIFT-3DRANSAC-CrossCorrelation
\end{verbatim}
\item{Txt to binary conversion:}
\begin{verbatim}
mm3d HomolFilterMasq "O.*tif" PostIn=-GuidedSIFT-3DRANSAC-CrossCorrelation-W10 
 PostOut=-GuidedSIFT-3DRANSAC-CrossCorrelation-W10-dat ANM=1 ExpTxt=1 
 ExpTxtOut=0
\end{verbatim}
\item{Merge intra- and inter-epoch tie-points:}
\begin{verbatim}
mm3d MergeHomol "Homol_1971-Ratafia|Homol_1981-Ratafia
|Homol-GuidedSIFT-3DRANSAC-CrossCorrelation-W10-dat"
 Homol_Merged-GuidedSIFT
\end{verbatim}
\item{Run bundle adjustment:}
\begin{verbatim}
mm3d Campari "O.*tif" 1981 Campari_Refined-GuidedSIFT SH=_Merged-GuidedSIFT
 AllFree=1 NbIterEnd=20 SigmaTieP=0.25
\end{verbatim}
\item{Get \ac{DSM} of epoch 1981:}
\begin{verbatim}
mm3d Malt Ortho OIS-Reech_IGNF_PVA_1-0__1981.*tif Campari_Refined-GuidedSIFT 
 NbVI=2 DirMEC=MEC-Malt_1981_Refined-GuidedSIFT EZA=1
 MasqImGlob=Fiducial_marks_masq-1981-3.tif ZoomF=2 DoOrtho=0
\end{verbatim}
\item{Get \ac{DSM} of epoch 1971:}
\begin{verbatim}
mm3d Malt Ortho OIS-Reech_IGNF_PVA_1-0__1971.*tif Campari_Refined-GuidedSIFT 
 NbVI=2 DirMEC=MEC-Malt_1971_Refined-GuidedSIFT
 MasqImGlob=Fiducial_marks_masq-1971-3.tif EZA=1 ZoomF=4 DoOrtho=0
\end{verbatim}
\item{Calculate DoD:}
\begin{verbatim}
mm3d CmpIm MEC-Malt_1971_Refined-GuidedSIFT/Z_Num7_DeZoom4_STD-MALT.tif 
 MEC-Malt_1981_Refined-GuidedSIFT/Z_Num8_DeZoom2_STD-MALT.tif UseFOM=1 
 FileDiff=DoD-Refined-GuidedSIFT.tif 16Bit=1
\end{verbatim}
\end{enumerate}


\cleardoublepage
%%!TEX root = Manuscript.tex

\chapter{Use case of matching guided by 2D similarity transformation}
\label{chap:appendix3}
In this section we compare the performance of \textit{state-of-the-art} matching methods: (1) SIFT and (2) SuperGlue, as well as our matching strategy: (3) SIFT under the guidance of 2D similarity transformation model followed by RANSAC to remove outliers. 
\par
For each keypoint in the master image, our strategy uses 2D similarity transformation model to predict a location in the secondary image and search only its neighborhood (30 pixel in our experiment) to reduce ambiguity.\\
\section{Dataset}
This dataset consists of several archival aerial images acquired in the year 1960, provided by the National Survey of Iceland. The block located in Hofsjökull in central Iceland. The images are over a snow-covered area with low contrast. In Figure~\ref{SnowData} we displayed 6 consecutive images in the same flight strip, with snow-covered area gradually expanding. We chose the most challenging image pair (i.e. image 5 and 6, as they are fully snow-covered with very limited context) for testing. Their common zone is labeled with red rectangles. The size of both images is 14014$\times$14009 pixels.\\
For generating the 2D similarity transformation parameters that are required for our matching strategy, we manually measure only 1 match to estimate the translation, as they are images taken at the same flight stripe, the scale and rotation are approximately equal to 1 and 0, respectively.\\
%Figure~\ref{Matchresult} demonstrated the matching result of SIFT, SuperGlue and our matching strategy.\\
\begin{figure*}[htbp]
	\begin{center}
		\subfigure[Image 1]{
			\begin{minipage}[t]{0.31\linewidth}
				\centering
				\includegraphics[width=4.3cm]{images/appendix3/3-6366_crp_8Bits_Zoom8.png}
			\end{minipage}%
		}
		\subfigure[Image 2]{
	\begin{minipage}[t]{0.31\linewidth}
		\centering
		\includegraphics[width=4.3cm]{images/appendix3/3-6367_crp_8Bits_Zoom8.png}
	\end{minipage}%
}
		\subfigure[Image 3]{
	\begin{minipage}[t]{0.31\linewidth}
		\centering
		\includegraphics[width=4.3cm]{images/appendix3/3-6368_crp_8Bits_Zoom8.png}
	\end{minipage}%
}
		\subfigure[Image 4]{
	\begin{minipage}[t]{0.31\linewidth}
		\centering
		\includegraphics[width=4.3cm]{images/appendix3/3-6369_crp_8Bits_Zoom8.png}
	\end{minipage}%
}
\subfigure[Image 5]{
	\begin{minipage}[t]{0.31\linewidth}
		\centering
		\includegraphics[width=4.3cm]{images/appendix3/3-6370_crp_8Bits_Zoom8.png}
	\end{minipage}%
}
\subfigure[Image 6]{
	\begin{minipage}[t]{0.31\linewidth}
		\centering
		\includegraphics[width=4.3cm]{images/appendix3/3-6371_crp_8Bits_Zoom8.png}
	\end{minipage}%
}
		\caption{Images demonstration of the dataset. (a-f) are consecutive images in the same flight line with a wide range of area covered with snow. (e) and (f) are chosen as the testing image pair, whose common zone is pointed out as red rectangles.}
		\label{SnowData}
	\end{center}
\end{figure*} 

\section{Result}
Figure~\ref{Matchresult} demonstrated the matching result of SIFT, SuperGlue and our matching strategy. As can be seen, SIFT and SuperGlue failed to found any correct matches, while our strategy obtained a large number of good matches with negligible manual labor.
\begin{figure*}[htbp]
	\begin{center}
		\subfigure[SIFT]{
			\begin{minipage}[t]{0.48\linewidth}
				\centering
				\includegraphics[width=6.3cm]{images/appendix3/Homol-TXT_3-6370_crp_8Bits_3-6371_crp_8Bits.png}
			\end{minipage}%
		}
		\subfigure[SuperGlue]{
			\begin{minipage}[t]{0.48\linewidth}
				\centering
				\includegraphics[width=6.3cm]{images/appendix3/Homol-SuperGlue_3-6370_crp_8Bits_3-6371_crp_8Bits.png}
			\end{minipage}%
		}
		\subfigure[Guided SIFT]{
			\begin{minipage}[t]{1\linewidth}
				\centering
				\includegraphics[width=12cm]{images/appendix3/Homol-SIFT2Step-2DRANSAC_3-6370_crp_8Bits_3-6371_crp_8Bits.png}
			\end{minipage}%
		}
		\caption{Matching result of image pair from Figure~\ref{SnowData} (e) and (f). (a) and (b) are matches recovered by SIFT and SuperGlue individually. (c) displayed the matches found by our matching strategy, which is achieved by narrowing down the search space under the guidance of 2D similarity transformation model.}
		\label{Matchresult}
	\end{center}
\end{figure*} 

%%!TEX root = Manuscript.tex

\chapter{Comparison between $SIFT_{ours}$ and $SIFT_{orig}$}
\label{chap:appendix1}

\section{Comparison on method $SIFT_{ImgPairs}$}

\begin{figure*}[htbp]
	\scriptsize 
	\begin{center}
		\subfigure[Image pair]{
			\begin{minipage}[t]{0.48\linewidth}
				\centering
				\includegraphics[width=7.5cm]{images/appendix/OIS-Reech_IGNF_PVA_1-0__1971-06-21__C2844-0141_1971_FR2117_1124_15FD3425x00034_02911.png}
			\end{minipage}%
		}
		\subfigure[Match number (\textit{ImgPairs})]{
			\begin{minipage}[t]{0.48\linewidth}
				\centering
				\includegraphics[width=4.8cm]{images/appendix/PlotCurves-SIFTComp_OIS-Reech_IGNF_PVA_1-0__1971-06-21__C2844-0141_1971_FR2117_1124_15FD3425x00034_02911.png}
			\end{minipage}%
		}
		\subfigure[$SIFT_{ours}^{RANSAC Inliers}$]{
			\begin{minipage}[t]{0.48\linewidth}
				\centering
				\includegraphics[width=6cm]{images/appendix/Homol-SIFT2Step_Test-Rough-2DRANSAC_OIS-Reech_IGNF_PVA_1-0__1971-06-21__C2844-0141_1971_FR2117_1124_15FD3425x00034_02911.png}
			\end{minipage}%
		}
		\subfigure[$SIFT_{orig}^{RANSAC Inliers}$]{
			\begin{minipage}[t]{0.48\linewidth}
				\centering
				\includegraphics[width=6cm]{images/appendix/Homol-txt-2DRANSAC_OIS-Reech_IGNF_PVA_1-0__1971-06-21__C2844-0141_1971_FR2117_1124_15FD3425x00034_02911.png}
			\end{minipage}%
		}
		\caption{{\scriptsize Comparison between $SIFT_{ours}$ and $SIFT_{orig}$ on a pair of images from Pezenas 1971 and Pezenas 2015 individually. (a) Image pair to be matched, with red rectangles (\textcolor{red}{red rectangled to be drawn}) indicating the common zone. (b) Numbers of total matches, GT inliers and RANSAC inliers of $SIFT_{ours}$ and $SIFT_{origin}$. (c) Visualization of RANSAC inliers based on $SIFT_{ours}$. (d)Visualization of RANSAC inliers based on $SIFT_{orig}$.}}
		\label{Match result}
	\end{center}
\end{figure*} 

\section{Comparison on method $SIFT_{Ortho}$}
\begin{figure*}[htbp]
	\begin{center}
		\subfigure[Orthophotos]{
			\begin{minipage}[t]{0.48\linewidth}
				\centering
				\includegraphics[width=7.5cm]{images/appendix/Ortho-MEC-Malt_Tapas_1981_Ortho-MEC-Malt_2015.png}
			\end{minipage}%
		}
		\subfigure[Match number (\textit{Ortho})]{
			\begin{minipage}[t]{0.48\linewidth}
				\centering
				\includegraphics[width=4.8cm]{images/appendix/PlotCurves-SIFTComp_Ortho-MEC-Malt_Tapas_1981_Ortho-MEC-Malt_2015.png}
			\end{minipage}%
		}
		\subfigure[$SIFT_{ours}^{RANSAC Inliers}$]{
			\begin{minipage}[t]{0.48\linewidth}
				\centering
				\includegraphics[width=6cm]{images/appendix/Homol-SIFT2Step-Rough-2DRANSAC_Ortho-MEC-Malt_Tapas_1981_Ortho-MEC-Malt_2015.png}
			\end{minipage}%
		}
		\subfigure[$SIFT_{orig}^{RANSAC Inliers}$]{
			\begin{minipage}[t]{0.48\linewidth}
				\centering
				\includegraphics[width=6cm]{images/appendix/Homol-SIFT-2DRANSAC_Ortho-MEC-Malt_Tapas_1981_Ortho-MEC-Malt_2015.png}
			\end{minipage}%
		}
		\caption{Comparison between $SIFT_{ours}$ and $SIFT_{orig}$ on a pair of orthophotos from Pezenas 1981 and Pezenas 2015 individually. (a) Image pair to be matched, with red rectangles indicating the common zone. (b) Numbers of total matches, GT inliers and RANSAC inliers of $SIFT_{ours}$ and $SIFT_{origin}$. (c) Visualization of RANSAC inliers based on $SIFT_{ours}$. (d)Visualization of RANSAC inliers based on $SIFT_{orig}$.}
		\label{Match result}
	\end{center}
\end{figure*} 

\section{Comparison on method $SIFT_{DSM}$}
\begin{figure*}[htbp]
	\begin{center}
		\subfigure[DSMs]{
			\begin{minipage}[t]{0.65\linewidth}
				\centering
				\includegraphics[width=8.8cm]{images/appendix/MEC-Malt_Tapas_1954_MEC-Malt_2014.png}
			\end{minipage}%
		}
		\subfigure[Match number (\textit{DSM})]{
			\begin{minipage}[t]{0.3\linewidth}
				\centering
				\includegraphics[width=4.8cm]{images/appendix/PlotCurves-SIFTComp_MEC-Malt_Tapas_1954_MEC-Malt_2014.png}
			\end{minipage}%
		}
		\subfigure[$SIFT_{ours}^{RANSAC Inliers}$]{
			\begin{minipage}[t]{0.48\linewidth}
				\centering
				\includegraphics[width=6.8cm]{images/appendix/Homol-SIFT2Step-Rough-2DRANSAC_MEC-Malt_Tapas_1954_MEC-Malt_2014.png}
			\end{minipage}%
		}
		\subfigure[$SIFT_{orig}^{RANSAC Inliers}$]{
			\begin{minipage}[t]{0.48\linewidth}
				\centering
				\includegraphics[width=6.8cm]{images/appendix/Homol-SIFT-2DRANSAC_MEC-Malt_Tapas_1954_MEC-Malt_2014.png}
			\end{minipage}%
		}
		\caption{Comparison between $SIFT_{ours}$ and $SIFT_{orig}$ on a pair of DSMs from Fr{\'e}jus 1954 and Fr{\'e}jus 2014 individually. (a) Image pair to be matched, with red rectangles indicating the common zone. (b) Numbers of total matches, GT inliers and RANSAC inliers of $SIFT_{ours}$ and $SIFT_{origin}$. (c) Visualization of RANSAC inliers based on $SIFT_{ours}$. (d)Visualization of RANSAC inliers based on $SIFT_{orig}$.}
		\label{Match result}
	\end{center}
\end{figure*} 


%And I cite myself to show by bibtex style file (two authors)~\cite{Commowick_MICCAI_2007}.
%This for other bibtex stye file : only one author~\cite{Oakes_RStat_1999} and many authors~\cite{Guimond_CVIU_2000}.
%%!TEX root = Manuscript.tex

\chapter{Comparison between $SuperGlue_{ours}$ and $SuperGlue_{orig}$}
\label{chap:appendix2}

\section{Comparison on method $SuperGlue_{Ortho}$}
\begin{figure*}[htbp]
	\begin{center}
		\subfigure[Orthophotos]{
			\begin{minipage}[t]{0.48\linewidth}
				\centering
				\includegraphics[width=7.5cm]{images/Chapitre3/Ortho-MEC-Malt_Tapas_1970_Ortho-MEC-Malt_2014.png}
			\end{minipage}%
		}
		\subfigure[Match number (\textit{Ortho})]{
			\begin{minipage}[t]{0.48\linewidth}
				\centering
				\includegraphics[width=4.8cm]{images/appendix2/PlotCurves_TileTest-Ortho-MEC-Malt_Tapas_1970_Ortho-MEC-Malt_2014.png}
			\end{minipage}%
		}
		\subfigure[$SuperGlue_{ours}^{RANSAC Inliers}$]{
			\begin{minipage}[t]{0.48\linewidth}
				\centering
				\includegraphics[width=6cm]{images/Chapitre3/Homol-SubPatch_R270-2DRANSAC_Ortho-MEC-Malt_Tapas_1970_Ortho-MEC-Malt_2014.png}
			\end{minipage}%
		}
		\subfigure[$SuperGlue_{orig}^{TotalMatches}$]{
			\begin{minipage}[t]{0.48\linewidth}
				\centering
				\includegraphics[width=6cm]{images/appendix2/Homol-SuperGlue_Ortho-MEC-Malt_Tapas_1970_Ortho-MEC-Malt_2014.png}
			\end{minipage}%
		}
		\caption{Comparison between $SuperGlue_{ours}$ and $SuperGlue_{orig}$ on orthophotos from Fr{\'e}jus 1970 and 2014 individually. (a) Orthophotos to be matched, with red rectangles indicating the common zone. (b) Numbers of total matches, GT inliers and RANSAC inliers of $SuperGlue_{ours}$ and $SuperGlue_{orig}$. (c) Visualization of RANSAC inliers based on $SuperGlue_{ours}$. (d)Visualization of total matches based on $SuperGlue_{orig}$.}
		\label{Match result}
	\end{center}
\end{figure*} 

\section{Comparison on method $SIFT_{DSM}$}

\begin{figure*}[htbp]
	\begin{center}
		\subfigure[DSMs]{
			\begin{minipage}[t]{0.48\linewidth}
				\centering
				\includegraphics[width=7.5cm]{images/Chapitre3/MEC-Malt_Tapas_1970_MEC-Malt_2014.png}
			\end{minipage}%
		}
		\subfigure[Match number (\textit{DSM})]{
			\begin{minipage}[t]{0.48\linewidth}
				\centering
				\includegraphics[width=4.8cm]{images/appendix2/PlotCurves_TileTest-MEC-Malt_Tapas_1970_MEC-Malt_2014.png}
			\end{minipage}%
		}
		\subfigure[$SuperGlue_{ours}^{RANSAC Inliers}$]{
			\begin{minipage}[t]{0.48\linewidth}
				\centering
				\includegraphics[width=6cm]{images/Chapitre3/Homol-SubPatch_R270-2DRANSAC_MEC-Malt_Tapas_1970_MEC-Malt_2014.png}
			\end{minipage}%
		}
		\subfigure[$SuperGlue_{orig}^{TotalMatches}$]{
			\begin{minipage}[t]{0.48\linewidth}
				\centering
				\includegraphics[width=6cm]{images/appendix2/Homol-SuperGlue_MEC-Malt_Tapas_1970_MEC-Malt_2014.png}
			\end{minipage}%
		}
		\caption{Comparison between $SuperGlue_{ours}$ and $SuperGlue_{orig}$ on DSMs from Fr{\'e}jus 1970 and 2014 individually. (a) DSMs to be matched, with red rectangles indicating the common zone. (b) Numbers of total matches, GT inliers and RANSAC inliers of $SuperGlue_{ours}$ and $SuperGlue_{orig}$. (c) Visualization of RANSAC inliers based on $SuperGlue_{ours}$. (d)Visualization of total matches based on $SuperGlue_{orig}$.}
		\label{Match result}
	\end{center}
\end{figure*} 


%And I cite myself to show by bibtex style file (two authors)~\cite{Commowick_MICCAI_2007}.
%This for other bibtex stye file : only one author~\cite{Oakes_RStat_1999} and many authors~\cite{Guimond_CVIU_2000}.
%%!TEX root = Manuscript.tex

\chapter{Comparison of precise matching on DSMs and original RGB images}
\label{chap:appendix4}
In order to decide which type of images (DSMs or original images) is more suitable for executing the precise matching, we apply our pipeline \textit{Patch} on both DSMs and original images of Fr{\'e}jus 1970 and 2014.
The final matches are displayed in Figure~\ref{precisematchingdepth} (a) and (b). 
To asses quantitatively the results, we created a GT depth map and calculated the accuracy (correct matches / total matches). In Figure~\ref{precisematchingdepth}~(c) we plot the accuracy curves while varying the reprojection error threshold from 0 to 10 pixels. 
{It is clear that} the result using the original images is more accurate, even though the DSMs recovered more correspondences.
This is because historical DSMs at full resolution are too noisy to guarantee high precision measurements (see the DSM shaded image in Figure~\ref{precisematchingdepth} (d).
\begin{figure*}[htbp]
	\begin{center}
		\subfigure[Matches on RGB images]{
			\begin{minipage}[t]{0.48\linewidth}
				\centering
				\includegraphics[width=6cm]{images/appendix4/TiePtOriImg.png}
				%\caption{DoD$_{Pezenas\_1971}^{Co-Reg}$}
			\end{minipage}%
		}
		\subfigure[Matches on DSMs]{
			\begin{minipage}[t]{0.48\linewidth}
				\centering
				\includegraphics[width=6cm]{images/appendix4/TiePtDepth.png}
				%\caption{DoD$_{Pezenas\_1971}^{Guided}$}
			\end{minipage}  
		}       
		\subfigure[Accuracy of (a) and (b)]{
			\begin{minipage}[t]{0.58\linewidth}
				\centering
				\includegraphics[width=7cm]{images/appendix4/pdfdepth.png}
				%\caption{DoD$_{Pezenas\_1981}^{Co-Reg}$}
			\end{minipage}%
		}
		\subfigure[Shaded image of historical DSM]{
			\begin{minipage}[t]{0.38\linewidth}
				\centering
				\includegraphics[width=5cm]{images/appendix4/DepthShade.png}
				%\caption{DoD$_{Pezenas\_1981}^{Co-Reg}$}
			\end{minipage}%
		}
		\caption{Comparison of precise matching on original RGB images and DSMs.}
		\label{precisematchingdepth}
	\end{center}
\end{figure*} 


%Other appendices:\\
%tutorials?\\

\bibliographystyle{ThesisStyleWithEtAl}
\bibliography{Thesis}

% And here comes an example abstract (often required for PhDs, example extracted from my PhD just to show formatting).

\cleardoublepage
\begin{vcenterpage}
%\noindent\rule[2pt]{\textwidth}{0.5pt}
%\begin{center}
%{\large\textbf{Design and Use of Numerical Anatomical Atlases for Radiotherapy\\}}
%\end{center}
%{\large\textbf{Abstract:}}
%The main objective of this thesis is to provide radio-oncology specialists with automatic tools for delineating organs at risk of a patient undergoing a radiotherapy treatment of cerebral or head and neck tumors.
%\\
%To achieve this goal, we use an anatomical atlas, i.e., a representative anatomy associated to a clinical image representing it. The registration of this atlas allows to segment automatically the patient structures and to accelerate this process. Contributions in this method are presented on three axes.
%\\
%First, we want to obtain a registration method which is as independent as possible w.r.t. the setting of its parameters. This setting, done by the clinician, indeed needs to be minimal while guaranteeing a robust result. We therefore propose registration methods allowing to better control the obtained transformation, using outlier rejection techniques or locally affine transformations.
%\\
%The second axis is dedicated to the consideration of structures associated with the presence of the tumor. These structures, not present in the atlas, indeed lead to local errors in the atlas-based segmentation. We therefore propose methods to delineate these structures and take them into account in the registration.
%\\
%Finally, we present the construction of an anatomical atlas of the head and neck region and its evaluation on a database of patients. We show in this part the feasibility of the use of an atlas for this region, as well as a simple method to evaluate the registration methods used to build an atlas.
%\\
%All this research work has been implemented in a commercial software (Imago from DOSIsoft), allowing us to validate our results in clinical conditions.
%\\
%{\large\textbf{Keywords:}}
%Atlas-based Segmentation, non rigid registration, radiotherapy, atlas creation
%\\
%\noindent\rule[2pt]{\textwidth}{0.5pt}
\end{vcenterpage}

\end{document}

