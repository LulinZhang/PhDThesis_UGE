%!TEX root = Manuscript.tex

%解释DSM,DoD,GCP,multi-epoch, intra and inter epoch, GT
\chapter{Introduction}
\label{chap:intro}
\minitoc

\section{Motivation and objectives}
\subsection{Why are historical images interesting}
Historical (i.e., analogue or archival) aerial images play an important role in providing unique information about evolution of land-covers. 
%They are objective witness over time and sometimes the only remaining visual source of historical land-form. 
They are valuable assets for a wide range of applications such as (1) analyzing of natural disasters (e.g., earthquake, landslides, volcano, floods, avalanche, etc), (2) eco-environmental monitoring (e.g., forest, atmosphere, glaciers, water, etc), (3) urban expansion and (4) environmental pollution and protection and so on.
%, (4) change detection

% (1) change detection, (2) spatial and urban planning, (3) long-term environmental monitoring including but not limited to forests, ice glaciers and coastlines, (4) analysis of natural disaster (e.g., earthquake, volcano eruption, landslide, to name a few) and the estimation of its future trends, etc.
%用途:地震,冰川,土地规划?

%long-term environmental monitoring: forests, ice glaciers, coastlines etc.
%change detection; 
%spatial and urban planning, boundary disputes, etc.
%enable the analysis of natural disaster (e.g., earthquake, volcano eruption, landslide ect.) and the estimation of its future trends.
\par
Historical aerial images have regularly been acquired since the 1920’s by mapping, military or cadastral agencies all over the world. A mass amount of them have been digitized and made accessible through web services~\cite{sebastien2019archiving,earthexplorer,remonterletemps}. 
For example, according to a survey taken place at the beginning of 2017 in Europe~\cite{sebastien2019archiving}, there are approximately 50 million of aerial images archived in Europe, with around 37.8\% of them digitized.
%A survey called "State of current practices concerning archival aerial images" in Europe~\cite{sebastien2019archiving} took place in 2017, 19 organizations from 13 countries (including Austria, Norway, Finland, Spain, Slovenia, Cyprus, Czech Republic, Finland, Germany, Switzerland, Sweden, Poland, France and United Kingdom) replied. According to it, there are in total 50 million of aerial images archived, with around 37.8\% of them digitized.
The images are of high spatial resolution, and are acquired in stereoscopic configuration, allowing for 3D restitution of territories. 
They are often accompanied by metadata, in most cases including the camera focal length and the physical sensor size, which are usually saved or mentioned on the films. Other metadata such as flight plans, camera calibration certificates or orientations are not commonly available. 
\par
When the camera calibration parameters are unknown, they should be evaluated by a process called self-calibrating bundle adjustment. \ac{GCP}s are required, otherwise inaccurately estimated camera parameters will lead to systematic error surfaces called dome effect (i.e., bowel effect).
Generally, \ac{GCP}s originate from (1) field surveys \cite{micheletti2015application,walstra2004time,cardenal2006use}, (2) recent orthophotos and DEM \cite{nurminen2015automation,ellis2006measuring,fox2008unlocking} and (3) recent satellite images \cite{ellis2006measuring,ford2013shoreline}. The most challenging part is to identify the \ac{GCP}s on the historical images, which is not easy due to inevitable scene changes. \ac{GCP}s are usually manually measured with the help of recent photos, however, it is still monotonous and time-consuming. 
There is an urgent need to automatically identify corresponding points (i.e., matches) on historical and recent images.\\
When users are only interesting in comparing different historical epochs, the self-calibration can be accomplished without \ac{GCP}s. Matches between different epochs would serve as observations in bundle adjustment to eliminate the systematic errors in surfaces. In conclusion, the bottleneck of historical image self-calibration located in recovering matches on images taken at different times (i.e., multi-epoch).
%困难:
%(1) 数据异构: images exhibit highly heterogeneous spatial resolutions, with very different acquisition conditions (season, sensors etc.); 
%(2) camera's parameters  may not be available as they might  be  lost  during  the  years  or  never documented; inappropriate film/glass plate preservation; deformation due to scanning process; 
%(3)Drastic scene changes
%(4)Low radiometric quality: old-fashioned radiometric characteristics; noisy; even scratchs on the original films; deterioration due to the aging of the film

%Corresponding points 
%Besides, inappropriate film/glass plate preservation and the scanning process enforce reestimating of the camera calibrations (i.e., the self-calibrating).\\

\subsection{How to match multi-epoch historical images}
However, matching multi-epoch historical images remains challenging, despite the fact that there exists a large number of image matching algorithms with their effectiveness proven on modern images. The reasons include:
\begin{enumerate}
	\item Multi-epoch images are often acquired at different times of day and in various weathers or seasons, which unavoidably leading to appearance differences.
	\item The scene changes over time due to anthropogenic phenomena (e.g., urban planning) or natural ones (e.g., earthquake), especially for large time gaps.
	\item Multi-epoch images often exhibit heterogeneous spatial resolutions, accompanied with different acquisition conditions (sensors, spectral channels, etc.).
	\item Historical images are often facing low radiometric quality, including low contrast, image noise, deterioration due to the aging of films, or even scratches on the films.
\end{enumerate}
Simply applying \textit{state-of-the-art} feature matching methods (e.g., SIFT~\cite{lowe2004distinctive} or SuperGlue~\cite{sarlin2020superglue}) on multi-epoch image pair often leads to unsatisfactory results. An example is given in Figure~\ref{MultiEpochImgPair}. A pair of multi-epoch images are demonstrated with red rectangles indicating the common area in Figure~\ref{MultiEpochImgPair}(a). The left and right images are taken at the same place in 1954 and 1970 individually. The scene changed significantly, a lot of new buildings arose, the color tones are very different. In Figure~\ref{MultiEpochImgPair}(b-d), the matching result of SIFT, SuperGlue and Ours are displayed for comparison. As can be seen, SIFT failed to find any matches. SuperGlue recovered 369 matches, most of which seem good, but at a closer look the details reveal poor localization precision. Our method found 1463 matches with high accuracy, thanks to the help of (1) 3D geometry and (2) the divide and conquer strategy, which are elaborated in the following texts.
\begin{figure*}[htbp]
	\begin{center}
		\subfigure[Multi-epoch image pair]{
			\begin{minipage}[t]{0.45\linewidth}
				\centering
				\includegraphics[width=6.2cm]{images/Chapitre1/OIS-Reech_IGNF_PVA_1-0__1954-03-06__C3544-0211_1954_CDP866_0630_OIS-Reech_IGNF_PVA_1-0__1970__C3544-0221_1970_CDP6452_1409.png}
			\end{minipage}%
		}
		\subfigure[Result of SIFT (0 matches)]{
	\begin{minipage}[t]{0.45\linewidth}
		\centering
		\includegraphics[width=6.2cm]{images/Chapitre1/Homol-SIFT_OIS-Reech_IGNF_PVA_1-0__1954-03-06__C3544-0211_1954_CDP866_0630_OIS-Reech_IGNF_PVA_1-0__1970__C3544-0221_1970_CDP6452_1409.png}
	\end{minipage}%
}
		\subfigure[Result of SuperGlue]{
	\begin{minipage}[t]{0.45\linewidth}
		\centering
		\includegraphics[width=6.2cm]{images/Chapitre1/Homol-SuperGlue_OIS-Reech_IGNF_PVA_1-0__1954-03-06__C3544-0211_1954_CDP866_0630_OIS-Reech_IGNF_PVA_1-0__1970__C3544-0221_1970_CDP6452_1409.png}
	\end{minipage}%
}
		\subfigure[Result of Ours]{
	\begin{minipage}[t]{0.45\linewidth}
		\centering
		\includegraphics[width=6.2cm]{images/Chapitre1/Homol-Ours_OIS-Reech_IGNF_PVA_1-0__1954-03-06__C3544-0211_1954_CDP866_0630_OIS-Reech_IGNF_PVA_1-0__1970__C3544-0221_1970_CDP6452_1409.png}
	\end{minipage}%
}
		\caption{(a) A pair of multi-epoch images with red rectangles indicating the common area. (b-d) Matching result of SIFT, SuperGlue and Ours.}
		\label{MultiEpochImgPair}
	\end{center}
\end{figure*}

%\subsubsection{Take advantage of 3D geometry}
\paragraph{Advantages of 3D geometry}
RGB images are widely used for image matching. However, they have the following shortcomings:\\
(1) Their appearances change over time (see Figure~\ref{AppearanceChange}), and over varying view angles on non-Lambertian surfaces (see Figure~\ref{PoorlyTextured}).\\
(2) Self similarities (e.g., repetitive patterns) favor false matches (see Figure~\ref{PoorlyTextured}).\\
Fortunately, 3D geometry such as \ac{DSM} makes up for these shortcomings perfectly. As can be seen in Figure~\ref{AppearanceChange}, the RGB images look very different because the scene changed a lot. However, the corresponding \ac{DSM}s look similar, which is reasonable, as the 3D landscape is more stable over time. Besides, \ac{DSM} is more distinctive than RGB images when it comes to non-Lambertian surfaces and repetitive patterns, as shown in Figure~\ref{PoorlyTextured}. 
Even though 3D geometry lacks textures and details compared to RGB images, it serves as an ideal supplement. Besides, it plays an important role in provides the 3D information for establish 3D Helmert transformation model between epochs to (1) move different epochs into the same coordinate frame and (2) remove false matches in a RANSAC routine which is more reliable than 2D transformation models.

\begin{figure*}[htbp]
	\begin{center}
		\subfigure[Image 1971]{
			\begin{minipage}[t]{0.45\linewidth}
				\centering
				\includegraphics[width=6.2cm]{images/Chapitre1/AppearanceChangeRGBL.png}
			\end{minipage}%
		}
		\subfigure[Image 2015]{
			\begin{minipage}[t]{0.45\linewidth}
				\centering
				\includegraphics[width=6.2cm]{images/Chapitre1/AppearanceChangeRGBR.png}
			\end{minipage}%
		}
		\subfigure[\ac{DSM} 1971]{
			\begin{minipage}[t]{0.45\linewidth}
				\centering
				\includegraphics[width=6.2cm]{images/Chapitre1/AppearanceChangeDSML.png}
			\end{minipage}%
		}
		\subfigure[\ac{DSM} 2015]{
			\begin{minipage}[t]{0.45\linewidth}
				\centering
				\includegraphics[width=6.2cm]{images/Chapitre1/AppearanceChangeDSMR.png}
			\end{minipage}%
		}
		\caption{The same zone observed in different times. The images changed a lot while the \ac{DSM}s stayed stable over time.}
		\label{AppearanceChange}
	\end{center}
\end{figure*} 


\begin{figure*}[htbp]
	\begin{center}
		\subfigure[Image 1971]{
			\begin{minipage}[t]{0.45\linewidth}
				\centering
				\includegraphics[width=6.2cm]{images/Chapitre1/PoorlyTexturedRGBL.png}
			\end{minipage}%
		}
		\subfigure[Image 2015]{
			\begin{minipage}[t]{0.45\linewidth}
				\centering
				\includegraphics[width=6.2cm]{images/Chapitre1/PoorlyTexturedRGBR.png}
			\end{minipage}%
		}
		\subfigure[\ac{DSM} 1971]{
			\begin{minipage}[t]{0.45\linewidth}
				\centering
				\includegraphics[width=6.2cm]{images/Chapitre1/PoorlyTexturedDSML.png}
			\end{minipage}%
		}
		\subfigure[\ac{DSM} 2015]{
			\begin{minipage}[t]{0.45\linewidth}
				\centering
				\includegraphics[width=6.2cm]{images/Chapitre1/PoorlyTexturedDSMR.png}
			\end{minipage}%
		}
		\caption{The same vegetation observed in different times. They present non-Lambertian reflection and self similarities, while the \ac{DSM}s are distinctive.}
		\label{PoorlyTextured}
	\end{center}
\end{figure*} 

%\subsubsection{Rough-to-precise strategy}
\paragraph{Divide and Conquer}
\zll{Since the task of recovering robust and precise matches on multi-epoch image pairs is difficult, we divide the task into two sub-tasks and conquer them individually with the "rough-to-precise" strategy.} It is illustrated in Figure~\ref{rough-to-precise}. The two sub-tasks includes:\\
%\par 
\begin{enumerate}
	\item Rough co-registration, as illustrated in Figure~\ref{rough-to-precise}(b). Its goal is to roughly align the multi-epoch image pairs by focusing on robustness and relaxing the requirement for accuracy.
	\item Precise matching, as illustrated in Figure~\ref{rough-to-precise}(c). It refines the matches predicted by the rough co-registration result by searching only the local neighborhood to reduce ambiguity.
\end{enumerate}
%\par
%分成CoReg和Precise,前者侧重robust,后者侧重精度.
\begin{figure*}[htbp]
	\begin{center}
		\subfigure[Example of an inter-epoch image pair]{
			\begin{minipage}[t]{1\linewidth}
				\centering
				\includegraphics[width=1\columnwidth]{images/Chapitre1/imagepair.png}
			\end{minipage}%
		}
		\subfigure[Rough co-registration]{
			\begin{minipage}[t]{1\linewidth}
				\centering
				\includegraphics[width=0.68\columnwidth]{images/Chapitre1/CoReg.jpg}
			\end{minipage}%
		}
		\subfigure[Precise matching]{
			\begin{minipage}[t]{1\linewidth}
				\centering
				\includegraphics[width=0.55\columnwidth]{images/Chapitre1/Precise.png}
			\end{minipage}%
		}
		\caption{ Rough-to-precise strategy. (a) An example of an inter-epoch image pair to be matched. $I^{e_1}$ and $I^{e_2}$ represents images take at $epoch_1$ and $epoch_2$ individually. (b) Illustration of rough co-registration between $I^{e_1}$ and $I^{e_2}$. As a result, $I^{e_1}$ is roughly but not precisely aligned with $I^{e_2}$. (c) Illustration of precise matching. For keypoints in $I^{e_1}$ (green cross), a location is predicted in $I^{e_2}$ (purple cross) based on rough co-registration, whose local neighborhood will be searched to find the precise match (yellow cross).}
		\label{rough-to-precise}
	\end{center}
\end{figure*}

%\section{Objective}

\section{Contributions}
\label{sec:contributions}
%Our main contributions include:\\
%\begin{enumerate}
%	\item Rough-to-precise matching strategy that helps to drastically reduce ambiguity. In particular, we use the depth information to roughly co-register our epochs. The 3D landscape is globally stable over time and provides sufficient correspondences through time. Once co-registered, we levarage the 3D a priori to narrow down the search space in precise matching.
%	\item Upscaling of the learning based feature matching algorithms to high resolution imagery. To do that, we introduced an image tiling scheme.
%\end{enumerate}

In this thesis we present a rough-to-precise pipeline for matching multi-epoch images. It is suitable for aerial, satellite and mixed images, which is helpful for geo-referencing historical images without any \ac{GCP}s required. Six variants are provided for the rough co-registration stage and two variants for the precise matching stage. Each variant has its own characteristic:\\
\begin{enumerate}
	\item For rough co-registration variants: \zll{(1) the ones based on the idea of matching \ac{DSM}s generally lead to the most robust results; (2) the ones that match orthophotos could serve as an alternate in rare scenarios with perfectly flat terrain where \ac{DSM}s fail to provide useful information; (3) the others that match original image pairs often lead to less satisfactory results, but they are the only options suitable for terrestrial images.}
	\item For precise matching variants: (1) $Patch$ is based on learned matching methods, it generally results in more matches as it is more invariant over time. (2) $Guided$ is based on hand-crafted methods, it is more efficient in terms of the use of memory and CPU resources as it doesn't involve resampling patches, which is necessary for $Patch$. %often provides relatively more accurate matches under moderate scene changes.
\end{enumerate}
\par
Our method aims to unlock the huge potential of historical images for tracking environmental conditions.
% such as (1) the impact of natural disasters (e.g., earthquake, landslides, volcano, floods, etc), (2) urban expansion, (3) landuse evolution (e.g., cultivated land, forest, meadow, badlands, etc), 
%finding changes in vegetation, evaluating desertification and detecting specific urban or natural variations in the environment. 
We are currently collaborating with several institutes to apply our method in different applications, including:
\begin{enumerate}
	\item Institut de Physique du Globe de Paris (IPGP) and Korea Institute of Geoscience and Mineral Resources (KIGAM) for quantification of deformations of the earth crust to understand the seismic events.
	\item National Research Council, Research Institute for Hydrogeological Protection (CNR-IRPI) for analyzing landslide evolution in Italy.
	\item Department of Earth and Environmental Sciences in University of Pavia for analyzing badland evolution in Europe.
\end{enumerate}
\par
We also developed two thorough tutorials accompanied with test dataset to familiarize users with our pipeline implemented in MicMac\cite{HistoPcode} (more details are introduced in Appendix \ref{chap:appendixC}):
\begin{enumerate}
	\item Tutorial of matching aerial images \cite{tuto-aerial} %\href{https://colab.research.google.com/drive/1poEXIeKbPcJT_2hyQOBhzcj1EEhO8OgD}
	\item Tutorial of matching mixed images (i.e., aerial and satellite images) \cite{tuto-mixed} %\href{https://colab.research.google.com/drive/14okQ8bBhEZmy6EGRIQvazTqrN39oc_K5}
\end{enumerate}

\par
Publications of the author:
\begin{enumerate}
	\item Zhang, L., Rupnik, E., Pierrot-Deseilligny, M., 2021. Feature matching for multi-epoch historical aerial images. ISPRS Journal of Photogrammetry and Remote Sensing, 182, 176-189.
	\item Lulin Zhang, Ewelina Rupnik and Marc Pierrot-Deseilligny.	Guided feature matching for multi-epoch historical image blocks pose estimation. In ISPRS Ann. Photogramm. Remote Sens. Spatial Inf. Sci., 2020.
\end{enumerate}
%, blog \cite{HistoPBlog}
We also provide video \cite{HistoPVideo}, slides \cite{HistoPSlides} and project website \cite{HistoPProj} to improve the visibility of our work.

\section{Organization of the thesis}
This thesis presents a fully automatic pipeline to match inter-epoch images. %It consists of two steps: a rough co-registration which roughly align the whole block by building a globally consistent transformation model between epochs, and a precise matching under the guidance of the rough co-registration result to reduce ambiguity. \\
A brief presentation of the \textit{state-of-the-art} is given in \textbf{Chapter}~\ref{chap:review}. 

In \textbf{Chapter}~\ref{chap:ApplicationsAndDatasets}, applications as well as 5 sets of representative datasets are introduced, which would be used to test our pipeline.

In \textbf{Chapter}~\ref{chap:RoughCoReg}, six rough co-registration variants are elaborated to roughly align the whole block by building a globally consistent transformation model between epochs.
Experiments as well as comparisons are presented.

In \textbf{Chapter}~\ref{chap:Precisematching}, two precise matching variants are introduced to get accurate matches under the guidance of rough co-registered orientations and \ac{DSM}s.

Finally, in \textbf{Chapter} ~\ref{chap:conclusion} conclusion and perspective are given.

