%!TEX root = Manuscript.tex

\chapter{Precise matching}
\label{chap:Precisematching}
\minitoc

\section{Introduction}
\subsection{Objective and motivation}
%Inter-epoch image pairs are roughly aligned 
The rough co-registration stage elaborated in chapter~\ref{chap:RoughCoReg} laid a solid foundation for matching inter-epoch images, as they roughly aligned images from different epochs in a globally consistent way. However, the alignment is not accurate enough for high precision ground surveys. Therefore, we propose a precise matching stage to get matches with higher accuracy. 
For each inter-epoch image pair $I^{e_1}$ and $I^{e_2}$ to be matched, our goal is to find precise matches $M({\mathbf{K}^{e_1},\mathbf{K}^{e_2}})$ ($\mathbf{K}^{e_i}$ represents keypoints in image $I^{e_i}$). 
Based on the roughly co-registered orientations and \ac{DSM}s resulted from chapter~\ref{chap:RoughCoReg}, 
we can readily predict a potential matching point $\widetilde{\mathbf{K}}^{e_2}$ in $I^{e_2}$ for keypoint ${K}^{e_1}$ in $I^{e_1}$. As rough co-registration provides robust alignment which is inferior in accuracy, the precise matching point for keypoint ${K}^{e_1}$ must located in somewhere near the predicted point $\widetilde{\mathbf{K}}^{e_2}$. Therefore, we can narrow down the search space in precise matching stage by only considering the neighborhood of the predicted point to reduce the ambiguity tremendously.\\
Similar to rough co-registration, both SIFT and SuperGlue are adopted in our precise matching pipeline.\\
For hand-crafted methods like SIFT, the strategy of predicting keypoints followed by narrowing down the search space can be readily applied. Besides, as SIFT provides explicitly the scale and rotation angle of the keypoints, we can take advantage of that to check if they coincide with the scale and rotation difference predicted by rough co-registration.\\
For learned methods like SuperGlue, it is not easy to intervene the algorithm, as it inevitably involves retraining the model, which is not a small workload. Therefore we propose an \textit{one-to-many tiling scheme} to feed roughly aligned patches into the model to get precise matches, the merits are twofold: (1) up-scaling the learning based feature matching algorithms to high resolution imagery, as directly feeding the original images often lead to unsatisfactory results; (2) narrowing down the searching space in an elegant way without modifying the model.
%key-point predictions from one can be readily made 
%\subsection{Motivation}
\subsection{Contributions}
Our main contributions include:\\
\begin{enumerate}
	\item Reduce the difficulty in precise matching under the guidance of co-registered orientations and DSMs by: (1) narrowing down the search space; (2) reducing the combinatorial complexity since only overlapping images are analyzed.
	\item Introduce \textit{one-to-many tiling scheme} to (1) scale-up the deep learning methods and (2) narrow down the searching space without retraining the model.
	\item Take scale and rotation information of SIFT keypoint into consideration to remove potential matches that are not consistent with the prediction of rough co-registration.
	\item Perform RANSAC to estimate the 3D Helmert transformation between surfaces (i.e., DSMs) calculated in different epochs. Compared to the classical essential/fundamental matrix filtering, with less data (3 versus 5 points) we impose stricter rules on the sets of points. 
\end{enumerate}

\section{Methodology}
To compute precise {inter-epoch} matches, we perform matching on original RGB images {under the guidance of co-registered orientations and DSMs}. {It consists of extracting tentative inter-epoch matches, followed by a 3D-RANSAC filter and a cross correlation stage to remove outliers.} The workflow is displayed in Figure~\ref{WorkflowPatch}(a).
\par
We choose matching RGB images for precise matching instead of DSMs, as the DSMs are noisy due to low radiometric quality of the images. In section ~\ref{CompareRGBDSM} %Appendix~\ref{chap:appendix4} 
we compared the matching results on RGB images and DSMs over the same area. It demonstrated that there are more numerous but less accurate matches found in DSMs. As the goal of precise matching is to recover accurate matches, the RGB images are more suitable than DSMs.\\

\begin{figure*}[htbp]
	\begin{center}
		\subfigure[Workflow]{
			\begin{minipage}[t]{1\linewidth}
				\centering
				\includegraphics[width=1\columnwidth]{images/Chapitre4/precisematching.png}
			\end{minipage}%
		}
		\subfigure[Patch matching]{
			\begin{minipage}[t]{0.35\linewidth}
				\centering
				\includegraphics[width=4.5cm]{images/Chapitre4/patchmatching.png}
			\end{minipage}%
		}
		\subfigure[Buffer zone of tiles]{
			\begin{minipage}[t]{0.25\linewidth}
				\centering
				\includegraphics[width=3.8cm]{images/Chapitre4/tilingScheme.png}
			\end{minipage}%
		}
		\subfigure[Guided matching]{
			\begin{minipage}[t]{0.35\linewidth}
				\centering
				\includegraphics[width=4.5cm]{images/Chapitre4/guidedmatching.png}
			\end{minipage}%
		}
		\caption{(a) Workflow of precise matching. It is carried out by performing patch or guided matching to obtain tentative matches, followed by 3D-RANSAC filter and cross correlation, giving rise to final matches. (b) and (d) illustrate toy-examples of the patch matching and guided matching, respectively. (c) displays the feature correspondences where $\mathbf{K}^{e_1}$ exceeds the original tile size (dark green area) and therefore will be abandoned.}
		\label{WorkflowPatch}
	\end{center}
\end{figure*}

\subsection{Get tentative matches with patch/guided matching}\label{patch matching}
We offer two alternatives to recover tentative matches: patch or guided matching. {The former has better overall} performance while the latter is more efficient in terms of the use of memory and CPU resources.\\
\subsubsection{Patch matching for learned features}
{{It} is based on \textit{one-to-one tiling scheme}} {(not to confuse with the \textit{one-to-many tiling scheme} presented in the \textit{rough co-registration}), as shown in Figure~\ref{WorkflowPatch}(b) and detailed below}:
\begin{enumerate}
	%\item Crop the master RGB image $I^{e_1}$ into M tiles ($T^{e_1}$) of certain size (640$\times$480 pixels in our experiments), with a buffer zone (64 pixels in the width and 48 pixels in the height in our experiments) overlapped with each other;
	\item Crop the master RGB image $I^{e_1}$ into M original tiles of certain size, and expand them with a buffer zone (as shown in Figure~\ref{WorkflowPatch}(c)), giving rise to M buffered tiles ($T^{e_1}$);
	\item Project each buffered tile $T^{e_1}$ onto the DSM $D_{co}^{e_1}$ and backproject to secondary RGB image $I^{e_2}$ to find the corresponding tile $T^{e_2}$;
	\item Resample $T^{e_2}$ to $\widetilde{T}^{e_2}$, so that the tile pair $P({T^{e_1},\widetilde{T}^{e_2}})$ is free from differences of rotation, scale and extent;
	\item Apply SuperGlue on tile pair $P({T^{e_1},\widetilde{T}^{e_2}})$ to find matches $M({\mathbf{K}^{e_1},\mathbf{K}^{e_2}})$ ($\mathbf{K}^{e_i}$ represents keypoints in $I^{e_i}$);
	\item Merge together the matches from M tile pairs by removing the matches with $\mathbf{K}^{e_i}$ located in the buffer zone.\\
\end{enumerate}

Because the orientations and DSMs are only roughly co-registered, we have to take into account the margin of error when projecting tiles to overlapping images. This is why we add a buffer zone in the tile $T^{e_1}$. In our experiments, the sizes of original and buffered tiles are set to 512$\times$384 and 640$\times$480 pixels individually.\\
For better understanding, in Figure~\ref{patchexample} we displayed an example of an inter-epoch image pair, as well as the tile pairs resulted from the \textit{one-to-one tiling scheme}.\\
Our patch matching experiments are performed based on SuperGlue, however, other learned methods can be adopted readily. \\

\begin{figure*}[htbp]
	\begin{center}
		\subfigure[Example of an image pair]{
			\begin{minipage}[t]{1\linewidth}
				\centering
				\includegraphics[width=1\columnwidth]{images/Chapitre4/example.png}
			\end{minipage}%
		}
		\subfigure[Demonstration of tile pairs]{
			\begin{minipage}[t]{1\linewidth}
				\centering
				\includegraphics[width=1\columnwidth,trim=10 0 0 0,clip]{images/Chapitre4/patchexample.png}
			\end{minipage}%
		}
		%			\subfigure[Example of keypoint prediction]{
		%		\begin{minipage}[t]{1\linewidth}
		%			\centering
		%			\includegraphics[width=1\columnwidth]{images/Chapitre4/guidedexample.png}
		%		\end{minipage}%
		%	}
		\caption{(a) Example demonstration of an image pair. The master image ($I^{e_1}$) and secondary image ($I^{e_2}$) are taken at Fr{\'e}jus in 1954 and 2014 individually. (b) Tile pairs resulted from \textit{one-to-one tiling scheme}, the tile zones before and after buffering are marked as red and green rectangles.}
		\label{patchexample}
	\end{center}
\end{figure*}

\begin{figure*}[htbp]
	\begin{center}
		%\subfigure[Example of keypoint prediction]{
		%	\begin{minipage}[t]{1\linewidth}
				\centering
				\includegraphics[width=1\columnwidth]{images/Chapitre4/guidedexample.png}
		%	\end{minipage}%
		%}
		\caption{Example demonstration of keypoint prediction (cross symbols) accompanied with search space (circles) on an image pair, the master image ($I^{e_1}$) and secondary image ($I^{e_2}$) are taken at Fr{\'e}jus in 1954 and 2014 individually.}
		\label{guidedexample}
	\end{center}
\end{figure*}

\subsubsection{Guided matching for hand-crafted features} 
The patch matching substitute orientated towards hand-crafted features is the guided matching, as shown in Figure~\ref{WorkflowPatch}(d). It leverages the positions of predicted keypoints, {the known scale ratio and rotation differences to narrow down the list of the matching candidates}. In our experiments, we use the SIFT points, but the pipeline is suitable to any hand-crafted extractor.
The strategy {is as follows}:\\
\begin{enumerate}
	\item {Compute the scale ratio $R_{scl}$ and the rotation $D_{rot}$ between two images by sequentially projecting the $I^{e_1}$ image corners to the co-registered DSM $D_{co}^{e_1}$ and to image $I^{e_2}$;} %\textcolor{red}{Project the image corners of $I^{e_1}$ to the co-registered DSM $D_{co}^{e_1}$, and back-project them to $I^{e_2}$ to estimate the scale ratio $R_{scl}$ and angle difference $D_{ang}$ between images $I^{e_1}$ and $I^{e_2}$.}
	\item Extract keypoints $\mathbf{K}^{e_1}$ in image $I^{e_1}$ and $\mathbf{K}^{e_2}$ in image $I^{e_2}$;
	\item Intersect the keypoints $\mathbf{K}^{e_1}$ with the co-registered DSM $D_{co}^{e_1}$;
	\item Back-project them to image $I^{e_2}$, giving rise to predicted keypoints $\widetilde{\mathbf{K}}^{e_2}$;
	\item Search for a subset of points in $\mathbf{K}^{e_2}$ located within a radius $S$ (100 pixels in our experiments) centered at the predicted positions $\widetilde{\mathbf{K}}^{e_2}$;%\textcolor{green}{(I don't use a distance threshold.)}
	\item {Remove candidate matches whose scales and rotations computed by SIFT are incoherent with $R_{scl}$ and $D_{rot}$ computed from image orientations and the co-registered DSM (i.e., step 1);}
	\item {Find the best match with mutual nearest neighbor and apply the first to second nearest neighbor ratio test~\cite{lowe2004distinctive}.}
\end{enumerate}
For better understanding, in Figure~\ref{guidedexample} we displayed an example of an inter-epoch image pair, with demonstration of keypoint prediction (cross symbols) accompanied with search space (circles) superposed on them.\\

\subsection{Get enhanced matches with 3D-RANSAC}
To compute enhanced matches, we apply a 3D-RANSAC filter on the previously obtained tentative matches. {More precisely, we do the following}: (1) for each match $M({\mathbf{K}^{e_1},\mathbf{K}^{e_2}})$, the keypoints $\mathbf{K}^{e_1}$ and $\mathbf{K}^{e_2}$ are projected onto DSM $D_{co}^{e_1}$ and $D_{ini}^{e_2}$ individually to get 3D points $M({\mathbf{G}^{e_1},\mathbf{G}^{e_2}})$; and (2) {the matches} $M({\mathbf{G}^{e_1},\mathbf{G}^{e_2}})$ are iteratively sampled to compute the 3D Helmert transformation RANSAC model:
\begin{equation}
\left [ \begin{array}{c}
{KG}_x^{e_2}\\
{KG}_y^{e_2}\\
{KG}_z^{e_2}
\end{array}
\right ] =\lambda \cdot \mathbf{R} \cdot {\left [ \begin{array}{c}
	{KG}_x^{e_1}\\
	{KG}_y^{e_1}\\
	{KG}_z^{e_1}
	\end{array}
	\right ]} + \left [ \begin{array}{c}
\Delta_x\\
\Delta_y\\
\Delta_z
\end{array}
\right ]. \label{eq:2DSim}
\end{equation}
where $\lambda$ is the scale factor, $\mathbf{R}$ is the rotation matrix and $\left [ \begin{array}{c}
\Delta_x, \Delta_y, \Delta_z
\end{array}
\right ]$ $^{^T}$ is the translation vector.
We set the number of RANSAC iterations to 1000, and consider matches within $T_r$ of its predicted position as inliers. In our experiment, {$T_r$ was set to 10$\times$$GSD$ where $GSD$ is the mean ground sampling distance in the coordinate frame of epoch ${e_2}$. This distance is computed as the ground distance between two adjacent image pixels.}

\subsection{Get final matches with cross correlation}
In the preceding step we got rid of a substantial number of outliers, however, we believe that not all outliers could be identified. 
Besides, our goal is to get moderate number of reliable matches instead of numerous unreliable ones. Therefore we introduce another filtering method, cross-correlation, which is different than the previous step to remove false matches slipped through the net. Matches with their correlation scores below a predefined threshold (0.6 in our experiments) are discarded. The correlation window size was set to be large enough to take into account the context around a point (32$\times$32 pixels in our experiment). Figure~\ref{crossc} shows an example of a false match (red) eliminated by cross correlation, while the true match (blue) is kept.
\begin{figure*}[htbp]
	\begin{center}
		\includegraphics[width=0.8\columnwidth]{images/Chapitre4/tiept.png}
		\caption{Demonstration of the validation with cross-correlation. Considering poor quality of historical images, the window size (blue and red rectangles) was set to 32$\times$32 pixels. False match (red) is eliminated by cross correlation, while true match (blue) is kept.}
		\label{crossc}
	\end{center}
\end{figure*}

\subsection{Refine orientations}
Based on the intra-epoch and inter-epoch matches, a free network \ac{BBA} is performed to refine all the image orientations and camera calibrations. If the results need to be analyzed in a metric scale, a spatial similarity transformation will be performed to move the refined acquisitions in an arbitrary reference frame to a metric one. If the precise orientations for one of the epochs were known, their parameters will be fixed during the \ac{BBA} and the subsequent spatial similarity transformation will be skipped.
We adopted the Fraser model ~\cite{fraser1997digital} to calibrate the cameras and allowed image-dependent affine parameters, the remaining parameters were shared among all images.


\section{Experiments}
As described in the previous section, our precise matching pipeline consists of 3 main steps to get the tentative, enhanced and final matches. There are 2 alternatives for obtaining tentative matches (i.e. patch or guided matching), leading to 2 precise matching methods:\\
\begin{enumerate}
	\item \textit{Patch}: recover tentative matches with patch matching, followed by 3D-RANSAC and cross correlation to remove outliers;
	\item \textit{Guided}: same as \textit{Patch}, except replacing patch matching with guided matching.
\end{enumerate}
For each dataset, we choose the rough co-registration results calculated with $SIFT_{DSM}$ and $SuperGlue_{DSM}$ individually (as they are the most robust methods for rough co-registration) to guide the precise matching \textit{Patch} or \textit{Guided}, leading to 4 sets of result, which are referred to as:\\
\begin{enumerate}
	\item $Patch_{SpGDSM}$
	\item $Guided_{SpGDSM}$
	\item $Patch_{SIFTDSM}$
	\item $Guided_{SIFTDSM}$
\end{enumerate}

\subsection{Implementation details}
Same as Section ~\ref{Implementationdetails}, all input images are downsampled by a factor of 3 beforehand to improve efficiency except for dataset Alberona. To calculate the DSMs, we further downsample the images by a factor of 4 (different from 8 in Section ~\ref{Implementationdetails}), which amounts to a total downsampling factor of 12 with respect to the input images (total factor of 4 for Alberona). For example, the images in Fr{\'e}jus 1970 are downsampled from [8766, 8763] to [730, 730] for calculating DSMs. 
Note that the DSMs serve 2 purposes in precise matching: (1) narrowing down the search space in precise matching, (2) providing 3D coordinates for 3D-RANSAC filter. A low resolution surface is good enough for these tasks and improves the efficiency.\\
To balance the number of the intra- and inter-epoch matches, we perform intra-epoch matches reduction available as command \textit{Ratafia} in MicMac~\cite{marc2016micmac}. %, with the objective of limiting the number of matches while maintaining a good photogrammetric distribution. 
If the intra-epoch matches after reduction are still obviously more numerous than the inter-epoch ones, we can set the relative observation weight in the \ac{BBA}. The matches reduction algorithm maximizes good spatial distribution, points' multiplicity and low reprojection error, it also helps to speed up the \ac{BBA}.\\
Inter-epoch matches are extracted {for every possible combination of 2 epochs and finally merged}.\\


\subsection{Datasets}
We tested our precise matching methods on the 4 sets of datasets which are elaborated in Chapter~\ref{chap:ApplicationsAndDatasets}: Fr{\'e}jus, Pezenas, Kobe and Alberona.

%We tested our precise matching methods on the 4 sets of datasets: Fr{\'e}jus, Pezenas and Kobe. Details of the datasets are demonstrated in Section~\ref{Datasets}.\\
For Fr{\'e}jus, Kobe and Alberona, we keep all the epochs for experiments, as Fr{\'e}jus displayed drastic scene changes, while Kobe and Alberona witnessed earthquake and landslide individually. For Pezenas, we choose aerial epoch 1971 and satellite epoch 2014 to test our precise matching and ignore other epochs to simplify the processing, since Pezenas is less challenging case.\\
The orientations of \ac{GT} epochs (i.e. Pezenas 2014 and Fr{\'e}jus 2014) were treated as fixed during the combined \ac{BBA} since they were accurately known \textit{a-priori}, while all the remaining orientations were considered as free parameters. At first, {interior orientation parameters} were shared among all images. Once stable initial values were known, interior parameters were further refined with image-dependent affine parameters. The affine component of the camera calibration is expected to model, at least partially, the shear of the analog film.\\
\subsection{Evaluation}
%Bar chart of recovered match numbers, Match visulization after CC (4 methods, 3 bars each. Kobe and Pezenas: one kit; Frejus: 6 kits)? DoD and statistic info, ground displacment, check pts? 
In order to evaluate the results qualitatively and quantitatively, the following criterias would be applied:\\
\begin{enumerate}
	\item \textbf{Matches visualization}. The number of tentative, enhanced and final matches are displayed together in bar charts; in the meantime, the final matches are visualized and demonstrated.
	%    \item Ground check points: the co-registered orientations calculated by our methods would be used to triangulate the ground check points and the coordinate differences \textcolor{red}{(explain the diff means the mean diff of x, y and z)} will be displayed. The better the epochs co-register, the smaller the difference is.
	\item \textbf{\ac{DoD}}. For each method, the refined orientations would be used to calculate DSMs in order to generate \ac{DoD}. The visualization of \ac{DoD} as well as the statistical information would be displayed. Since the orientations are refined with precise matches, \ac{DoD}s with dome effect mitigated or even eliminated are expected.\\
	For Pezenas and Fr\'ejus datasets, DoDs are calculated between historical epochs and the available \ac{GT} epochs. For Kobe and Alberona datasets there is no \ac{GT} so we calculate the \ac{DoD}s between the 2 epochs available in each dataset instead.\\
	%ideally the DoD should only display the scene changes without dome effect. If the dome effect appears, it indicates the systematic errors caused by poorly estimated camera parameters.   
	\item \textbf{Ground displacement}. For the dataset that witnessed an earthquake (i.e. Kobe), we: (1) calculate the DSMs; (2) orthorectify the images; and (3) perform 2D correlation of the respective orthophotos ~\cite{rosu2015measurement} to see whether we can observe the slip of the tectonic plate.
\end{enumerate}


\subsection{Comparison}
\subsubsection{Matches visualization}
\label{matchVizMainBody}
For each dataset, every possible combination of 2 epochs are matched with 4 methods (\ding{172} $Patch_{SpGDSM}$, \ding{173} $Guided_{SpGDSM}$, \ding{174} $Patch_{SIFTDSM}$ and \ding{175} $Guided_{SIFTDSM}$). 

As the matches showed similar pattern, for the sake of simplicity, the matches visualizations of datasets Fr{\'e}jus and Alberona are displayed in this chapter, the rest results of Pezenas and Kobe are demonstrated in Section~\ref{sec:PrecisematchViz}.\\

%The matches for dataset Alberona are visualized and displayed in Figure~\ref{MatchVizAlberona}. Matches visualization for datasets Fr{\'e}jus, Pezenas and Kobe are displayed in Section~\ref{chap:appendixB} for the sake of simplicity.\\
\begin{enumerate}
	\item For Fr{\'e}jus, there exist 4 epochs, leading to 6 sets of epoch combination, the matches visualizations of which are displayed in Figure \ref{MatchVizFrejus1954-2014}, \ref{MatchVizFrejus1966-2014}, \ref{MatchVizFrejus1970-2014}, \ref{MatchVizFrejus1954-1970}, \ref{MatchVizFrejus1954-1966} and \ref{MatchVizFrejus1966-1970}. Both $Patch$ and $Guided$ recovered a lot of final matches, except for the ones involving epoch 1954 based on rough co-registration of $SIFT_{DSM}$ (Figue \ref{MatchVizFrejus1954-2014} (e, f), \ref{MatchVizFrejus1954-1970} (e, f) and \ref{MatchVizFrejus1954-1966} (e, f)). The reason lies in the fact that the rough co-registration basic of $SIFT_{DSM}$ for epoch 1954 is inferior, as we mentioned in Section ~\ref{sec:matchViz}.

	\item For Alberona, there exist 2 epochs, leading to 1 set of epoch combination, the matches visualization of which is displayed in Figure~\ref{MatchVizAlberona}. As the basics of rough co-registration are reliable, both $Patch$ and $Guided$ recovered numerous final matches for all the methods.
\end{enumerate}

Besides, common pattern presents in both datasets: 3D-RANSAC filter and cross correlation removed considerable number of matches, at the same time enough matches survived, which guaranteed robustness of our method. Moreover, $Patch$ recovered considerably more matches than $Guided$, which is understandable as SuperGlue is more invariant over time than SIFT.\\

Similar pattern appears for the datasets Pezenas and Kobe. For more details please refer to Section~\ref{sec:PrecisematchViz}.

\begin{figure*}[htbp]
	\begin{center}
		\subfigure[Common zone]{
			\begin{minipage}[t]{0.48\linewidth}
				\centering
				\includegraphics[width=6.8cm]{images/Chapitre3/Pseudo-Ortho-MEC-Malt_Tapas_1954_Ortho-MEC-Malt_2014.png}
			\end{minipage}%
		}
		\subfigure[Number of recovered matches]{
			\begin{minipage}[t]{0.48\linewidth}
				\centering
				\includegraphics[width=5.8cm]{images/Chapitre4/PlotBarH-Frejus1954-2014.png}
			\end{minipage}%
		}
		\subfigure[$Patch_{SpGDSM}$]{
			\begin{minipage}[t]{0.48\linewidth}
				\centering
				\includegraphics[width=6.8cm]{images/Chapitre4/Precise-SpGDSMHomol-1954-2014-SuperGlue-3DRANSAC-CrossCorrelation-PileImg_Ortho-MEC-Malt_Tapas_1954_Ortho-MEC-Malt_2014.png}
			\end{minipage}%
		}
		\subfigure[$Guided_{SpGDSM}$]{
			\begin{minipage}[t]{0.48\linewidth}
				\centering
				\includegraphics[width=6.8cm]{images/Chapitre4/Precise-SpGDSMHomol-1954-2014-GuidedSIFT-3DRANSAC-CrossCorrelation-PileImg_Ortho-MEC-Malt_Tapas_1954_Ortho-MEC-Malt_2014.png}
			\end{minipage}%
		}
		\subfigure[$Patch_{SIFTDSM}$]{
			\begin{minipage}[t]{0.48\linewidth}
				\centering
				\includegraphics[width=6.8cm]{images/Chapitre4/Precise-SIFTDSMHomol-1954-2014-SuperGlue-3DRANSAC-CrossCorrelation-PileImg_Ortho-MEC-Malt_Tapas_1954_Ortho-MEC-Malt_2014.png}
			\end{minipage}%
		}
		\subfigure[$Guided_{SIFTDSM}$]{
			\begin{minipage}[t]{0.48\linewidth}
				\centering
				\includegraphics[width=6.8cm]{images/Chapitre4/Precise-SIFTDSMHomol-1954-2014-GuidedSIFT-3DRANSAC-CrossCorrelation-PileImg_Ortho-MEC-Malt_Tapas_1954_Ortho-MEC-Malt_2014.png}
			\end{minipage}%
		}
		\caption{Precise matching visualization of \textbf{Fr{\'e}jus 1954 and 2014}. (a) Image pairs to be matched, with red rectangles indicating the common zone. (b) Numbers of tentative, enhanced and final matches recovered with $Patch_{SpGDSM}$, $Guided_{SpGDSM}$, $Patch_{SIFTDSM}$ and $Guided_{SIFTDSM}$ individually. (c-f) Visualization of final matches recovered with $Patch_{SpGDSM}$, $Guided_{SpGDSM}$, $Patch_{SIFTDSM}$ and $Guided_{SIFTDSM}$ individually.}
		\label{MatchVizFrejus1954-2014}
	\end{center}
\end{figure*} 


\begin{figure*}[htbp]
	\begin{center}
		\subfigure[Common zone]{
			\begin{minipage}[t]{0.48\linewidth}
				\centering
				\includegraphics[width=6.8cm]{images/Chapitre3/Pseudo-Ortho-MEC-Malt_Tapas_1966_Ortho-MEC-Malt_2014.png}
			\end{minipage}%
		}
		\subfigure[Number of recovered matches]{
			\begin{minipage}[t]{0.48\linewidth}
				\centering
				\includegraphics[width=5.8cm]{images/Chapitre4/PlotBarH-Frejus1966-2014.png}
			\end{minipage}%
		}
		\subfigure[$Patch_{SpGDSM}$]{
			\begin{minipage}[t]{0.48\linewidth}
				\centering
				\includegraphics[width=6.8cm]{images/Chapitre4/Precise-SpGDSMHomol-1966-2014-SuperGlue-3DRANSAC-CrossCorrelation-PileImg_Ortho-MEC-Malt_Tapas_1966_Ortho-MEC-Malt_2014.png}
			\end{minipage}%
		}
		\subfigure[$Guided_{SpGDSM}$]{
			\begin{minipage}[t]{0.48\linewidth}
				\centering
				\includegraphics[width=6.8cm]{images/Chapitre4/Precise-SpGDSMHomol-1966-2014-GuidedSIFT-3DRANSAC-CrossCorrelation-PileImg_Ortho-MEC-Malt_Tapas_1966_Ortho-MEC-Malt_2014.png}
			\end{minipage}%
		}
		\subfigure[$Patch_{SIFTDSM}$]{
			\begin{minipage}[t]{0.48\linewidth}
				\centering
				\includegraphics[width=6.8cm]{images/Chapitre4/Precise-SIFTDSMHomol-1966-2014-SuperGlue-3DRANSAC-CrossCorrelation-PileImg_Ortho-MEC-Malt_Tapas_1966_Ortho-MEC-Malt_2014.png}
			\end{minipage}%
		}
		\subfigure[$Guided_{SIFTDSM}$]{
			\begin{minipage}[t]{0.48\linewidth}
				\centering
				\includegraphics[width=6.8cm]{images/Chapitre4/Precise-SIFTDSMHomol-1966-2014-GuidedSIFT-3DRANSAC-CrossCorrelation-PileImg_Ortho-MEC-Malt_Tapas_1966_Ortho-MEC-Malt_2014.png}
			\end{minipage}%
		}
		\caption{Precise matching visualization of \textbf{Fr{\'e}jus 1966 and 2014}. (a) Image pairs to be matched, with red rectangles indicating the common zone. (b) Numbers of tentative, enhanced and final matches recovered with $Patch_{SpGDSM}$, $Guided_{SpGDSM}$, $Patch_{SIFTDSM}$ and $Guided_{SIFTDSM}$ individually. (c-f) Visualization of final matches recovered with $Patch_{SpGDSM}$, $Guided_{SpGDSM}$, $Patch_{SIFTDSM}$ and $Guided_{SIFTDSM}$ individually.}
		\label{MatchVizFrejus1966-2014}
	\end{center}
\end{figure*} 


\begin{figure*}[htbp]
	\begin{center}
		\subfigure[Common zone]{
			\begin{minipage}[t]{0.48\linewidth}
				\centering
				\includegraphics[width=6.8cm]{images/Chapitre3/Pseudo-Ortho-MEC-Malt_Tapas_1970_Ortho-MEC-Malt_2014.png}
			\end{minipage}%
		}
		\subfigure[Number of recovered matches]{
			\begin{minipage}[t]{0.48\linewidth}
				\centering
				\includegraphics[width=5.8cm]{images/Chapitre4/PlotBarH-Frejus1970-2014.png}
			\end{minipage}%
		}
		\subfigure[$Patch_{SpGDSM}$]{
			\begin{minipage}[t]{0.48\linewidth}
				\centering
				\includegraphics[width=6.8cm]{images/Chapitre4/Precise-SpGDSMHomol-1970-2014-SuperGlue-3DRANSAC-CrossCorrelation-PileImg_Ortho-MEC-Malt_Tapas_1970_Ortho-MEC-Malt_2014.png}
			\end{minipage}%
		}
		\subfigure[$Guided_{SpGDSM}$]{
			\begin{minipage}[t]{0.48\linewidth}
				\centering
				\includegraphics[width=6.8cm]{images/Chapitre4/Precise-SpGDSMHomol-1970-2014-GuidedSIFT-3DRANSAC-CrossCorrelation-PileImg_Ortho-MEC-Malt_Tapas_1970_Ortho-MEC-Malt_2014.png}
			\end{minipage}%
		}
		\subfigure[$Patch_{SIFTDSM}$]{
			\begin{minipage}[t]{0.48\linewidth}
				\centering
				\includegraphics[width=6.8cm]{images/Chapitre4/Precise-SIFTDSMHomol-1970-2014-SuperGlue-3DRANSAC-CrossCorrelation-PileImg_Ortho-MEC-Malt_Tapas_1970_Ortho-MEC-Malt_2014.png}
			\end{minipage}%
		}
		\subfigure[$Guided_{SIFTDSM}$]{
			\begin{minipage}[t]{0.48\linewidth}
				\centering
				\includegraphics[width=6.8cm]{images/Chapitre4/Precise-SIFTDSMHomol-1970-2014-GuidedSIFT-3DRANSAC-CrossCorrelation-PileImg_Ortho-MEC-Malt_Tapas_1970_Ortho-MEC-Malt_2014.png}
			\end{minipage}%
		}
		\caption{Precise matching visualization of \textbf{Fr{\'e}jus 1970 and 2014}. (a) Image pairs to be matched, with red rectangles indicating the common zone. (b) Numbers of tentative, enhanced and final matches recovered with $Patch_{SpGDSM}$, $Guided_{SpGDSM}$, $Patch_{SIFTDSM}$ and $Guided_{SIFTDSM}$ individually. (c-f) Visualization of final matches recovered with $Patch_{SpGDSM}$, $Guided_{SpGDSM}$, $Patch_{SIFTDSM}$ and $Guided_{SIFTDSM}$ individually.}
		\label{MatchVizFrejus1970-2014}
	\end{center}
\end{figure*} 

%%%%%%%%%%%%%%%%%%Frejus histo

\begin{figure*}[htbp]
	\begin{center}
		\subfigure[Common zone]{
			\begin{minipage}[t]{0.48\linewidth}
				\centering
				\includegraphics[width=6.8cm]{images/Chapitre4/Pseudo-Ortho-MEC-Malt_Tapas_1954_Ortho-MEC-Malt_Tapas_1970.png}
			\end{minipage}%
		}
		\subfigure[Number of recovered matches]{
			\begin{minipage}[t]{0.48\linewidth}
				\centering
				\includegraphics[width=5.8cm]{images/Chapitre4/PlotBarH-Frejus1954-1970.png}
			\end{minipage}%
		}
		\subfigure[$Patch_{SpGDSM}$]{
			\begin{minipage}[t]{0.48\linewidth}
				\centering
				\includegraphics[width=6.8cm]{images/Chapitre4/Precise-SpGDSMHomol-1954-1970-SuperGlue-3DRANSAC-CrossCorrelation-PileImg_Ortho-MEC-Malt_Tapas_1954_Ortho-MEC-Malt_Tapas_1970.png}
			\end{minipage}%
		}
		\subfigure[$Guided_{SpGDSM}$]{
			\begin{minipage}[t]{0.48\linewidth}
				\centering
				\includegraphics[width=6.8cm]{images/Chapitre4/Precise-SpGDSMHomol-1954-1970-GuidedSIFT-3DRANSAC-CrossCorrelation-PileImg_Ortho-MEC-Malt_Tapas_1954_Ortho-MEC-Malt_Tapas_1970.png}
			\end{minipage}%
		}
		\subfigure[$Patch_{SIFTDSM}$]{
			\begin{minipage}[t]{0.48\linewidth}
				\centering
				\includegraphics[width=6.8cm]{images/Chapitre4/Precise-SIFTDSMHomol-1954-1970-SuperGlue-3DRANSAC-CrossCorrelation-PileImg_Ortho-MEC-Malt_Tapas_1954_Ortho-MEC-Malt_Tapas_1970.png}
			\end{minipage}%
		}
		\subfigure[$Guided_{SIFTDSM}$]{
			\begin{minipage}[t]{0.48\linewidth}
				\centering
				\includegraphics[width=6.8cm]{images/Chapitre4/Precise-SIFTDSMHomol-1954-1970-GuidedSIFT-3DRANSAC-CrossCorrelation-PileImg_Ortho-MEC-Malt_Tapas_1954_Ortho-MEC-Malt_Tapas_1970.png}
			\end{minipage}%
		}
		\caption{Precise matching visualization of \textbf{Fr{\'e}jus 1954 and 1970}. (a) Image pairs to be matched, with red rectangles indicating the common zone. (b) Numbers of tentative, enhanced and final matches recovered with $Patch_{SpGDSM}$, $Guided_{SpGDSM}$, $Patch_{SIFTDSM}$ and $Guided_{SIFTDSM}$ individually. (c-f) Visualization of final matches recovered with $Patch_{SpGDSM}$, $Guided_{SpGDSM}$, $Patch_{SIFTDSM}$ and $Guided_{SIFTDSM}$ individually.}
		\label{MatchVizFrejus1954-1970}
	\end{center}
\end{figure*} 



\begin{figure*}[htbp]
	\begin{center}
		\subfigure[Common zone]{
			\begin{minipage}[t]{0.48\linewidth}
				\centering
				\includegraphics[width=6.8cm]{images/Chapitre4/Pseudo-Ortho-MEC-Malt_Tapas_1966_Ortho-MEC-Malt_Tapas_1970.png}
			\end{minipage}%
		}
		\subfigure[Number of recovered matches]{
			\begin{minipage}[t]{0.48\linewidth}
				\centering
				\includegraphics[width=5.8cm]{images/Chapitre4/PlotBarH-Frejus1966-1970.png}
			\end{minipage}%
		}
		\subfigure[$Patch_{SpGDSM}$]{
			\begin{minipage}[t]{0.48\linewidth}
				\centering
				\includegraphics[width=6.8cm]{images/Chapitre4/Precise-SpGDSMHomol-1966-1970-SuperGlue-3DRANSAC-CrossCorrelation-PileImg_Ortho-MEC-Malt_Tapas_1966_Ortho-MEC-Malt_Tapas_1970.png}
			\end{minipage}%
		}
		\subfigure[$Guided_{SpGDSM}$]{
			\begin{minipage}[t]{0.48\linewidth}
				\centering
				\includegraphics[width=6.8cm]{images/Chapitre4/Precise-SpGDSMHomol-1966-1970-GuidedSIFT-3DRANSAC-CrossCorrelation-PileImg_Ortho-MEC-Malt_Tapas_1966_Ortho-MEC-Malt_Tapas_1970.png}
			\end{minipage}%
		}
		\subfigure[$Patch_{SIFTDSM}$]{
			\begin{minipage}[t]{0.48\linewidth}
				\centering
				\includegraphics[width=6.8cm]{images/Chapitre4/Precise-SIFTDSMHomol-1966-1970-SuperGlue-3DRANSAC-CrossCorrelation-PileImg_Ortho-MEC-Malt_Tapas_1966_Ortho-MEC-Malt_Tapas_1970.png}
			\end{minipage}%
		}
		\subfigure[$Guided_{SIFTDSM}$]{
			\begin{minipage}[t]{0.48\linewidth}
				\centering
				\includegraphics[width=6.8cm]{images/Chapitre4/Precise-SIFTDSMHomol-1966-1970-GuidedSIFT-3DRANSAC-CrossCorrelation-PileImg_Ortho-MEC-Malt_Tapas_1966_Ortho-MEC-Malt_Tapas_1970.png}
			\end{minipage}%
		}
		\caption{Precise matching visualization of \textbf{Fr{\'e}jus 1966 and 1970}. (a) Image pairs to be matched, with red rectangles indicating the common zone. (b) Numbers of tentative, enhanced and final matches recovered with $Patch_{SpGDSM}$, $Guided_{SpGDSM}$, $Patch_{SIFTDSM}$ and $Guided_{SIFTDSM}$ individually. (c-f) Visualization of final matches recovered with $Patch_{SpGDSM}$, $Guided_{SpGDSM}$, $Patch_{SIFTDSM}$ and $Guided_{SIFTDSM}$ individually.}
		\label{MatchVizFrejus1966-1970}
	\end{center}
\end{figure*} 


\begin{figure*}[htbp]
	\begin{center}
		\subfigure[Common zone]{
			\begin{minipage}[t]{0.48\linewidth}
				\centering
				\includegraphics[width=6.8cm]{images/Chapitre4/Pseudo-Ortho-MEC-Malt_Tapas_1954_Ortho-MEC-Malt_Tapas_1966.png}
			\end{minipage}%
		}
		\subfigure[Number of recovered matches]{
			\begin{minipage}[t]{0.48\linewidth}
				\centering
				\includegraphics[width=5.8cm]{images/Chapitre4/PlotBarH-Frejus1954-1966.png}
			\end{minipage}%
		}
		\subfigure[$Patch_{SpGDSM}$]{
			\begin{minipage}[t]{0.48\linewidth}
				\centering
				\includegraphics[width=6.8cm]{images/Chapitre4/Precise-SpGDSMHomol-1954-1966-SuperGlue-3DRANSAC-CrossCorrelation-PileImg_Ortho-MEC-Malt_Tapas_1954_Ortho-MEC-Malt_Tapas_1966.png}
			\end{minipage}%
		}
		\subfigure[$Guided_{SpGDSM}$]{
			\begin{minipage}[t]{0.48\linewidth}
				\centering
				\includegraphics[width=6.8cm]{images/Chapitre4/Precise-SpGDSMHomol-1954-1966-GuidedSIFT-3DRANSAC-CrossCorrelation-PileImg_Ortho-MEC-Malt_Tapas_1954_Ortho-MEC-Malt_Tapas_1966.png}
			\end{minipage}%
		}
		\subfigure[$Patch_{SIFTDSM}$]{
			\begin{minipage}[t]{0.48\linewidth}
				\centering
				\includegraphics[width=6.8cm]{images/Chapitre4/Precise-SIFTDSMHomol-1954-1966-SuperGlue-3DRANSAC-CrossCorrelation-PileImg_Ortho-MEC-Malt_Tapas_1954_Ortho-MEC-Malt_Tapas_1966.png}
			\end{minipage}%
		}
		\subfigure[$Guided_{SIFTDSM}$]{
			\begin{minipage}[t]{0.48\linewidth}
				\centering
				\includegraphics[width=6.8cm]{images/Chapitre4/Precise-SIFTDSMHomol-1954-1966-GuidedSIFT-3DRANSAC-CrossCorrelation-PileImg_Ortho-MEC-Malt_Tapas_1954_Ortho-MEC-Malt_Tapas_1966.png}
			\end{minipage}%
		}
		\caption{Precise matching visualization of \textbf{Fr{\'e}jus 1954 and 1966}. (a) Image pairs to be matched, with red rectangles indicating the common zone. (b) Numbers of tentative, enhanced and final matches recovered with $Patch_{SpGDSM}$, $Guided_{SpGDSM}$, $Patch_{SIFTDSM}$ and $Guided_{SIFTDSM}$ individually. (c-f) Visualization of final matches recovered with $Patch_{SpGDSM}$, $Guided_{SpGDSM}$, $Patch_{SIFTDSM}$ and $Guided_{SIFTDSM}$ individually.}
		\label{MatchVizFrejus1954-1966}
	\end{center}
\end{figure*} 

\begin{figure*}[htbp]
	\begin{center}
		\subfigure[Common zone]{
			\begin{minipage}[t]{0.48\linewidth}
				\centering
				\includegraphics[width=4.8cm]{images/Chapitre3/Pseudo-Ortho-MEC-Malt_Tapas_1954_Ortho-MEC-Malt_Tapas_2003.png}
			\end{minipage}%
		}
%		\subfigure[Number of recovered matches]{
%			\begin{minipage}[t]{0.48\linewidth}
%				\centering
%				\includegraphics[width=5.8cm]{images/Chapitre4/PlotBarH-Alberona1954-2003.png}
%			\end{minipage}%
%		}
%		\subfigure[$Patch_{SpGDSM}$]{
%			\begin{minipage}[t]{0.48\linewidth}
%				\centering
%				\includegraphics[width=3.8cm]{images/Chapitre4/Precise-SpGDSMHomol-SuperGlue-3DRANSAC-CrossCorrelation-PileImg_Ortho-MEC-Malt_Tapas_1954_Ortho-MEC-Malt_Tapas_2003.png}
%			\end{minipage}%
%		}
%		\subfigure[$Guided_{SpGDSM}$]{
%			\begin{minipage}[t]{0.48\linewidth}
%				\centering
%				\includegraphics[width=3.8cm]{images/Chapitre4/Precise-SpGDSMHomol-GuidedSIFT-3DRANSAC-CrossCorrelation-PileImg_Ortho-MEC-Malt_Tapas_1954_Ortho-MEC-Malt_Tapas_2003.png}
%			\end{minipage}%
%		}
%		\subfigure[$Patch_{SIFTDSM}$]{
%			\begin{minipage}[t]{0.48\linewidth}
%				\centering
%				\includegraphics[width=3.8cm]{images/Chapitre4/Precise-SIFTDSMHomol-SuperGlue-3DRANSAC-CrossCorrelation-PileImg_Ortho-MEC-Malt_Tapas_1954_Ortho-MEC-Malt_Tapas_2003.png}
%			\end{minipage}%
%		}
%		\subfigure[$Guided_{SIFTDSM}$]{
%			\begin{minipage}[t]{0.48\linewidth}
%				\centering
%				\includegraphics[width=3.8cm]{images/Chapitre4/Precise-SIFTDSMHomol-GuidedSIFT-3DRANSAC-CrossCorrelation-PileImg_Ortho-MEC-Malt_Tapas_1954_Ortho-MEC-Malt_Tapas_2003.png}
%			\end{minipage}%
%		}
		\caption{Precise matching visualization of \textbf{Alberona 1954 and 2003}. (a) Image pairs to be matched, with red rectangles indicating the common zone. (b) Numbers of tentative, enhanced and final matches recovered with $Patch_{SpGDSM}$, $Guided_{SpGDSM}$, $Patch_{SIFTDSM}$ and $Guided_{SIFTDSM}$ individually. (c-f) Visualization of final matches recovered with $Patch_{SpGDSM}$, $Guided_{SpGDSM}$, $Patch_{SIFTDSM}$ and $Guided_{SIFTDSM}$ individually.}
		\label{MatchVizAlberona}
	\end{center}
\end{figure*} 


\subsubsection{DoD}
%The visualization of \ac{DoD}s for dataset Alberona is demonstrated in~\ref{DoDAlberona}. The corresponding statistical information is displayed in ~\ref{PreciseDoDStatistic}.\\

As the \ac{DoD}s showed similar pattern, for the sake of simplicity, 
we only show the results of 2 sets of datasets: Fr{\'e}jus and Alberona, as Fr{\'e}jus is interesting for representing the general case for change detection, and Alberona is interesting as it witnessed landslide. 
For Kobe, the more interesting result is the ground displacement caused by the earthquake, which would be demonstrated later, so we move the \ac{DoD}s of Kobe as well as Pezenas to Section~\ref{sec:PreciseDoD}.\\

The \ac{DoD}s for Fr{\'e}jus and Alberona are demonstrated in Figure~\ref{PreciseDoDFrejus} and ~\ref{PreciseDoDAlberona}. 
In each figure, the \ac{DoD}s resulted from rough co-registered orientations using methods $SuperGlue_{DSM}$ and $SIFT_{DSM}$ (elaborated in Chapter~\ref{chap:RoughCoReg}, hereinafter referred to as DoD$^{SpGDSM}$ and DoD$^{SIFTDSM}$) are displayed as bases, and the \ac{DoD}s resulted from refined orientations using methods $Patch_{SpGDSM}$, $Guided_{SpGDSM}$, $Patch_{SIFTDSM}$ and $Guided_{SIFTDSM}$ (hereinafter termed as DoD$^{Patch_{SpGDSM}}$, DoD$^{Guided_{SpGDSM}}$, DoD$^Patch_{SIFTDSM}$ and DoD$^Guided_{SIFTDSM}$) are given for comparison. 
The corresponding statistical information is displayed in Table~\ref{PreciseDoDStatistic}.\\

For dataset Fr{\'e}jus, as can be seen in Figure~\ref{PreciseDoDFrejus}:\\
\begin{enumerate}
	\item The dome effect appears in all the DoD$^{SpGDSM}$ and DoD$^{SIFTDSM}$ (i.e. the first column of the subgraphs), as the camera parameters are poorly estimated without the precise matches.
	\item For (e) \ac{DoD}$_{Frejus1954}^{Patch_{SIFTDSM}}$ and (f) \ac{DoD}$_{Frejus1954}^{Guided_{SIFTDSM}}$, the dome effect is worse than their corresponding basic (i.e. (d) \ac{DoD}$_{Frejus1954}^{{SIFTDSM}}$).  The reason lies in the fact that the rough co-registration basic is inferior, as we mentioned in Section ~\ref{sec:matchViz}.
	\item For the other \ac{DoD}s resulted from refined orientations (i.e. the second and third columns of subgraphs except for (e) and (f)), the dome effect is effectively mitigated thanks to our numerous and precise matches. In the meantime, the real scene changes are reserved, such as the new buildings and seaports.
\end{enumerate}

For dataset Alberona, as can be seen in Figure~\ref{PreciseDoDAlberona}:\\
\begin{enumerate}
	\item The dome effect presented in DoD$^{SpGDSM}$ and DoD$^{SIFTDSM}$ (the first column of the subgraphs) is mitigated in DoD$^{Patch_{SpGDSM}}$, DoD$^{Guided_{SpGDSM}}$, DoD$^Patch_{SIFTDSM}$ and DoD$^Guided_{SIFTDSM}$ (the second and third column of subgraphs). However, we still observer some systematic errors such as the slightly residual dome effect, which we attribute to: (1) the images are poorly preserved and scanned with non photogrammetric scanner; (2) limited number of images lead to a lack of redundant observation.
	\item In DoD$^{Patch_{SpGDSM}}$, DoD$^{Guided_{SpGDSM}}$, DoD$^Patch_{SIFTDSM}$ and DoD$^Guided_{SIFTDSM}$ (the second and third column of subgraphs), the topographic changes are well detected in the landslide area (indicated with black lines).
\end{enumerate}

Regarding to the comparison between $Patch$ and $Guided$, according to the absolute average value $|\mu|$ of all the \ac{DoD}s in Figure \ref{PreciseDoDFrejus} and \ref{PreciseDoDAlberona} (cf. Talbe \ref{PreciseDoDStatistic}), $Guided$ leads to \ac{DoD}s with slightly better accuracy, even though it is less invariant than $Patch$, it recovers less numerous but more precise matches with the help of rough co-registered orientations and \ac{DSM}s.
%In conclusion:\\
%\begin{itemize}
%	\item Both precise matching pipelines (i.e. $Patch$ and $Guided$) works, as long as the rough co-registration basic is reliable.\\
%\end{itemize}

%, the dome effect presented in DoD$^{SpGDSM}$ and DoD$^{SIFTDSM}$ (the first column of the subgraphs) is effectively mitigated in DoD$^{Patch_{SpGDSM}}$, DoD$^{Guided_{SpGDSM}}$, DoD$^Patch_{SIFTDSM}$ and DoD$^Guided_{SIFTDSM}$ for all the 3 historical epochs (the second and third column of subgraphs), except for (e) \ac{DoD}$_{Frejus1954}^{Patch_{SIFTDSM}}$ and (f) \ac{DoD}$_{Frejus1954}^{Guided_{SIFTDSM}}$. The reason lies in the fact that the rough co-registration basic of (e) and (f) is inferior, as we mentioned in Section ~\ref{sec:matchViz}. For the other \ac{DoD}s resulted from refined orientations, \\
%, thanks to the numerous and accurate matches recovered in the precise matching stage

%Both precise matching pipelines (i.e. $Patch$ and $Guided$) works for all the datasets, as long as the rough co-registration basic is reliable.\\



\begin{figure*}[htbp]
	\begin{center}
		\subfigure[\ac{DoD}$_{Frejus1954}^{{SpGDSM}}$]{
			\begin{minipage}[t]{0.31\linewidth}
				\centering
				\includegraphics[width=3.3cm,trim=740 80 50 230,clip]{images/Chapitre3/DoD1954DSM-SuperGlue.png}
			\end{minipage}%
		}
		\subfigure[\ac{DoD}$_{Frejus1954}^{Patch_{SpGDSM}}$]{
			\begin{minipage}[t]{0.31\linewidth}
				\centering
				\includegraphics[width=3.3cm,trim=840 70 290 340,clip]{images/Chapitre4/DoD1954_Patch_SpGDSM.png}
			\end{minipage}%
		}
		\subfigure[\ac{DoD}$_{Frejus1954}^{Guided_{SpGDSM}}$]{
			\begin{minipage}[t]{0.31\linewidth}
				\centering
				\includegraphics[width=3.3cm,trim=840 70 290 340,clip]{images/Chapitre4/DoD1954_Guided_SpGDSM.png}
			\end{minipage}%
		}\\
		\subfigure[\ac{DoD}$_{Frejus1954}^{{SIFTDSM}}$]{
			\begin{minipage}[t]{0.31\linewidth}
				\centering
				\includegraphics[width=3.3cm,trim=740 100 50 200,clip]{images/Chapitre3/DoD1954DSM-SIFT.png}
			\end{minipage}%
		}
		\subfigure[\ac{DoD}$_{Frejus1954}^{Patch_{SIFTDSM}}$]{
			\begin{minipage}[t]{0.31\linewidth}
				\centering
				\includegraphics[width=3.3cm,trim=840 70 290 340,clip]{images/Chapitre4/DoD1954_Patch_SIFTDSM.png}
			\end{minipage}%
		}
		\subfigure[\ac{DoD}$_{Frejus1954}^{Guided_{SIFTDSM}}$]{
			\begin{minipage}[t]{0.31\linewidth}
				\centering
				\includegraphics[width=3.3cm,trim=840 70 290 340,clip]{images/Chapitre4/DoD1954_Guided_SIFTDSM.png}
			\end{minipage}%
		}\\
		
		\subfigure[\ac{DoD}$_{Frejus1966}^{{SpGDSM}}$]{
			\begin{minipage}[t]{0.31\linewidth}
				\centering
				\includegraphics[width=3.3cm,trim=740 230 50 380,clip]{images/Chapitre3/DoD1966DSM-SuperGlue.png}
			\end{minipage}%
		}
		\subfigure[\ac{DoD}$_{Frejus1966}^{Patch_{SpGDSM}}$]{
			\begin{minipage}[t]{0.31\linewidth}
				\centering
				\includegraphics[width=3.3cm,trim=660 70 80 340,clip]{images/Chapitre4/DoD1966_Patch_SpGDSM.png}
			\end{minipage}%
		}
		\subfigure[\ac{DoD}$_{Frejus1966}^{Guided_{SpGDSM}}$]{
			\begin{minipage}[t]{0.31\linewidth}
				\centering
				\includegraphics[width=3.3cm,trim=660 70 80 340,clip]{images/Chapitre4/DoD1966_Guided_SpGDSM.png}
			\end{minipage}%
		}\\
		\subfigure[\ac{DoD}$_{Frejus1966}^{{SIFTDSM}}$]{
			\begin{minipage}[t]{0.31\linewidth}
				\centering
				\includegraphics[width=3.3cm,trim=740 230 50 380,clip]{images/Chapitre3/DoD1966DSM-SIFT.png}
			\end{minipage}%
		}
		\subfigure[\ac{DoD}$_{Frejus1966}^{Patch_{SIFTDSM}}$]{
			\begin{minipage}[t]{0.31\linewidth}
				\centering
				\includegraphics[width=3.3cm,trim=660 70 80 340,clip]{images/Chapitre4/DoD1966_Patch_SIFTDSM.png}
			\end{minipage}%
		}
		\subfigure[\ac{DoD}$_{Frejus1966}^{Guided_{SIFTDSM}}$]{
			\begin{minipage}[t]{0.31\linewidth}
				\centering
				\includegraphics[width=3.3cm,trim=660 70 80 340,clip]{images/Chapitre4/DoD1966_Guided_SIFTDSM.png}
			\end{minipage}%
		}\\
		
		
		\subfigure[\ac{DoD}$_{Frejus1970}^{{SpGDSM}}$]{
			\begin{minipage}[t]{0.31\linewidth}
				\centering
				\includegraphics[width=3.3cm,trim=680 180 50 260,clip]{images/Chapitre3/DoD1970DSM-SuperGlue.png}				
			\end{minipage}%
		}
		\subfigure[\ac{DoD}$_{Frejus1970}^{Patch_{SpGDSM}}$]{
			\begin{minipage}[t]{0.31\linewidth}
				\centering
				\includegraphics[width=3.3cm,trim=730 50 150 380,clip]{images/Chapitre4/DoD1970_Patch_SpGDSM.png}
			\end{minipage}%
		}
		\subfigure[\ac{DoD}$_{Frejus1970}^{Guided_{SpGDSM}}$]{
			\begin{minipage}[t]{0.31\linewidth}
				\centering
				\includegraphics[width=3.3cm,trim=730 50 150 380,clip]{images/Chapitre4/DoD1970_Guided_SpGDSM.png}
			\end{minipage}%
		}\\
		\subfigure[\ac{DoD}$_{Frejus1970}^{{SIFTDSM}}$]{
			\begin{minipage}[t]{0.31\linewidth}
				\centering
				\includegraphics[width=3.3cm,trim=680 180 50 260,clip]{images/Chapitre3/DoD1970DSM-SIFT.png}
			\end{minipage}%
		}
		\subfigure[\ac{DoD}$_{Frejus1970}^{Patch_{SIFTDSM}}$]{
			\begin{minipage}[t]{0.31\linewidth}
				\centering
				\includegraphics[width=3.3cm,trim=730 50 150 380,clip]{images/Chapitre4/DoD1970_Patch_SIFTDSM.png}
			\end{minipage}%
		}
		\subfigure[\ac{DoD}$_{Frejus1970}^{Guided_{SIFTDSM}}$]{
			\begin{minipage}[t]{0.31\linewidth}
				\centering
				\includegraphics[width=3.3cm,trim=730 50 150 380,clip]{images/Chapitre4/DoD1970_Guided_SIFTDSM.png}
			\end{minipage}%
		}\\
		
		\subfigure[\ac{DoD} legend]{
			\begin{minipage}[t]{1\linewidth}
				\centering
				\includegraphics[width=11cm]{images/Chapitre4/LegendDoD.png}
			\end{minipage}%
		}
		\caption{{\scriptsize (a-f) \ac{DoD}s between free epoch \textbf{Frejus 1954} and reference epoch \textbf{2014}. (g-l) \ac{DoD}s between free epoch \textbf{Frejus 1966} and reference epoch \textbf{2014}. (m-r) \ac{DoD}s between free epoch \textbf{Frejus 1970} and reference epoch \textbf{2014}. (a, d, g, j, m, p) are \ac{DoD}s resulted from roughly co-registered orientations using methods $SuperGlue_{DSM}$ and $SIFT_{DSM}$ (elaborated in Chapter 3). (b, c, e, f, h, i, k, l, n, o, q, r) are \ac{DoD}s resulted from refined orientations using methods $Patch_{SpGDSM}$, $Guided_{SpGDSM}$, $Patch_{SIFTDSM}$ and $Guided_{SIFTDSM}$ individually.}}
		\label{PreciseDoDFrejus}
	\end{center}
\end{figure*} 


\begin{figure*}[htbp]
	\begin{center}
		\subfigure[\ac{DoD}$_{Alberona}^{{SpGDSM}}$]{
			\begin{minipage}[t]{0.31\linewidth}
				\centering
				\includegraphics[width=4.3cm,trim=870 100 180 180,clip]{images/Chapitre3/DoDAlberona1954DSM-SuperGlue.png}
			\end{minipage}%
		}
		\subfigure[\ac{DoD}$_{Alberona}^{Patch_{SpGDSM}}$]{
			\begin{minipage}[t]{0.31\linewidth}
				\centering
				\includegraphics[width=4.5cm,trim=580 30 780 160,clip]{images/Chapitre4/AlberonaDoD1954_Patch_SpGDSM.png}
			\end{minipage}%
		}
		\subfigure[\ac{DoD}$_{Alberona}^{Guided_{SpGDSM}}$]{
			\begin{minipage}[t]{0.31\linewidth}
				\centering
				\includegraphics[width=4.5cm]{images/Chapitre4/AlberonaDoD1954_Guided_SpGDSM.png}
			\end{minipage}%
		}\\
		\subfigure[\ac{DoD}$_{Alberona}^{{SIFTDSM}}$]{
			\begin{minipage}[t]{0.31\linewidth}
				\centering
				\includegraphics[width=4.3cm,trim=870 100 180 180,clip]{images/Chapitre3/DoDAlberona1954DSM-SIFT.png}
			\end{minipage}%
		}
		\subfigure[\ac{DoD}$_{Alberona}^{Patch_{SIFTDSM}}$]{
			\begin{minipage}[t]{0.31\linewidth}
				\centering
				\includegraphics[width=4.5cm,trim=580 30 780 160,clip]{images/Chapitre4/AlberonaDoD1954_Patch_SIFTDSM.png}
			\end{minipage}%
		}
		\subfigure[\ac{DoD}$_{Alberona}^{Guided_{SIFTDSM}}$]{
			\begin{minipage}[t]{0.31\linewidth}
				\centering
				\includegraphics[width=4.5cm]{images/Chapitre4/AlberonaDoD1954_Guided_SIFTDSM.png}
			\end{minipage}%
		}
		
		\subfigure[\ac{DoD} legend]{
			\begin{minipage}[t]{1\linewidth}
				\centering
				\includegraphics[width=11cm]{images/Chapitre4/LegendDoD.png}
			\end{minipage}%
		}
		\caption{{\scriptsize \ac{DoD}s between free epoch \textbf{Alberona 1954} and reference epoch \textbf{2003}. (a) and (d) are \ac{DoD}s resulted from roughly co-registered orientations using methods $SuperGlue_{DSM}$ and $SIFT_{DSM}$ (elaborated in Chapter 3). (b, c, e, f) are \ac{DoD}s resulted from refined orientations using methods $Patch_{SpGDSM}$, $Guided_{SpGDSM}$, $Patch_{SIFTDSM}$ and $Guided_{SIFTDSM}$ individually.}}
		\label{PreciseDoDAlberona}
	\end{center}
\end{figure*} 


%
%\begin{table}%[H]
%	\footnotesize
%	\centering
%	\begin{tabular}{||l|l|c|c|c||}\hline
%		& &$\mu$ [m]&$\sigma$ [m]&$|\mu|$ [m]\\\hline\hline
%%		\multirow{6}{*}{$DoD^{Frejus}_{1954-2014}$}
%%		&${SuperGlue_{ImgPairs}}$ & 5.70 & 6.32 & 6.62\\
%%		&${SuperGlue_{Ortho}}$ & 2.19 & 6.46 & 4.55\\
%%		&${SuperGlue_{DSM}}$ & 2.07 & 4.87 & \textbf{3.83} \\
%%		&${SIFT_{ImgPairs}}$ & / & / & / \\
%%		&${SIFT_{Ortho}}$ & / & / & / \\
%%		&${SIFT_{DSM}}$ & 6.92 & 8.47 & 8.04\\\hline
%%		
%%		\multirow{6}{*}{$DoD^{Frejus}_{1966-2014}$}
%%		&${SuperGlue_{ImgPairs}}$ & -1.36 & 3.82 & 2.90\\
%%		&${SuperGlue_{Ortho}}$ & -0.37 & 4.22 & 3.01\\
%%		&${SuperGlue_{DSM}}$ & -0.46 & 3.77 & \textbf{2.68}\\
%%		&${SIFT_{ImgPairs}}$ & / & / & / \\
%%		&${SIFT_{Ortho}}$ & / & / & / \\
%%		&${SIFT_{DSM}}$ & -1.72 & 4.92 & 3.75\\\hline
%%		
%%		\multirow{6}{*}{$DoD^{Frejus}_{1970-2014}$}
%%		&${SuperGlue_{ImgPairs}}$ & -5.04 & 5.09 & 5.70\\
%%		&${SuperGlue_{Ortho}}$ & -2.63 & 5.18 & \textbf{4.39}\\
%%		&${SuperGlue_{DSM}}$ & -1.71 & 5.75 & 4.61\\
%%		&${SIFT_{ImgPairs}}$ & / & / & / \\
%%		&${SIFT_{Ortho}}$ & / & / & / \\
%%		&${SIFT_{DSM}}$ & -3.17 & 5.44 & 4.71\\\hline
%		
%		%%%%%%%%%%%%%%%%%%%%%%%%satellite
%		\multirow{4}{*}{$DoD^{Pezenas}_{1971-2014(Satellite)}$}
%&${Patch_{SpGDSM}}$ & -0.34 & 4.39 & 2.28\\
%&${Guided_{SpGDSM}}$ & -0.65 & 4.46 & 2.45\\
%&${Patch_{SIFTDSM}}$ & -0.49 & 4.41 & 2.29\\
%&${Guided_{SIFTDSM}}$ & -0.57 & 4.38 & \textbf{2.27}\\\hline
%
%
%%		\multirow{4}{*}{$DoD^{Pezenas}_{1971-2014(Satellite)}$}
%%		%&${SuperGlue_{DSM}}$ & -3.70 & 10.65 & 8.29\\
%%		&${Patch_{SpGDSM}}$ & -0.45 & 4.62 & 2.34\\
%%		&${Guided_{SpGDSM}}$ & -0.61 & 4.75 & 2.47\\
%%		%&${SIFT_{DSM}}$ & -0.68 & 8.11 & 5.80 \\
%%		&${Patch_{SIFTDSM}}$ & -0.35 & 4.70 & \textbf{2.28}\\
%%		&${Guided_{SIFTDSM}}$ & -0.48 & 4.39 & 2.33\\\hline
%				
%%		&${SuperGlue_{Ortho}}$ & / & / & /\\
%%		&${SuperGlue_{DSM}}$ & -3.70 & 10.65 & 8.29\\
%%		%&${SuperGlue_{DSM}}$ & -1.85 & 13.24 & 9.15\\
%%		&${SIFT_{Ortho}}$ & / & / & /\\
%%		&${SIFT_{DSM}}$ & -0.68 & 8.11 & \textbf{5.80} \\\hline
%%		%&${SIFT_{DSM}}$ & 0.58 & 7.82 & \textbf{3.18}\\\hline
%		
%		\multirow{6}{*}{$DoD^{Kobe}_{1991-1995}$}
%		&${Patch_{SpGDSM}}$ & 1.93 & 10.26 & 3.99\\
%		&${Guided_{SpGDSM}}$ & 2.03 & 11.74 & 4.30\\
%		&${Patch_{SIFTDSM}}$ & 1.80 & 10.36 & 4.00\\
%		&${Guided_{SIFTDSM}}$ & 1.84 & 9.48 & \textbf{3.87}\\
%
%%		&${SuperGlue_{ImgPairs}}$ & -1.63 & 13.85 & \textbf{7.24}\\
%%		%&${SIFT_{ImgPairs}}$ & 167.46 & 121.26 & 168.47\\
%%		&${SuperGlue_{Ortho}}$ & -0.54 & 14.83 & 7.78\\
%%		%&${SIFT_{Ortho}}$ & 4.29 & 41.89 & 26.01\\
%%		&${SuperGlue_{DSM}}$ & -0.75 & 14.62 & 7.95\\
%%		&${SIFT_{ImgPairs}}$ & / & / & / \\
%%		&${SIFT_{Ortho}}$ & / & / & / \\
%%		&${SIFT_{DSM}}$ & 0.27 & 14.40 & 7.57\\\hline
%		
%	\end{tabular}
%	\caption{Average value $\mu$, standard deviation $\sigma$, and absolute average value $|\mu|$ of all the \ac{DoD}s in Figure~\ref{DoDFrejus}, ~\ref{DoDPezenas}, ~\ref{DoDPezenas-Satellite} and ~\ref{DoDKobe}.}
%	\label{PreciseDoDStatistic}
%\end{table}


\begin{table}%[H]
	\footnotesize
	\centering
	\begin{tabular}{||l|l|c|c|c||}\hline
		& &$\mu$ [m]&$\sigma$ [m]&$|\mu|$ [m]\\\hline\hline
\multirow{6}{*}{$DoD^{Frejus}_{1954-2014}$}
&${{SpGDSM}}$ & 2.07 & 4.87 & {3.83} \\
&${Patch_{SpGDSM}}$ & -1.91 & 3.24 & 2.69\\
&${Guided_{SpGDSM}}$ & -1.33 & 3.28 & \textbf{2.30}\\
&${{SIFTDSM}}$ & 6.92 & 8.47 & 8.04\\
&${Patch_{SIFTDSM}}$ & -6.28 & 6.72 & 6.62\\
&${Guided_{SIFTDSM}}$ & 3.58 & 16.62 & 12.56\\\hline

\multirow{6}{*}{$DoD^{Frejus}_{1966-2014}$}
&${{SpGDSM}}$ & -0.46 & 3.77 & {2.68}\\
&${Patch_{SpGDSM}}$ & -0.56 & 3.44 & 2.27\\
&${Guided_{SpGDSM}}$ & -0.12 & 3.45 & \textbf{2.19}\\
&${{SIFTDSM}}$ & -1.72 & 4.92 & 3.75\\
&${Patch_{SIFTDSM}}$ & -0.59 & 3.36 & 2.29\\
&${Guided_{SIFTDSM}}$ & 0.14 & 3.35 & 2.25\\\hline

\multirow{6}{*}{$DoD^{Frejus}_{1970-2014}$}
&${{SpGDSM}}$ & -1.71 & 5.75 & 4.61\\
&${Patch_{SpGDSM}}$ & -0.86 & 3.20 & 2.12\\
&${Guided_{SpGDSM}}$ & -0.18 & 3.27 & 2.12\\
&${{SIFTDSM}}$ & -3.17 & 5.44 & 4.71\\
&${Patch_{SIFTDSM}}$ & -0.98 & 3.26 & 2.20\\
&${Guided_{SIFTDSM}}$ & -0.15 & 3.17 & \textbf{2.05}\\\hline		

	\end{tabular}
	\caption{Average value $\mu$, standard deviation $\sigma$, and absolute average value $|\mu|$ of all the \ac{DoD}s in Figure~\ref{PreciseDoDPezenas-Satellite} and ~\ref{PreciseDoDAlberona}.}
	\label{PreciseDoDStatistic}
\end{table}


\subsubsection{Ground displacement}
The northeastward displacement maps of Kobe dataset (hereinafter termed as Gd) as well as the \ac{GT} Gd provided by the Japan meteorological agency~\cite{ian1996morphological} are presented in Figure~\ref{GdKobe}. 
The Gds resulted from rough co-registered orientations using methods $SuperGlue_{DSM}$ and $SIFT_{DSM}$ (i.e. (b) and (e), elaborated in Chapter~\ref{chap:RoughCoReg}) are displayed as bases, and the Gds resulted from refined orientations using methods $Patch_{SpGDSM}$, $Guided_{SpGDSM}$, $Patch_{SIFTDSM}$ and $Guided_{SIFTDSM}$ (i.e. (c, d) and (f, g)) are given for comparison. 
An up-lateral strike-slip movement along the sea is present in Figure~\ref{GdKobe} (c, d) and (f, g), but not in (b) and (e). The observed signal is coherent with the fault of the Kobe earthquake known from \ac{GT}, according to which a striking north $30^{\circ}$-$60^{\circ}$ east rupture occurred along the northeast-southwest coastline across $ \sim $18 km.\\

\begin{figure*}[htbp]
	\begin{center}
		\subfigure[Gd$_{Kobe}^{{GT}}$]{
			\begin{minipage}[t]{1\linewidth}
				\centering
				\includegraphics[width=10cm]{images/Chapitre4/Kobe-faultmap.png}
			\end{minipage}%
		}
		\subfigure[Gd$_{Kobe}^{{SpGDSM}}$]{
			\begin{minipage}[t]{1\linewidth}
				\centering
				\includegraphics[width=10cm,trim=680 330 100 570,clip]{images/Chapitre4/Gd1991_SpGDSM.png}
			\end{minipage}%
		}
		\subfigure[Gd$_{Kobe}^{Patch_{SpGDSM}}$]{
			\begin{minipage}[t]{1\linewidth}
				\centering
				\includegraphics[width=10cm,trim=680 330 100 570,clip]{images/Chapitre4/Gd1991_Patch_SpGDSM.png}
			\end{minipage}%
		}
		\subfigure[Gd$_{Kobe}^{Guided_{SpGDSM}}$]{
			\begin{minipage}[t]{1\linewidth}
				\centering
				\includegraphics[width=10cm,trim=680 330 100 570,clip]{images/Chapitre4/Gd1991_Guided_SpGDSM.png}
			\end{minipage}%
		}\\
		\subfigure[Gd$_{Kobe}^{{SIFTDSM}}$]{
			\begin{minipage}[t]{1\linewidth}
				\centering
				\includegraphics[width=10cm,trim=680 330 100 570,clip]{images/Chapitre4/Gd1991_SIFTDSM.png}
			\end{minipage}%
		}
		\subfigure[Gd$_{Kobe}^{Patch_{SIFTDSM}}$]{
			\begin{minipage}[t]{1\linewidth}
				\centering
				\includegraphics[width=10cm,trim=680 330 100 570,clip]{images/Chapitre4/Gd1991_Patch_SIFTDSM.png}
			\end{minipage}%
		}
		\subfigure[Gd$_{Kobe}^{Guided_{SIFTDSM}}$]{
			\begin{minipage}[t]{1\linewidth}
				\centering
				\includegraphics[width=10cm,trim=680 330 100 570,clip]{images/Chapitre4/Gd1991_Guided_SIFTDSM.png}
			\end{minipage}%
		}
		
		\subfigure[Gd legend]{
			\begin{minipage}[t]{1\linewidth}
				\centering
				\includegraphics[width=11cm]{images/Chapitre4/legend-Gd.png}
			\end{minipage}%
		}
		\caption{{\scriptsize ground displacement(Gd) between free epoch \textbf{Kobe 1991} and reference epoch \textbf{1995}. (a) is the ground truth GD provided by the Japan meteorological agency. (b) and (e) are Gds resulted from roughly co-registered orientations using methods $SuperGlue_{DSM}$ and $SIFT_{DSM}$ (elaborated in Chapter 3). (c, d, f, g) are Gds resulted from refined orientations using methods $Patch_{SpGDSM}$, $Guided_{SpGDSM}$, $Patch_{SIFTDSM}$ and $Guided_{SIFTDSM}$ individually.}}
		\label{GdKobe}
	\end{center}
\end{figure*} 

\section{Discussion}
\subsection{Comparison of precise matching on DSMs and original RGB images}
\label{CompareRGBDSM}
In order to decide which type of images (DSMs or original images) is more suitable for executing the precise matching, we apply our pipeline \textit{Patch} on both DSMs and original images of Fr{\'e}jus 1970 and 2014.
The final matches are displayed in Figure~\ref{precisematchingdepth} (a) and (b). 
To asses quantitatively the results, we created a GT depth map and calculated the accuracy (correct matches / total matches). In Figure~\ref{precisematchingdepth}~(c) we plot the accuracy curves while varying the reprojection error threshold from 0 to 10 pixels. 
It is clear that the result using the original images is more accurate, even though the DSMs recovered more matches.
This is because historical RGB images are inevitably accompanied with noise, and the noise gets worse in \ac{DSM}s at full resolution (see the DSM shaded image in Figure~\ref{precisematchingdepth} (d)) due to the information loss and errors introduced in calculating the \ac{DSM}. Therefore RGB images are more suitable for precise matching.\\
\begin{figure*}[htbp]
	\begin{center}
		\subfigure[Matches on RGB images]{
			\begin{minipage}[t]{0.48\linewidth}
				\centering
				\includegraphics[width=6cm]{images/appendix4/TiePtOriImg.png}
				%\caption{DoD$_{Pezenas\_1971}^{Co-Reg}$}
			\end{minipage}%
		}
		\subfigure[Matches on DSMs]{
			\begin{minipage}[t]{0.48\linewidth}
				\centering
				\includegraphics[width=6cm]{images/appendix4/TiePtDepth.png}
				%\caption{DoD$_{Pezenas\_1971}^{Guided}$}
			\end{minipage}  
		}       
		\subfigure[Accuracy of (a) and (b)]{
			\begin{minipage}[t]{0.58\linewidth}
				\centering
				\includegraphics[width=7cm]{images/appendix4/pdfdepth.png}
				%\caption{DoD$_{Pezenas\_1981}^{Co-Reg}$}
			\end{minipage}%
		}
		\subfigure[Shaded image of historical DSM]{
			\begin{minipage}[t]{0.38\linewidth}
				\centering
				\includegraphics[width=5cm]{images/appendix4/DepthShade.png}
				%\caption{DoD$_{Pezenas\_1981}^{Co-Reg}$}
			\end{minipage}%
		}
		\caption{Comparison of precise matching on original RGB images and DSMs.}
		\label{precisematchingdepth}
	\end{center}
\end{figure*} 

\subsection{Comparison between SIFT and SuperGlue}
SuperGlue is more invariant over time than SIFT, leading to more numerous matches in the precise matching stage. However, combined with reliable rough co-registration result, SIFT is able to recover less numerous yet more accurate matches than SuperGlue.

\section{Conclusion}
In this section we elaborated two pipelines for precise matching: $Patch$ and $Guided$. 
We tested each pipeline on two sets of rough co-registration basics: $SIFT_{DSM}$ and $SuperGlue_{DSM}$, which leads to 4 methods (\ding{172} $Patch_{SpGDSM}$, \ding{173} $Guided_{SpGDSM}$, \ding{174} $Patch_{SIFTDSM}$ and \ding{175} $Guided_{SIFTDSM}$.
We applied all the 4 methods on 4 datasets (Fr{\'e}jus, Pezenas, Kobe and Alberona), including the cases of (1) matching aerial epochs only and (2) matching aerial and satellite epochs mixed.
Experiments showed that:\\
\begin{enumerate}
	\item Both precise matching pipelines (i.e. $Patch$ and $Guided$) are capable of recovering numerous and accurate matches, as long as the rough co-registration basic is reliable.\\
	\item By adopting the precise matches in a \ac{BBA} routine, the systematic errors in the surfaces are effectively mitigated while the real scene changes are reserved.
	\item Our pipelines are able to provide precise results for variable application cases including general change detection, landslides and earthquakes.
\end{enumerate}
%\section{Discussion}

