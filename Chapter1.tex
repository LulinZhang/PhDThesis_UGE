%!TEX root = Manuscript.tex

\chapter{Introduction}
\label{chap:intro}
\minitoc

\section{Motivation}
Historical (i.e. analogue or archival) aerial images play an important role in providing unique information about evolution of land-covers. 
They are objective witness over time and sometimes the only remaining visual source of historical land-form. Therefore, they are valuable assets for a wide range of applications such as: (1) change detection, (2) spatial and urban planning, (3) long-term environmental monitoring including but not limited to forests, ice glaciers and coastlines, (4) analysis of natural disaster (e.g. earthquake, volcano eruption, landslide ect.) and the estimation of its future trends, etc.
%用途:地震,冰川,土地规划?
%long-term environmental monitoring: forests, ice glaciers, coastlines etc.
%change detection; 
%spatial and urban planning, boundary disputes, etc.
%enable the analysis of natural disaster (e.g. earthquake, volcano eruption, landslide ect.) and the estimation of its future trends.
\par
Historical aerial images have regularly been acquired since the 1920’s by mapping, military or cadastral agencies all over the world. A mass amount of them have been digitized and made accessible through web services~\cite{sebastien2019archiving,earthexplorer,remonterletemps}. 
For example, according to a survey in Europe~\cite{sebastien2019archiving}, there are in total 50 million of aerial images archived, with around 37.8\% of them digitized.
%A survey called "State of current practices concerning archival aerial images" in Europe~\cite{sebastien2019archiving} took place in 2017, 19 organizations from 13 countries (including Austria, Norway, Finland, Spain, Slovenia, Cyprus, Czech Republic, Finland, Germany, Switzerland, Sweden, Poland, France and United Kingdom) replied. According to it, there are in total 50 million of aerial images archived, with around 37.8\% of them digitized.
The images are of high spatial resolution, and are acquired in stereoscopic configuration, allowing for 3D restitution of territories. 
They are often accompanied by metadata, in most cases including the camera focal length and the physical sensor size. Other metadata such as flight plans, camera calibration certificates or orientations are not commonly available. 
\par
When the camera calibration parameters are unknown, they should be evaluated by a process called self-calibrating bundle adjustment. Adequate Ground Control Points (GCPs) are required, otherwise inaccurately estimated camera parameters will lead to systematic error surfaces called dome effect (i.e. bowel effect).
Generally, GCPs originate from (i) field surveys \cite{micheletti2015application,walstra2004time,cardenal2006use}, (ii) recent orthophotos and DEM \cite{nurminen2015automation,ellis2006measuring,fox2008unlocking} and (iii) recent satellite images \cite{ellis2006measuring,ford2013shoreline}. The most challenging part is to identify the GCPs on the historical images, due to inevitable scene changes. GCPs are usually manually measured with the help of recent photos, however, it is still monotonous and time-consuming. 
There is an urgent need to automatically identify corresponding points (i.e. correspondences) on historical and recent images. However, recovering correspondences on images taken at different times (i.e. multi-epoch) remains challenging for the following reasons:
\begin{itemize}
	\item Multi-epoch images are often acquired at different times of day and in different weathers or seasons, which unavoidably leading to appearance differences.
	\item The scene changes over time due to anthropogenic phenomenas (e.g. urban planning) or natural ones (e.g. earthquake), especially for large time gaps.
	\item Multi-epoch images often exhibit heterogeneous spatial resolutions, accompanied with different acquisition conditions (sensors, spectral channels, etc.).
	\item Historical images are often facing low radiometric quality, including low contrast, image noise, deterioration due to the aging of the films, or even scratches on the films.
\end{itemize}
%困难:
%(1) 数据异构: images exhibit highly heterogeneous spatial resolutions, with very different acquisition conditions (season, sensors etc.); 
%(2) camera's parameters  may not be available as they might  be  lost  during  the  years  or  never documented; inappropriate film/glass plate preservation; deformation due to scanning process; 
%(3)Drastic scene changes
%(4)Low radiometric quality: old-fashioned radiometric characteristics; noisy; even scratchs on the original films; deterioration due to the aging of the film

%Corresponding points 
%Besides, inappropriate film/glass plate preservation and the scanning process enforce reestimating of the camera calibrations (i.e., the self-calibrating).\\


\section{Objective}

\section{Contribution}

\section{Organization of the thesis}
